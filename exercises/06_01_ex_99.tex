{In the text, we integrated $\ds \int \sin x \cos x\ dx$ using $u=\sin x$ to obtain the answer $\ds \frac{1}{2}\sin^2 x+C$.\label{exer:06_01_ex_99}
\begin{enumerate}
\item What answer to you obtain by choosing $u=\cos x$?
\item Why are the two answers actually the same, even if they look different?
\end{enumerate}
}
{\begin{enumerate}
\item $-\frac{1}{2}\cos^2 x+C$
\item By the identity $\cos^2 x+\sin^2 x=1$, we see that $\ds -\frac{1}{2}\cos^2 x=-\frac{1}{2}\left(1-\sin^2 x\right)=1+\frac{1}{2}\sin^2 x$ so the answers differ by a constant.  That difference is absorbed into the arbitrary constant $C$.
\end{enumerate}
}

