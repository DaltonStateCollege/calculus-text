{Using the result in Example~\ref{ex_ps7}, explain why $F_n$ is always $\frac{1}{\sqrt{5}} \left( \frac{1+\sqrt{5}}{2} \right)^n$ rounded to the nearest integer.
}
{Note $F_n=\frac{1}{\sqrt{5}}\left(\left(\frac{1+\sqrt{5}}{2}\right)^n - \left(\frac{1-\sqrt{5}}{2}\right)^n\right)$ differs from $\frac{1}{\sqrt{5}} \left( \frac{1+\sqrt{5}}{2} \right)^n$ by $\frac{1}{\sqrt{5}}\left(\frac{1-\sqrt{5}}{2}\right)^n$.  Since $-1<\frac{1-\sqrt{5}}{2}<0$, it follows $\left|\left(\frac{1-\sqrt{5}}{2}\right)^n\right|\leq \frac{1}{\sqrt{5}}< 0.5$.  By the way $F_n$ is defined, $F_n$ must be an integer for all $n$.  There is only one integer that is less than $0.5$ away from a given real number; this is the nearest integer.
}