\section{Green's Theorem}\label{sec:greens_theorem}

This section contains the Fundamental Theorem of Calculus extended to two dimensions. The formula was discovered 150 years after Newton and Leibniz, by a self-taught English mathematician named George Green. His theorem connects a double integral over a region $R$ to a line integral along its boundary curve $C$.\\

In Section \ref{sec:FTC} you learn that the integral of $\dfrac{df}{dx}$ from $x=a$ to $x=b$ equals $f(b) - f (a)$. This connects a one-dimensional integral to a zero-dimensional integral - the value $f(b) - f(a)$ is some kind of a point integral. It is this absolutely crucial idea (to integrate a derivative from information at the boundary) that extends Green's Theorem into two dimensions.\\

There are two important integrals around a curve $C$. The first is for a force field $\vec F$, and the work done is $\displaystyle\int_C \vec F \cdot \vec T \: ds = \displaystyle\int_C M \: dx + N \: dy.$  The second is for a flow field $\vec F$, and the flux is $\displaystyle\int_C \vec F \cdot \vec n \: dx = \displaystyle\int_C M \: dy - N \: dx$, where $\vec n$ is the normal vector. Green's Theorem handles both, in two dimensions. In three dimensions they split into the Divergence Theorem (Section \ref{sec:div_theorem}) and Stokes' Theorem (Section \ref{sec:stokes_theorem}). Green's Theorem applies to "smooth" functions $M(x, y)$ and $N(x, y)$ with continuous first derivatives in a region containing the interior $R$ inside the curve $C$. $M(x,y)$ and $N(x,y)$ will have a specific meaning in applications (to electricity and magnetism or to fluid flow or to mechanics). We capture the central idea first and the applications follow.

As Green's Theorem is about simple, closed curves, we first define some terminology.\\

\definition{def:simple_closed_ccw}{Simple, Closed Curves, and the Counterclockwise Direction}
{Recall that a curve is \textbf{closed} if its starting and ending points are the same.  A curve is \textbf{simple} if it does not intersect itself, except possibly at the starting and ending points.  Any simple, closed curve $C$ encloses a region $R$.  Suppose an ant is crawling along the curve $C$.  The \textbf{counterclockwise} direction is the one in which the enclosed region $R$ is always on the ant's left.
}\\

We are familiar with the term ``counterclockwise'' for circles.  This allows us to define it for any simple, closed curve, such as the wacky curve in Figure~\ref{fig:ccw_example}.\\

\mfigure[scale=0.39]{.314}{A simple closed curve.}{fig:ccw_example}{figures/counter_clockwise_example}

\theorem{thm:greens_theorem}{Green's Theorem}
{Suppose the region $R$ is bounded by a simple, closed, piecewise smooth curve $C$ oriented in the counterclockwise direction. If $M(x,y)$ and $N(x,y)$ have continuous first-order partial derivatives on an open region containing $R$, then
$$\oint_C M(x,y) \: dx + N(x,y) \: dy = \iint_R \dfrac{\partial N}{\partial x} - \dfrac{\partial M}{\partial y} \: dA$$
}\\

\example{ex_greenstheorem1}{}{First consider the case that $\vec F(x,y)$ is a conservative force field with a potential function $f(x,y)$ so that $\nabla f = \vec F$. Let $C$ be a simple, closed, piecewise smooth curve enclosing a region $R$. What does Green's Theorem say about the work done around $C$ in this case?}{Since $\vec F = M \: \vec i + N \: \vec j$ is conservative, we get that
$$\dfrac{\partial N}{\partial x} = \dfrac{\partial M}{\partial y}$$
from the last section.  Then Green's Theorem implies
$$\oint_C M(x,y) \: dx + N(x,y) \: dy = \iint_R 0 \: dA = 0.$$
This verifies the claim in the last section that line integrals of conservative vector fields around closed curves are zero.
}\\

\example{ex_greenstheorem3}{}{Compute the line integral $\oint_C x^3 \: dx + xy^2 \: dy$ where $C$ is the triangular curve consisting of line segments connecting $(0,0)$ to $(1,0)$, $(1,0)$ to $(1,1)$, and $(1,1)$ to $(0,0)$.}{Computing this directly would require three different line integrals and three different parametrizations for the line segments.  Instead we employ Green's Theorem and compute a single double integral.  Letting $R$ be the triangle inside the curves bounded by $y = 0$ and $y = x$, we get
$$\oint_C x^3 \: dx + xy^2 \: dy = \iint_R (y^2 - 0) \: dA = \int_0^1 \int_{0}^{x} y^2 \: dy \: dx = \dfrac{1}{12}.$$
}\\

\noindent\textbf{\large Proof of Green's Theorem}\\

Green's Theorem is not easy to prove in general.  We will, however, verify it is true in the case that $R$ is a simple region - one that is expressible in both the forms
$$R = \left\{ (x,y) \mid a \leq x \leq b, g_1(x) \leq y \leq g_2(x) \right\}$$
for some curves $y = g_1(x)$ and $y = g_2(x)$, and
$$R = \left\{ (x,y) \mid c \leq y \leq d, h_1(y) \leq x \leq h_2(y) \right\}$$
for some curves $x = h_1(y)$ and $x = h_2(y)$. Other regions which are the union of simple regions can then be done with this in hand.\\

For a vector field $\vec F(x,y) = M(x,y) \: \vec i + N(x,y) \vec j$ as above, the idea is to prove that
$$\oint_C M(x,y) \: dx = \iint_R -\dfrac{\partial M}{\partial y} \: dA$$
and
$$\oint_C N(x,y) \: dy = \iint_R \dfrac{\partial N}{\partial x} \: dA$$
separately. We verify the first and the leave the second, which is very similar, as an exercise.\\

Suppose that $R$ is the region given by
$$\left\{ (x,y) \mid a \leq x \leq b, g(x) \leq y \leq f(x) \right\}$$
for top curve $y = f(x)$ and bottom curve $y = g(x)$.  In the double integral, first integrate with respect to $y$ to get
$$\int_{g(x)}^{f(x)} -\dfrac{\partial M}{\partial y} \: dy = \Big( -M(x,y) \Big]_{y=g(x)}^{y=f(x)} = -M(x,f(x)) + M(x,g(x)).$$
Integrate with respect to $x$ to yield
$$-\int_a^b M(x,f(x)) \: dx + \int_a^b M(x,g(x)) \: dx.$$
The other side of the equation deals with a line integral. For this we get
$$\oint_C M(x,y) \: dx = \int_{\text{top}} M(x,y) \: dx + \int_{\text{bottom}} M(x,y) \: dx = \int_b^a M(x,f(x)) \: dx + \int_a^b M(x,g(x)) \: dx$$
which is the same as the other side due to the addition of a minus sign when the bounds of the first integral are switched.\\

\mfigure[scale=0.5]{.7}{Union of simple regions}{fig:fig_greens_thm_comp_region}{figures/fig_greens_thm_comp_region}

Next consider a region $R$ which is not simple but is instead a union of simple regions, such as in Figure \ref{fig:fig_greens_thm_comp_region}. As in the figure, we would break the region into three pieces $R_1$, $R_2$, and $R_3$ which are all simple. The three individual double integrals sum to the double integral of $R$, which would also equal the sum of the three line integrals by the above argument. When we add the line integrals, the line integrals over the cross cuts cancel out as they are going in opposite directions, resulting in only the boundary pieces adding up to the total line integral.  This leaves the double integral of the interior equal to the line integral of just the boundary, as Green's Theorem would state. The next example will pertain to a region that is a union of two simple regions. \\

If the region $R$ contains the piece $R_4$ in Figure \ref{fig:fig_greens_thm_comp_region}, then the theorem is still true. The
integral around the outside is still counterclockwise, but the integral is clockwise around the inner circle. Keeping the region $R$ to your left as you go around $C$ gives the counterclockwise direction according to our definition. The complete ring is doubly connected, not simply connected. Green's Theorem allows any finite number of regions $R_i$ and cross cuts and holes.\\

\example{ex_greenstheorem4}{}{Compute the line integral $\oint_C y^2 \: dx + 3xy \: dy$ where $C$ is the boundary of the semiannular region in the upper half-plane between $x^2 + y^2 = 1$ and $x^2 + y^2 = 4$.}{The enclosed region $R$ is not simple, but is a union of two simple pieces if you cut down the $y$-axis. Therefore Green's Theorem applies in this situation.  The theorem gives
$$\oint_C y^2 \: dx + 3xy \: dy = \iint_R 3y - 2y \: dA = \iint_R y \: dA.$$
Notice that this region can be integrated over easily via polar coordinates.  Therefore
$$\oint_C y^2 \: dx + 3xy \: dy = \int_0^{\pi} \int_{1}^{2} (r \sin\theta) r \: dr \: d\theta = \dfrac{14}{3}.$$
}\\\clearpage

\noindent\textbf{\large Usng Green's Theorem to Compute Areas}
\vskip\baselineskip

So far, we have used Green's Theorem to rewrite line integrals as double integrals, making them easier to compute.  However, Green's Theorem gives equality between a line integral and a double integral, and therefore can also be used to rewrite a double integral as a line integral.  This can be done for any double integral, but we wll only focus on $\iint_R 1 dA$, the area of a region $R$.

Consider the region $R$ in Figure~\ref{fig:fig_greens_thm_int_curve}.\mfigure[scale=0.9]{.7}{The region $R$ is bounded inside the simple closed curve $x=\cos(3t) - \sin t$, $y=2\cos t$}{fig:fig_greens_thm_int_curve}{figures/integral_curve}  This region is inside the simple closed curve given by the parametric equations $$x=\cos(3t) - \sin t \text{ and } y=2\cos t$$ traced once on the interval $[0,2\pi]$.  Note that determining bounds for this region in terms of $x$ andor $y$, as we would typically do to compute rewrite the double integral as an iterated integral, cannot be done in a simple way.  However, as we have a parameterization for the boundary curve, it is not hard to compute a line integral along that curve.

Consider any vector field $\vec F(x,y) = M \: \vec i + N \: \vec j$ for which $\dfrac{\partial N}{\partial x}-\dfrac{\partial M}{\partial y} = 1$.  There are infinitely many such vector fields, the most basic of which are $\vec F(x,y) = x \: \vec j$ or $\vec F(x,y) = -y \: \vec i$.  What is the line integral of $\vec F$ around a simple closed curve $C$ computing? By Green's Theorem
$$\oint_C \vec F \cdot d\vec r = \oint_C M(x,y) \: dx + N(x,y) \: dy = \iint_R 1 \: dA$$
which is the area of $R$, or the area enclosed by $C$.  In this way, one can use a line integral to compute area as long as one knows the boundary curve of the region.

We summarize this using a Key Idea before we continue with examples.

\keyidea{idea:green_area}{Area inside a simple closed curve $C$}{Let $R$ be the region inside a simple closed curve $C$, parameterized in the counterclockwise direction.  Let $\vec F(x,y) = M \: \vec i + N \: \vec j$ be any vector field for which $N_x - M_y = 1$  Then
$$\text{Area of }R = \iint_R 1\ dA = \oint_C M(x,y) \: dx + N(x,y) \: dy.$$
In particular,
$$\text{Area of }R = \iint_R 1\ dA = \oint_C x\: dy = - \oint_C y\: dx.$$
}\\

\example{ex_greenstheorem9}{Computing area using Green's Theorem}{Compute the area of the region $R$ inside the simple closed parametric curve $C$ given by $$x=\cos(3t) - \sin t \text{ and } y=2\cos t$$ traced once on the interval $[0,2\pi]$.} {This is the region shown in Figure~\ref{fig:fig_greens_thm_int_curve}.  The given parameterization gives the counterclockwise direction.  (If we did not know that, however, and computed a line integral in the opposite direction, it would give the opposite value.  Since an area must be positive, we could take the absolute value in which case the direction of the curve would not matter.)  Using Key Idea~\ref{idea:green_area},
\begin{align*}
\iint_R 1\ dA &= \oint_C x\: dy\\
&= \int_0^{2\pi} \big( \cos(3t) - \sin t \big) (-2\sin t)\ dt\\
&=-2 \int_0^{2\pi} \big( \sin t \cos(3t) - \sin^2 t \big)\ dt.
\end{align*}
Recall now that we must proceed with the Product-to-Sum identities,
\begin{align*}
\iint_R 1\ dA &= -2 \int_0^{2\pi} \big( \sin t \cos(3t) - \sin^2 t \big)\ dt\\
&=-2 \int_0^{2\pi} \left(  \frac{\sin(4t) + \sin(-2t)}{2} - \frac{1-\cos(2t)}{2}   \right)\ dt\\
&= \int_0^{2\pi} \big(  -\sin(4t) +\sin (2t) + 1 - \cos(2t)  \big)\ dt\\
&= \left. \frac{\cos(4t)}{4} - \frac{\cos(2t)}{2} + t - \frac{\sin(2t)}{2}\right|_0^{2\pi}\\
&=2\pi \text{ square units.}
\end{align*}
}\\

The ability to compute an area using Green's Theorem has physical value as well.  A \textbf{planimeter} is a device used to compute the area of an arbitrary two-dimensional shape.  It operates by tracing the boundary of the shape, and computing an appropriate line integral while tracing the boundary.

We finish this section with a more familiar shape.
\clearpage

\example{ex_greenstheorem2}{Area inside an ellipse}{Using a line integral, compute the area of the ellipse
$$\dfrac{x^2}{a^2} + \dfrac{y^2}{b^2} = 1$$
with semiaxes of length $a$ and $b$.}{\mfigure[scale=0.4]{.79}{Ellipse}{fig:fig_greens_thm_ellipse}{figures/fig_greens_thm_ellipse}
See Figure \ref{fig:fig_greens_thm_ellipse}.  By the above equation for the ellipse, we can parameterize the ellipse as
$$x = a \cos(t), \: \: y = b \sin(t)$$
for $0 \leq t \leq 2\pi$, which traces the ellipse once in the counterclockwise direction.  Using Key Idea~\ref{idea:green_area}, we get an area of
\begin{align*}
\iint_R 1\ dA &= \oint_C x \: dy \\
&= \int_0^{2\pi} a \cos(t) \: (b \cos(t)) \: dt \\
&= ab \int_0^{2\pi} \cos^2(t) \: dt \\
&= ab  \int_0^{2\pi} \frac{1+\cos(2t)}{2} \: dt \\
&= ab \left.\left(\frac{t}{2} + \frac{\sin(2t)}{4}\right)\right|_0^{2\pi}\\
&=\pi ab \text{ square units.}
\end{align*}
}\\

One can consider a region $R$ in the plane as a flat surface in three-dimensional space. Green's Theorem says that we can compute an integral over the surface $R$ by instead computing a line integral around the boundary of the region.  This will hold for more general surfaces in space, and the result that generalizes Green's Theorem is called Stokes' Theorem, discussed in Section \ref{sec:stokes_theorem}.

\printexercises{exercises/14_03_exercises}