\section{Basic Differentiation Rules}\label{sec:basic_diff_rules}

The derivative is a powerful tool but is admittedly awkward given its reliance on limits. Fortunately, one thing mathematicians are good at is \textit{abstraction.} For instance, instead of continually finding derivatives at a point, we abstracted and found the derivative function. 

Let's practice abstraction on linear functions, $y=mx+b$. What is $y\primeskip'$? Without limits, recognize that linear function are characterized by being functions with a constant rate of change (the slope). The derivative, $y\primeskip'$, gives the instantaneous rate of change; with a linear function, this is constant, $m$. Thus $y\primeskip'=m$. 

Let's abstract once more. Let's find the derivative of the general quadratic function, $f(x) = ax^2+bx+c$. Using the definition of the derivative, we have:
		\begin{align*}
		\fp(x) 	&=	\lim_{h\to 0}\frac{a(x+h)^2+b(x+h)+c-(ax^2+bx+c)}{h} \\
						&=	\lim_{h\to 0} \frac{ah^2+2ahx+bh}{h} \\
						&=	\lim_{h\to 0} ah+2ax+b\\
						&= 2ax+b.
		\end{align*}
		
So if $y = 6x^2+11x-13$, we can immediately compute $y\primeskip' = 12x+11$. \\

In this section (and in some sections to follow) we will learn some of what mathematicians have already discovered about the derivatives of certain functions and how derivatives interact with arithmetic operations.\\

\noindent\textbf{\large Power Rule}\\

From the rule above, if $f(x)=x^2$, then $f(x)=1x^2+0x+0$, so $\fp(x) = 2x+0=2x$.  Now we will do two more examples for derivatives of functions of the form $y=x^n$ to see if we notice a pattern.

Let's find the derivative of $f(x) = x^4$.  Again using the derivative definition, we have 
\begin{align*}
		\fp(x) 	&=	\lim_{h\to 0}\frac{(x+h)^4 - x^4}{h} \\
						&=	\lim_{h\to 0} \frac{x^4+4x^3h + 6x^2h^2 + 4xh^3 + h^4 - x^4}{h} \\
						&=	\lim_{h\to 0} 4x^3 + 6x^2h + 4xh^2 + h^3\\
						&= 4x^3.
		\end{align*}

And now we will try an example with a negative $n$.  If $\ds f(x)=\frac{1}{x} = x^{-1}$, then
\begin{align*}
		\fp(x) 	&= \lim_{h\to 0} \frac{\frac{1}{x+h}-\frac{1}{x}}{h}\\
&=	\lim_{h\to 0} \frac{1}{h}\left(\frac{1}{x+h}-\frac{1}{x}\right)\\						
&= \lim_{h\to 0} \frac{1}{h}\cdot\left(\frac{x}{x(x+h)} - \frac{x+h}{x(x+h)}\right)\\
						&=	\lim_{h\to 0} \frac 1h\cdot\left(\frac{x-(x+h)}{x(x+h)}\right)\\
	%\end{align*}%\\
	%\begin{align*}
						&=	\lim_{h\to 0} \frac1h\cdot\left(\frac{-h}{x(x+h)}\right)\\
						&=	\lim_{h\to 0} \frac{-1}{x(x+h)} \\
						&= \frac{-1}{x^2}
	\end{align*}

So we have the following.
\begin{align*}
\frac{d}{dx}\left(x^2\right) &= 2x = 2x^1\\
\frac{d}{dx}\left(x^4\right) &= 4x^3\\
\frac{d}{dx}\left(x^{-1}\right) &= \frac{-1}{x^2} = -1x^{-2}
\end{align*}

This suggests a pattern for the derivatives of Power Functions (of the form $y=x^n$): multiply by the power, then subtract 1 from the power.  This holds in general, even when $n$ is not an integer (such as $x^{1/2}$ or $x^\pi$).  We will not prove this in general, as there are tedious details involved.

\theorem{thm:finalpower}{Power Rule for Differentiation}
{Let $f(x) = x^n$, where $n$ is a real number. Then $\fp(x) = n\cdot x^{n-1}$.\index{derivative!Power Rule}\index{Power Rule!differentiation}
}

\enlargethispage{1\baselineskip}
Let's practice using this theorem.\\

\example{ex_deriv_rule1}{Using the Power Rule to find and use derivatives}
{Let $f(x)=x^3$. 

		%\noindent\begin{minipage}[t]{.5\textwidth}
		\begin{enumerate}
		\item		Find $\fp(x)$.
		\item		Find the equation of the line tangent to the graph of $f$ at $x=-1$. 
%		\end{enumerate}
%		\end{minipage}
%		\begin{minipage}[t]{.5\textwidth}
%		\begin{enumerate}
		\item		Use the tangent line to approximate $(-1.1)^3$.
		\item		Sketch $f$, $\fp$ and the found tangent line on the same axis.
		\end{enumerate}
%		\end{minipage}
}
{	\begin{enumerate}
\enlargethispage{\baselineskip}
		\item		The Power Rule states that if $f(x) = x^3$, then $\fp(x) = 3x^2$. 

		\item		To find the equation of the line tangent to the graph of $f$ at $x=-1$, we need a point and the slope. The point is $(-1,f(-1)) = (-1, -1)$. The slope is $\fp(-1)= 3$. Thus the tangent line has equation $y = 3(x-(-1))+(-1) = 3x+2$. 
		
		\item		We can use the tangent line to approximate $(-1.1)^3$ as $-1.1$ is close to $-1$. We have $$(-1.1)^3 \approx 3(-1.1)+2 = -1.3.$$
			We can easily find the actual answer; $(-1.1)^3 = -1.331$. 
		
		\item		See Figure \ref{fig:xcubedwithderiv}.
		\mfigure{.5}{A graph of $f(x) = x^3$, along with its derivative $\fp(x) = 3x^2$ and its tangent line at $x=-1$.}{fig:xcubedwithderiv}{figures/figxcubedwithderiv}
		\end{enumerate}
\vskip-1.5\baselineskip
}\\

One special case of the Power Rule is when $n=1$, i.e., when $f(x) = x$. What is $\fp(x)$? According to the Power Rule, $$\fp(x) = \frac{d}{dx}\big(x\big) = \frac{d}{dx}\big(x^1\big) = 1\cdot x^0 = 1.$$ In words, we are asking ``At what rate does $f$ change with respect to $x$?'' Since $f$ \textit{is} $x$, we are asking ``At what rate does $x$ change with respect to $x$?'' The answer is: 1. They change at the same rate.

Note that we can use the Power Rule to determine derivatives of functions that are not typically written in the form $y=x^n$.  Recall that $x^{m/n}$ means $\sqrt[n]{x^m}$.  We use this to determine the derivative of $f(x)=\sqrt{x}$ in the next example.\\

\example{ex_deriv_rule1a}{Using the Power Rule to find and use derivatives}
{Let $f(x)=\sqrt{x}$. 

		%\noindent\begin{minipage}[t]{.5\textwidth}
		\begin{enumerate}
		\item		Find $\fp(x)$.
		\item		Find the equation of the line tangent to the graph of $f$ at $x=16$. 
%		\end{enumerate}
%		\end{minipage}
%		\begin{minipage}[t]{.5\textwidth}
%		\begin{enumerate}
		\item		Use the tangent line to approximate $\sqrt{19}$ and $\sqrt{50}$.  Determine how far off the approximations are.
		\end{enumerate}
%		\end{minipage}
}
{	\begin{enumerate}
\enlargethispage{\baselineskip}
		\item		To use the Power Rule, we first rewrite the function in an equivalent form $f(x)=\sqrt{x}=x^{1/2}$.  Then the Power Rule states that if $$ \fp(x) = \frac{1}{2}x^{1/2-1}=\frac{1}{2}x^{-1/2}.$$  We can simplify $$\fp(x)= \frac{1}{2}x^{-1/2}=\frac{1}{2x^{1/2}}=\frac{1}{2\sqrt{x}}.$$

		\item		To find the equation of the line tangent to the graph of $f$ at $x=16$, we again need a point and the slope. The point is $(16,f(16))=(16,4)$. The slope is $\ds \fp(16)= \frac{1}{2\sqrt{16}}=\frac{1}{8}$.
Thus the tangent line has equation $$y = \frac{1}{8}(x-16)+ 4.$$
		
		\item		We can use the tangent line to approximate $$\sqrt{19}\approx \frac{1}{8}(19-16)+4 = 4.375$$ and $$\sqrt{50}\approx \frac{1}{8}(50-16)+4 = 8.25$$  From a calculator, we see that $\sqrt{19}\approx 4.359$ but $\sqrt{50}\approx 7.071$.  The tangent line gives a good approximation of $\sqrt{19}$, and a poor approximation of $\sqrt{50}$.  This is what we expect.  We used the tangent line at $x=16$, so the farther $x$ is from 16, the worse we expect the approximation to be.
		\end{enumerate}
\vskip-1.5\baselineskip
}\\

\noindent\textbf{\large Derivatives of Common Functions}\\

In Section~\ref{sec:derivative}, we found that the derivative of $\sin x$ is $\cos x$.  That proof was a bit tricky, using a special limit that was not straightforward to prove.  There are many rules.  We begin with the derivatives of some of the most common functions in mathematics.\\

%\setboxwidth{110pt}
%\noindent\hskip-110pt
%\begin{minipage}{\specialboxlength}
\theorem{thm:deriv_common}{Derivatives of Common Functions}{
		\begin{enumerate}
		\item		\sword{Constant Rule:}\index{derivative!Constant Rule} 
		$\ds \frac{d}{dx}\big( c\big) = 0$, where $c$ is a constant.%\addtocounter{enumi}{1}
\item		\sword{Power Rule:}\index{derivative!Power Rule}\index{Power Rule!differentiation}	
		$\ds \frac{d}{dx}\left(x^n\right) = nx^{n-1}$%.\addtocounter{enumi}{1}
		\item		$\ds \frac{d}{dx}(\sin x) = \cos x$%\addtocounter{enumi}{1}
		\item		$\ds \frac{d}{dx}(\cos x) = -\sin x$%\addtocounter{enumi}{1}
		\item		$\ds \frac{d}{dx}\left(e^x\right) = e^x$
		\item		$\ds \frac{d}{dx}\left(a^x\right) = \ln a\cdot a^x$, where $a>0$.
		\item		$\ds \frac{d}{dx}(\ln x) = \frac{1}{x}$
		\end{enumerate}\index{derivative!basic rules}
}
%\end{minipage}
%\restoreboxwidth

This theorem starts by stating an intuitive fact: constant functions have no rate of change as they are \textit{constant}.

We see something incredible about the function $y=e^x$: it is its own derivative.  But why does this function's derivative \textbf{not}
match up with the Power Rule's answer $xe^{x-1}$?  Recall that the derivative represents the function's rate of change as the variable $x$ changes.  So we cannot use the Power Rule when the variable is in the exponent.

Also we see here what really makes the number $e$ special, and one of the most important numbers in mathematics.  Many students have an easier time wrapping their head around a function such as $y=2^x$ as opposed to $y=e^x$.  However, $e$ is ``natural'' from the perspective of rates of change.  Not only is $e^x$ its own derivative, but also the derivative of $f(x)=2^x$ is $\fp(x)=\ln 2\cdot 2^x$.  This has the number $e$ in it, in the form of log base $e$.
The derivative rule for $y=e^x$ can be remembered as a special case of the $y=a^x$ rule, since $\ln e=1$.  However,
the base $e$ for exponential and logarithmic appears far more often than any other base, and the function $y=e^x$ is encountered far more often.
Any exponential function can be rewritten in terms of any base.  For example, $2=e^{\ln 2}$, so $2^x=\left(e^{\ln 2}\right)^x = e^{\ln(2) x}$.
In Section~\ref{sec:chainrule}, we will use this to prove the formula for $\frac{d}{dx}(a^x)$ from that of $e^x$.
In exponential growth and decay models, it is common and simpler to work only in the base $e$.

Note that, from this rule, $\frac{d}{dx}(1^x)=\ln 1\cdot 1^x = 0\cdot 1^x = 0$.  This makes sense because $1^x$ always equals 1, so it is a constant function.

In the next example, we prove $e^x$ is its own derivative.  Note that this proof uses the fact that $\ds \lim_{x\to 0} \frac{e^x - 1}{x}=1$ from Theorem~\ref{thm:special_limits} in Section~\ref{sec:limit_analytically}, another special property of the number $e$.\\

\example{ex_deriv_e}{Proving the derivative of $f(x)=e^x$ is $\fp(x)=e^x$}{
Verify from the definition that the derivative of $f(x)=e^x$ is $\fp(x)=e^x$.}
{		%\small
		\begin{align*}
		\fp(x) &= \lim_{h\to 0} \frac{e^{x+h}-e^x}{h}\\ %& & %\left(\text{ \parbox{70pt}{\centering Use rules of exponents.}}\right) \\
						&= \lim_{h\to 0} \frac{e^x e^h - e^x}{h}\\ %& & \\
						&= \lim_{h\to 0} \frac{e^x \left(e^h - 1 \right)}{h}\\% & & %\text{\scriptsize We can factor out $e^x$ from the limit, since it does not have $h$ in it.)}\\
						&=e^x \lim_{h\to 0} \frac{e^h - 1}{h}\\ %& & \left(\parbox{70pt}{ Use $\ds \lim_{h\to 0} \frac{e^h-1}{h} = 1$}\right)\\
						&=	e^x \cdot 1 \\
						&= e^x
		\end{align*}
		%\normalsize
}\\


Theorem \ref{thm:deriv_common} gives useful information, but we will need much more. For instance, using the theorem, we can easily find the derivative of $y=x^3$, but it does not tell how to compute the derivative of $y=2x^3$, $y=x^3+\sin x$ nor $y=x^3\sin x$. The following theorem helps with the first two of these examples (the third is more complicated and is answered in the next section).
\enlargethispage{\baselineskip}

\theorem{thm:deriv_prop}{Properties of the Derivative}
{Let $f$ and $g$ be differentiable on an open interval $I$ and let $c$ be a real number. Then:
	\begin{enumerate}
	%\item		\parbox{75pt}{\textbf{Sum/Difference Rule:}}\parbox[t]{176pt}{$\ds \frac{d}{dx}\Big(f(x) \pm g(x)\Big) = \frac{d}{dx}\Big(f(x)\Big) \pm \frac{d}{dx}\Big(g(x)\Big)$ $= \rule{0pt}{15pt}\fp(x)\pm g\primeskip'(x)$}
	\item	\sword{Sum/Difference Rule:}
	
	$\ds \frac{d}{dx}\Big(f(x) \pm g(x)\Big) = \frac{d}{dx}\Big(f(x)\Big) \pm \frac{d}{dx}\Big(g(x)\Big)$ $= \rule{0pt}{15pt}\fp(x)\pm g\primeskip'(x)$
	\index{derivative!Sum/Difference Rule}\index{Sum/Difference Rule!of derivatives}
	\item		\sword{Constant Multiple Rule:}
	
	$\ds \frac{d}{dx}\Big(c\cdot f(x)\Big) = c\cdot\frac{d}{dx}\Big(f(x)\Big) = c\cdot\fp(x)$.\index{derivative!Constant Multiple Rule}\index{Constant Multiple Rule!of derivatives}
	\end{enumerate}
}

Theorem \ref{thm:deriv_prop} allows us to find the derivatives of a wide variety of functions. It can be used in conjunction with the Power Rule to find the derivatives of any polynomial. Recall in Example \ref{ex_deriv1} that we found, using the limit definition, the derivative of $f(x) = 3x^2+5x-7$. We can now find its derivative without expressly using limits.\\

\example{ex_deriv_added1}{Using the Sum/Difference and the Constant Multiple Rules}{
Determine the derivative of $f(x) = 3x^2+5x-7$.}
{
		\begin{align*}
		\frac{d}{dx}\Big(3x^2+5x+7\Big) &= 3\frac{d}{dx}\Big(x^2\Big) + 5\frac{d}{dx}\Big(x\Big) + \frac{d}{dx}\Big(7\Big) \\
																		&= 3\cdot 2x+5\cdot 1+ 0\\
																		&= 6x+5.
		\end{align*}

We were a bit pedantic here, showing every step. Normally we would do all the arithmetic and steps in our head and readily find $\ds \frac{d}{dx}\Big(3x^2+5x+7\Big) = 6x+5.$
}\\

The next example has components that students often confuse at first.\\

\example{ex_deriv_added2}{Using the Sum/Difference Rule}{
Determine the derivative of $f(x) = x^3 + e^x + e^3 - 5^x$.}
{
This function $f(x)$ is composed by adding and subtracting four different functions.  Note that $\ds \frac{d}{dx}\left(x^3\right)=3x^2$ by the Power Rule.
Then $\ds \frac{d}{dx}\left(e^x\right)=e^x$.  Note, however, that $e^2$ is neither a power function, nor an exponential function.  It is a constant function.  Yes, $e^2$ is just a number; the variable $x$ does not appear in it at all.  So  $\ds \frac{d}{dx}\left(e^2\right)=0$.
Lastly,  $\ds \frac{d}{dx}\left(5^x\right)=\ln 5\cdot 5^x$.  Putting this together,
		\begin{align*}
		\frac{d}{dx}\Big(x^3 + e^x + e^3 - 5^x\Big) &= \frac{d}{dx}\Big(x^3\Big) + \frac{d}{dx}\Big(e^x\Big) + \frac{d}{dx}\Big(e^2\Big) +\frac{d}{dx}\Big(5^x\Big) \\
																		&= 3x^2+e^x+0+\ln 5\cdot 5^x\\
																		&= 3x^2+e^x+\ln 5\cdot 5^x.
		\end{align*}
Again, we usually do the steps in our head and just write the answer.
}\\

In the next example, we must first rewrite the function in an equivalent (not necessarily simpler) form in order to use the rules we know.

\example{ex_deriv_added3}{Using the Sum/Difference and the Constant Multiple Rules}{
Determine the derivative of $\ds f(x) = \frac{(3x-1)^2}{x^2}$.}
{The rules we know allow us to take derivatives of sums and differences of functions, and constant multiples.  We know the derivative of $x^2$ is $2x$, but we cannot just plug in $3x-1$ in for $x$ to get the derivative of $(3x-1)^2$.  Additionally, we cannot divide the derivatives of the numerator and denominator, the way we can for sums and differences.  In the upcoming sections, we will learn how to handle products, quotients, and compositions of functions.  That will allow us to take derivatives of a wider class of functions.  For now, though, we must rewrite $f(x)$ to apply the rules we know:
\begin{align*}
f(x)&=\frac{(3x-1)^2}{x^2}\\
&=\frac{9x^2-6x+1}{x^2}\\
&=9-\frac{6}{x}+\frac{1}{x^2}\\
&=9-6x^{-1}+x^{-2}.
\end{align*}

Therefore,$$\fp(x)=0-6\left(-1x^{-2}\right)+\left(-2x^{-3}\right)=\frac{6}{x^2}-\frac{2}{x^3}.$$
Even when we learn more rules to handle quotients and compositions of functions, this will be the fastest way to compute this particular function's derivative.  Notice the way the solution is written.  We use equal signs between expressions that are equal, when we rewrote the function in another form.  However, we did \textbf{not} write an equal sign between the function and it's derivative.  We labeled the functions $f(x)$ and $\fp(x)$; these functions are different.
}\\

\example{ex_der2}{Using the tangent line to approximate a function value}{
Let $f(x) = \sin x + 2x+1$. Approximate $f(3)$ using an appropriate tangent line.}
{This problem is intentionally ambiguous; we are to \textit{approximate} using an \textit{appropriate} tangent line. How good of an approximation are we seeking? What does appropriate mean?

In the ``real world,'' people solving problems deal with these issues all time. One must make a judgment using whatever seems reasonable. In this example, the actual answer is $f(3) = \sin 3 + 7$, where the real problem spot is $\sin 3$. What is $\sin 3$?

Since $3$ is close to $\pi$, we can assume $\sin 3\approx \sin \pi = 0$. Thus one guess is $f(3) \approx 7$. Can we do better? Let's use a tangent line as instructed and examine the results; it seems best to find the tangent line at $x=\pi$. 

Using Theorem \ref{thm:deriv_common} we find $\fp(x) = \cos x + 2$. The slope of the tangent line is thus $\fp(\pi) = \cos \pi + 2 =1$. Also, $f(\pi) = 2\pi+1 \approx 7.28$. So the tangent line to the graph of $f$ at $x=\pi$ is $y=1(x-\pi)+ 2\pi+1 =x+\pi+1 \approx x+4.14$. Evaluated at $x=3$, our tangent line gives $y=3+4.14 = 7.14$. Using the tangent line, our final approximation is that $f(3) \approx 7.14$.

Using a calculator, we get an answer accurate to 4 places after the decimal: $f(3) = 7.1411$. Our initial guess was $7$; our tangent line approximation was more accurate, at $7.14$.

The point is \textit{not} ``Here's a cool way to do some math without a calculator.'' Sure, that might be handy sometime, but your phone could probably give you the answer. Rather, the point is to say that tangent lines are a good way of approximating, and many scientists, engineers and mathematicians often face problems too hard to solve directly. So they approximate.
}\\

%\enlargethispage{2\baselineskip}
%\vskip \baselineskip
\noindent\textbf{\large Higher Order Derivatives}\\

\mnote{.4}{\textbf{Note:} Definition \ref{def:Higher_Deriv} comes with the caveat ``Where the corresponding limits exist.'' With $f$ differentiable on $I$, it is possible that $\fp$ is \textit{not} differentiable on all of $I$, and so on.}
The derivative of a function $f$ is itself a function, therefore we can take its derivative. The following definition gives a name to this concept and introduces its notation.
\enlargethispage{2\baselineskip}
%\clearpage
%\small
\definition{def:Higher_Deriv}{Higher Order Derivatives}
{Let $y=f(x)$ be a differentiable function on $I$. \index{derivative!higher order}\index{derivative!notation}
		\begin{enumerate}
		\item		The \textit{second derivative} of $f$ is: 
						$$ \fp'(x) = \frac{d}{dx}\Big(\fp(x)\Big) = \frac{d}{dx}\left(\frac{dy}{dx}\right) = \frac{d\primeskip^2y}{dx^2}=y\primeskip''.$$
				\item		The \textit{third derivative} of $f$ is: 
						$$ \fp''(x) = \frac{d}{dx}\Big(\fp'(x)\Big) = \frac{d}{dx}\left(\frac{d\primeskip^2y}{dx^2}\right) = \frac{d\primeskip^3y}{dx^3}=y\primeskip'''.$$
				\item		The \textit{n$^{\text{th}}$ derivative} of $f$ is:
						$$ f\,^{(n)}(x) = \frac{d}{dx}\left(f\,^{(n-1)}(x)\right) = \frac{d}{dx}\left(\frac{d\primeskip^{n-1}y}{dx^{n-1}}\right) = \frac{d\primeskip^ny}{dx^n}=y^{(n)}.$$
		\end{enumerate}
}
\normalsize

In general, when finding the fourth derivative and on, we resort to the $f\,^{(4)}(x)$ notation, not $\fp'''(x)$; after a while, too many ticks is too confusing.\\

Let's practice using this new concept.\\

\example{ex_high_order}{Finding higher order derivatives}{
Find the first four derivatives of the following functions:
	
	
	\noindent\begin{minipage}[t]{.5\textwidth}
	\begin{enumerate}
	\item		$f(x) = 4x^2$
	\item		$f(x) = \sin x$
	\end{enumerate}
	\end{minipage}
	\noindent\begin{minipage}[t]{.5\textwidth}
	\begin{enumerate}\addtocounter{enumi}{2}
	\item		$f(x) = 5e^x$
	\item		$f(x) = 5^x$
	\end{enumerate}
	\end{minipage}
}
{\begin{enumerate}
	\item		Using the Power and Constant Multiple Rules, we have: $\fp(x) = 8x$. Continuing on, we have 
	$$\fp'(x) = \frac{d}{dx}\big(8x\big) = 8;\qquad \fp''(x) = 0;\qquad f\,^{(4)}(x) = 0.$$ Notice how all successive derivatives will also be 0.
	\item		We employ Theorem \ref{thm:deriv_common} repeatedly.
	$$\fp(x) = \cos x;\qquad \fp'(x) = -\sin x;\qquad \fp''(x) = -\cos x;\qquad f\,^{(4)}(x) = \sin x.$$ Note how we have come right back to $f(x)$ again. (Can you quickly figure what $f\,^{(23)}(x)$ is?)
	\item		Employing Theorem \ref{thm:deriv_common} and the Constant Multiple Rule, we can see that $$\fp(x) = \fp'(x) = \fp''(x) = f\,^{(4)}(x) = 5e^x.$$
	\item 		From Theorem \ref{thm:deriv_common}, $\fp(x)=\ln 5 \cdot 5^x$.  Then to take the derivative again, we use the Constant Multiple Rule, $\fp'(x) = \ln 5 \left(\ln 5 \cdot 5^x\right) = (\ln 5)^2 5^x$.  Continuing, we get $\fp''(x) =  (\ln 5)^3 5^x$ and $\fp'''(x) =  (\ln 5)^4 5^x$.
	\end{enumerate}
\vskip-1.5\baselineskip
}
\vskip \baselineskip
\noindent\textbf{\large Interpreting Higher Order Derivatives}\\

What do higher order derivatives \textit{mean}? What is the practical interpretation? \index{derivative!higher order!interpretation}

Our first answer is a bit wordy, but is technically correct and beneficial to understand. That is,
	\begin{quote}
	The second derivative of a function $f$ is the rate of change of the rate of change of $f$.
	\end{quote}

One way to grasp this concept is to let $f$ describe a position function. Then, as stated in Key Idea \ref{idea:motion}, $\fp$ describes the rate of position change: velocity. We now consider $\fp'$, which describes the rate of velocity change. Sports car enthusiasts talk of how fast a car can go from 0 to 60 mph; they are bragging about the \textit{acceleration} of the car.

We started this chapter with amusement--park riders free--falling with position function $f(t) = -16t^2+150$. It is easy to compute $\fp(t)=-32t$ ft/s and $\fp'(t) = -32$ (ft/s)/s. We may recognize this latter constant; it is the acceleration due to gravity. In keeping with the unit notation introduced in the previous section, we say the units are ``feet per second per second.'' This is usually shortened to ``feet per second squared,'' written as ``ft/s$^2$.''

It can be difficult to consider the meaning of the third, and higher order, derivatives. The third derivative is ``the rate of change of the rate of change of the rate of change of $f$.'' That is essentially meaningless to the uninitiated. In the context of our position/velocity/acceleration example, the third derivative is the ``rate of change of acceleration,'' commonly referred to as ``jerk.'' 

Make no mistake: higher order derivatives have great importance even if their practical interpretations are hard (or ``impossible'') to understand at the moment. The mathematical topic of \textit{series} makes extensive use of higher order derivatives.

\printexercises{exercises/02_03_exercises}