\section{Surface Integrals}\label{sec:surface_integrals}

The double integral in Green's Theorem is over a region $R$ in the plane.  In this section we consider instead a curved surface $T$ in three-dimensional space, such as part of a sphere or of a cone. When the $z$ is a function of $x$ and $y$, it can be written $z = f(x,y)$, and the graph is a surface, analogous to how $y = f(x)$ was the graph of a particular type of curve.  In more general cases, a curve could be written parametrically, $x = x(t)$ and $y = y(t)$, as functions of a parameter $t$.  Similarly, a surface can be written parametrically as functions of two parameters.\\

When we compute the length of a curve, we had an integral $\int_C \: ds$, a line integral.  Similarly, when we compute the area of a given surface $T$, we will evaluate an integral $\iint_T \: dS$, called a surface integral. This type of integral will be used to compute surface area even when the surface is not given by a function $z = f(x,y)$ of two variables. \\

We first start with a review of the case of a surface $T$ given by $z = f(x,y)$ for $(x,y)$ in a region $R$ of the $xy$-plane, as in Section \ref{sec:surface_area}.  When we partitioned the region $R$ into small rectangular pieces, the associated pieces of the surface $z = f(x,y)$ will be almost flat, like small parallelograms. The surface area of such a piece ended up being
$$dS = \sqrt{1 + f_x(x,y)^2+f_y(x,y)^2} \: dx \: dy$$
computed using a cross product of the vectors making up the sides of the parallelogram. The total surface area of $T$ could then be computed as
$$S = \iint_R \: dS = \iint_R \sqrt{1 + f_x(x,y)^2+f_y(x,y)^2} \: dA.$$
We now generalize this method to the situation that the surface $T$ is given parametrically.\\

\noindent\textbf{\large Parametric Surfaces and Surface Area}\\

A \textbf{parametric surface} is one defined by three functions of two parameters, which we will denote by $u$ and $v$.  We will write these functions as $x = x(u,v)$, $y = y(u,v)$, and $z = z(u,v)$, where $(u,v)$ come from a region $R$ in the $uv$-plane.  Points in the $uv$-plane are mapped to three-dimensional points $(x,y,z)$ in three-space.\\

\example{ex_surface_int01}{}{Determine a parametric representation for a cylindrical surface $$x^2 + y^2 = 4$$ between $z = 0$ and $z = 8$.}{Using cylindrical coordinates, we know that
$$x = 2 \cos \theta \: \text{ and } \: y = 2 \sin \theta$$
and so we can use $u = \theta$ and $v = z$ as our two parameters. This results in parametric equations
$$x(u,v) = 2 \cos u, \: y(u,v) = 2 \sin u, \: z(u,v) = v$$
in the region described by $0 \leq u \leq 2\pi$ and $0 \leq v \leq 8$.  So a rectangle in the $uv$-plane maps to a cylindrical surface in three-dimensional space.
}\\

\example{ex_surface_int02}{}{Determine a parametric representation for the sphere $x^2 + y^2 + z^2 = a^2$, where $a$ is a constant.}{Now using spherical coordinates, we know that
$$x = a \sin\phi \cos\theta, \: y = a \sin\phi \sin\theta, \: z = a \cos\phi$$
and so we can use $u = \phi$ and $v = \theta$ as our two parameters. This results in parametric equations
$$x(u,v) = a \sin u \cos v, \: y(u,v) = a \sin u \sin v, \: z(u,v) = a \cos u$$
for $(u,v)$ in the rectangular region $R$ in the $uv$-plane given by $0 \leq u \leq \pi$ and $0 \leq v \leq 2\pi$. 
}\\

\example{ex_surface_int03}{}{Determine a parametric representation for any surface $z = f(x,y)$ given by a function of two variables, for $(x,y)$ in a region $R$.}{In this case, we can simply use $u = x$ and $v = y$ for the two parameters.  Then
$$x(u,v) = u, \: y(u,v) = v, \: z = f(u,v)$$
gives the parametric equations for the surface. The region $R$ in the $xy$-plane will be the same as the region $R$ in the $uv$-plane in this case.
}\\

Now we find $dS$ for a surface $T$ given parametrically by equations
$$x = x(u,v), \: y = y(u,v), \: z = z(u,v)$$
for $(u,v)$ in a region $R$, following the same idea as before.  Small increments $du$ and $dv$ in the parameters will result in small parallelogram pieces of the surface of area $dS$.  One side of the parallelogram comes from $du$ while the other comes from $dv$.  The two sides are given by vectors $\vec a \: du$ and $\vec b \: dv$ where
$$\vec a = \dfrac{\partial x}{\partial u} \vec i + \dfrac{\partial y}{\partial u} \vec j + \dfrac{\partial z}{\partial u} \vec k \: \text{ and } \: \vec a = \dfrac{\partial x}{\partial v} \vec i + \dfrac{\partial y}{\partial v} \vec j + \dfrac{\partial z}{\partial v} \vec k.$$
To find the area of the parallelogram piece, we take the norm (magnitude) of the cross product, so
$$dS = \left|\left| \vec a \times \vec b \right|\right| \: du \: dv,$$
and we integrate this over $R$ to get the total surface area. To clarify the notation, let us write
$$\vec r(u,v) = x(u,v) \vec i + y(u,v) \vec j + z(u,v) \vec k ,$$
a vector function describing the surface. Then 
$$\vec a = \vec r_u(u,v) \: \text{ and } \: \vec b = \vec r_v(u,v)$$
as above.  So $dS$ can now be written as
$$dS = \left|\left| \vec r_u \times \vec r_v \right|\right| \: du \: dv$$ and the total surface area is
$$S = \iint_R \left|\left| \vec r_u \times \vec r_v \right|\right| \: du \: dv.$$\\

\example{ex_surface_int04}{}{Use the above method to compute the surface area of the cone $z = \sqrt{x^2 + y^2}$ up to a height of $4$.}{Using $R$ for the circular region below the cone, the standard parameterization is
$$x = u, \: y = v, \: z = \sqrt{u^2 + v^2}$$
for $(u,v)$ in $R$, as above for a surface given by a function of two variables.  We compute $\vec r_u$, $\vec r_v$ and $\vec r_u \times \vec r_v$ first, using 
$$\vec r = u \: \vec i + v \: \vec j + \sqrt{u^2 + v^2} \: \vec k.$$ We have
$$\vec r_u = \vec i + \dfrac{u}{\sqrt{u^2 + v^2}} \vec k \: \text{ and } \: \vec r_v = \vec j + \dfrac{v}{\sqrt{u^2 + v^2}} \vec k$$
and so
$$\vec r_u \times \vec r_v = \left| \begin{array}{ccc} \vec i & \vec j & \vec k \\ 1 & 0 & \frac{u}{\sqrt{u^2 + v^2}} \\ 0 & 1 & \frac{v}{\sqrt{u^2 + v^2}} \end{array} \right| = -\dfrac{v}{\sqrt{u^2 + v^2}} \vec i - \dfrac{u}{\sqrt{u^2 + v^2}} \vec j + \vec k.$$
The norm $\left| \vec r_u \times \vec r_v \right|$ becomes
$$\left| \vec r_u \times \vec r_v \right| = \sqrt{1 + \dfrac{u^2}{u^2 + v^2} + \dfrac{v^2}{u^2 + v^2}} = \sqrt{2}.$$
Therefore the surface area is 
$$S = \iint_R \sqrt{2} \: du \: dv = 16\sqrt{2} \pi$$
square units, since the region $R$ is a circle of radius 4. Notice that $\left|\left| \vec r_u \times \vec r_v \right|\right|$ is the same as $\sqrt{1 + f_x(x,y)^2 + f_y(x,y)^2}$ using $x$ and $y$ instead of $u$ and $v$, as we would have obtained using the method from Section \ref{sec:surface_area}.
}\\

\example{ex_surface_int05}{}{Determine the surface area of the helicoid (a spiral ramp shown in Figure~\ref{fig:helicoid}) given parametrically by
$$\vec r(u,v) = u \cos(v) \: \vec i + u \sin(v) \: \vec j + v \: \vec k$$ for $0 \leq u \leq 1$ and $0 \leq v \leq 4\pi$.}{
\mfigure[scale=0.62]{.35}{Helicoid}{fig:helicoid}{figures/helicoid.jpg}
Notice that we will be integrating over a rectangle in the $uv$-plane. We can immediately calculate $\vec r_u$ and $\vec r_v$ as
$$\vec r_u = \cos(v) \: \vec i + \sin(v) \: \vec j \: \text{ and } \vec r_v = -u \sin(v) \: \vec i + u \cos(v) \: \vec j + \vec k$$
and so
$$\vec r_u \times \vec r_v = \left| \begin{array}{ccc} \vec i & \vec j & \vec k \\ \cos v & \sin v & 0 \\ -u \sin(v) & u \cos(v) & 1 \end{array} \right| = \sin(v) \: \vec i - \cos(v) \: \vec j + u \: \vec k.$$
Therefore the norm of this cross product is
$$\left|\left| \vec r_u \times \vec r_v \right|\right| = \sqrt{\sin^2 v + \cos^2 v + u^2} = \sqrt{u^2 + 1}.$$
Integrating over the rectangle in the $uv$-plane gives us the surface area
$$S = \int_0^{4\pi} \int_0^1 \sqrt{1+u^2} \: du \: dv = 4\pi \left(\left( \dfrac{1}{2}\left( u \sqrt{1+u^2} + \ln|\sqrt{1+u^2} + u| \right)  \right)\right|_0^1$$
which evaluates to $2\pi \left( \sqrt{2} + \ln(1 + \sqrt{2}) \right)$ square units.
}\\

\noindent\textbf{\large Vector Fields and Flux}\\

Given a parameterized surface $T$ over a region $R$ in the $uv$-plane, we have computed the surface area of $T$ as $\iint_T 1 \: dS$.  Replacing $1$ with any function of three variables, an integral of the form $\iint_T f(x,y,z) \: dS$ is called the \emph{surface integral} of $f(x,y,z)$ over the surface $T$. To evaluate such a surface integral, one would translate $f(x,y,z)$ to a function of $u$ and $v$ by using the parametric equation $x = x(u,v)$, $y = y(u,v)$, and $z = z(u,v)$ for the given surface.  Then multiply this expression by $dS = \left| \vec r_u \times \vec r_v \right| \: du \: dv$ and integrate over the region $R$ in the $uv$-plane to obtain the surface integral.\\

\definition{def:surface_integral}{Surface Integral}{Let $T$ be a surface defined by
$$\vec r(u,v) = x(u,v) \: \vec i + y(u,v) \: \vec j + z(u,v) \: \vec k$$
for $(u,v)$ in a region $R$ of the $uv$-plane.  The \textbf{surface integral} of a function $f(x,y,z)$ over $T$ is 
$$\iint_T f(x,y,z) \: dS = \iint_R f(x(u,v), y(u,v), z(u,v)) \: \left|\left| \vec r_u \times \vec r_v \right|\right| \: du \: dv.$$
}\\

\example{ex_surface_int06}{}{Compute the surface integral of $f(x,y,z) = xyz$ over that part of the cylinder $x^2 + y^2 = 1$ above the first quadrant in the $xy$-plane and between $z = 0$ and $z = 2$.}{We first need to parameterize the cylinder as we did earlier, to obtain
$$\vec r(u,v) = \cos(u) \: \vec i + \sin(u) \: \vec j + v \: \vec k$$
for $0 \leq u \leq \frac{\pi}{2}$ and $0 \leq v \leq 2$. Next, we compute $dS$ as before, using
$$\vec r_u \times \vec r_v = \left| \begin{array}{ccc} \vec i & \vec j & \vec k \\ -\sin(u) & \cos(u) & 0 \\ 0 & 0 & 1 \end{array} \right| = \cos(u) \: \vec i + \sin(u) \: \vec j$$
This gives
$$\left|\left| \vec r_u \times \vec r_v \right|\right| = \sqrt{\sin^2(u) + \cos^2(u) +0} = \sqrt{1} = 1$$
for our surface. Now to integrate $f(x,y,z)$ over $T$, we change $f$ into a function of $u$ and $v$ by writing
$$f(x,y,z) = xyz = \cos(u)  \: \sin(u) \: v$$
as in our parameterization.  Therefore the surface integral equals
$$\iint_T f(x,y,z) \: dS = \int_0^{2} \int_{0}^{\pi/2} \cos(u) \: \sin(u) \: v \: du \: dv = \left( \left.\dfrac{1}{2}v^2 \right|_0^2\right) \left(\left. \dfrac{1}{2}\sin^2(u) \right|_0^{\pi/2}\right) = 1.$$
}\\

A key example of an application of a surface integral is in computing the flow or \emph{flux} of a vector field across a given surface, which we discuss next.  As before, we will first address the case when the surface can be written in the form $z = f(x,y)$.\\

\definition{defn_flux}{Flux}{Let $z = f(x,y)$ define a surface $T$ for $(x,y)$ in a region $R$ in the $xy$-plane, and let $$\vec F(x,y,z) = M(x,y,z) \vec i + N(x,y,z) \vec j + P(x,y,z) \vec k$$ be a vector field. The \textbf{flux} of $\vec F$ across $T$ is the surface integral
$$\iint_T \vec F \cdot \vec n \: dS = \iint_R -M \: f_x(x,y) - N \: f_y(x,y) + P \: dx \: dy$$
where $\vec n$ is the unit normal vector to the surface. 
}

Recall that a vector $\vec N$ normal to the surface will be
$$\vec N = -f_x(x,y) \vec i - f_y(x,y) \vec j + \vec k$$
and so the unit normal vector is found by dividing this by its norm
$$\vec n = \dfrac{\vec N}{||\vec N||} = \dfrac{\vec N}{\sqrt{1 + f_x(x,y)^2 + f_y(x,y)^2}}.$$
Notice that the expression representing $| \vec N |$ is exactly the same as in $dS$, so these will cancel to obtain the right-most definition above. That is,
\begin{eqnarray*}
\vec F \cdot \vec n \: dS & = & \dfrac{\vec F \cdot \vec N}{\sqrt{1 + f_x(x,y)^2 + f_y(x,y)^2}} \sqrt{1 + f_x(x,y)^2 + f_y(x,y)^2} \: dx \: dy \\
& = & -M \: f_x(x,y) - N \: f_y(x,y) + P \: \: dx \: dy.
\end{eqnarray*}\\

It is worth noting that we are only considering \emph{orientable} surfaces here, which are surfaces with two sides. In the case of a sphere, for example, there is a normal vector pointing inward and another pointing outward. For a closed surface such as a sphere it is conventional to take the normal vector pointing outward for $\vec n$. That is, we adopt the convention that outward flux is positive. A surface such as a M\"obius strip would not be orientable since it has only one side; a pencil traced along one side around the surface will eventually arrive back at its starting position.\\

\example{ex_surface_int07}{}{Let $\vec F(x,y,z) = x \: \vec i + y \: \vec j + z \: \vec k$ and consider the conical surface $T$ given by $z = f(x,y) = \sqrt{x^2 + y^2}$ for $(x,y)$ within the disk $R$ given by $x^2 + y^2 \leq 4$. Compute the flux of $\vec F$ across this surface.}{For the above definition, we have $M = x$, $N = y$, and $P = z$.  We also have
$$f_x(x,y) = \dfrac{x}{\sqrt{x^2 + y^2}} \: \text{ and } \: f_y(x,y) = \dfrac{y}{\sqrt{x^2 + y^2}}$$
as previously.  Therefore the flux across $T$ is
$$\iint_T \vec F \cdot \vec n \: dS = \iint_R -\dfrac{x^2}{\sqrt{x^2 + y^2}} - \dfrac{y^2}{\sqrt{x^2 + y^2}} + \sqrt{x^2 + y^2} \: dx \: dy$$
since $P = z = \sqrt{x^2 + y^2}$.  Notice that the integrand simplifies to zero, and so the flux across $T$ is equal to zero.  This makes sense since the vector field $\vec F$ goes straight out from the origin, which is along the conical surface, not across it.  That is, there is no flow across $T$ because $\vec F \cdot \vec n = 0$. 
}\\

Next we address the case when the surface $T$ is given parametrically by a function $\vec r(u,v)$ of two parameters $u$ and $v$ from a region $R$ in the $uv$-plane. For surfaces that fold and twist, the formulas can look more complicated, but often the calculations are simpler. Recall that a small parallelogram piece of the surface $T$ will have area $\left|\left| \vec r_u \times \vec r_v \right|\right| \: du \: dv$.  The vectors $\vec r_u$ and $\vec r_v$ are tangent to the surface along the sides.\\

We now put the cross product to another use, because $\vec F \cdot \vec n \: dS$ involves not only area but \emph{direction} as well. We need the unit vector $\vec n$ to know how much flow goes through, since $\vec r_u \times \vec r_v$ is perpendicular to the surface.  Also since $\vec n = \frac{\vec N}{|| \vec N ||}$ we get $$dS = || \vec N || \: du \: dv,$$ as the square root cancels out in $\vec n \: dS$.  This leave the following formula for the flux of a vector field across the surface $T$.\\

\definition{defn_flux_2}{Flux}{Let $\vec r(u,v)$ define a surface $T$ for $(u,v)$ in a region $R$ in the $uv$-plane, and let $$\vec F(x,y,z) = M(x,y,z) \vec i + N(x,y,z) \vec j + P(x,y,z) \vec k$$ be a vector field. The \textbf{flux} of $\vec F$ across $T$ is  
$$\iint_T \vec F \cdot \vec n \: dS = \iint_R \vec F \cdot \left( \vec r_u \times \vec r_v \right) \: du \: dv$$
where $\vec r_u \times \vec r_v$ is normal to the surface.
}

\example{ex_surface_int08}{}{Let $\vec F(x,y,z) = \vec k$ and consider the spherical surface $T$ given by $x^2 + y^2 + z^2 = 9$. Compute the flux of $\vec F$ across the top half of the surface $T$.}{Using spherical coordinates and parameters $u = \phi$ and $v = \theta$, we have
$$\vec r(u,v) = 3 \sin(u) \cos(v) \: \vec i + 3 \sin(u) \sin(v) \: \vec j + 3 \cos(u) \: \vec k$$
and so
\begin{eqnarray*}
\vec N = \vec r_u \times \vec r_v & = & \left| \begin{array}{ccc} \vec i & \vec j & \vec k \\  3 \cos(u) \cos(v) & 3 \cos(u) \sin(v) & -3\sin(u) \\ -3 \sin(u) \sin(v) & 3 \sin(u) \cos(v) & 0 \end{array} \right| \\
 & = & 9 \left( \sin^2(u) \cos(v) \: \vec i + \sin^2(u) \sin(v) \: \vec j + \cos(u) \sin(u) \: \vec k \right).
\end{eqnarray*}
Taking the dot product with $\vec F = \vec k$ we get
$$\vec F \cdot \vec N = 9 \cos(u) \sin(u).$$
Lastly, we integrate on $0 \leq u \leq \frac{\pi}{2}$ and $0 \leq v \leq 2\pi$ to obtain a flux of
$$\int_0^{2\pi} \int_0^{\pi/2} 9 \cos(u) \sin(u) \: du \: dv = 18 \pi \left.\left( \dfrac{1}{2} \sin^2(u) \right)\right|_0^{\pi/2} = 9 \pi$$
across the hemisphere.
}\\
 
We will see another generalization of Green's Theorem in the next section, called the Divergence Theorem, where we look at flux across closed surfaces.

\printexercises{exercises/14_04_exercises_old}