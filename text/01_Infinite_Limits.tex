\section{Infinite Limits}\label{sec:infinite_limits}%\label{sec:limits_infty}

In Definition \ref{def:limit} we stated that in the equation $\ds \lim_{x\to c}f(x) = L$, $L$ was a number. In this section we relax that definition a bit by considering situations when it makes sense to let $L$ be ``infinity.''

As a motivating example, consider $f(x) = 1/x^2$, as shown in Figure \ref{fig:oneoverxsquared}. Note how, as $x$ approaches 0, $f(x)$ grows very, very large. It seems appropriate, and descriptive, to state that $$\lim_{x\rightarrow 0} \frac1{x^2}=\infty.$$ 

\mfigure{.8}{Graphing $f(x) = 1/x^2$ for values of $x$ near 0.}{fig:oneoverxsquared}{figures/figoneoverxsquared}

%In the graph of $f(x)=1/x^2$ shown below, we see that near 0, the function explodes, getting larger and larger, heading off to infinity.

%\myincludegraphics[scale=.4]{text/apex-limits_involving_infinity1.png}
%
%In a case like this, we write
%$$\lim_{x\rightarrow 0} \frac1{x^2}=\infty.$$
%We can make more precise this notion precise as follows:

\definition{def:limit_of_infinity}{Limit of Infinity, $\infty$}
{We say $\ds \lim_{x\rightarrow c} f(x)=\infty$ if for every $M>0$ there exists $\delta>0$ such that for all $x\neq c$, if  $|x-c|<\delta$, then $f(x)\geq M$. \index{limit!of infinity}  
}

This is just like the $\epsilon$--$\delta$ definition from Section \ref{sec:limit_def}.  In that definition, given any (small) value $\epsilon$, if we let $x$ get close enough to $c$ (within $\delta$ units of $c$) then $f(x)$ is guaranteed to be within $\epsilon$ of $f(c)$.  Here, given any (large) value $M$, if we let $x$ get close enough to $c$ (within $\delta$ units of $c$), then $f(x)$ will be at least as large as $M$.  In other words, if we get close enough to $c$, then we can make $f(x)$ as large as we want.  We can define limits equal to $-\infty$ in a similar way.

It is important to note that by saying $\ds \lim_{x\to c}f(x) = \infty$ we are implicitly stating that \textit{the} limit of $f(x)$, as $x$ approaches $c$, \textit{does not exist.} A limit only exists when $f(x)$ approaches an actual numeric value. We use the concept of limits that approach infinity because it is helpful and descriptive.\\

\example{ex_inflim1}{Evaluating infinite limits}{Find $\displaystyle \lim_{x\rightarrow 1}\frac1{(x-1)^2}$ as shown in Figure \ref{fig:inflim1}.

\mfigure{.3}{Observing infinite limit as $x\to 1$ in Example \ref{ex_inflim1}.}{fig:inflim1}{figures/fignolimit2}}
{In Example \ref{ex_no_limit2} of Section \ref{sec:limit_intro}, by inspecting values of $x$ close to 1 we concluded that this limit does not exist.  That is, it cannot equal any real number.  But the limit could be infinite.  And in fact, we see that the function does appear to be growing larger and larger, as $f(.99)=10^4$, $f(.999)=10^6$, $f(.9999)=10^8$.  A similar thing happens on the other side of 1.  In general, let a ``large'' value $M$ be given. Let $\delta=1/\sqrt{M}$. If $x$ is within $\delta$ of 1, i.e., if $|x-1|<1/\sqrt{M}$, then:
	\begin{align*}
	|x-1| &< \frac{1}{\sqrt{M}} \\
	(x-1)^2 &< \frac{1}{M}\\
	\frac{1}{(x-1)^2} &> M,
	\end{align*}
	which is what we wanted to show.  So we may say $\ds\lim_{x\rightarrow 1}1/{(x-1)^2}=\infty$.
}\\

\example{ex_inflim2}{Evaluating infinite limits}{
Find $\displaystyle\lim_{x\rightarrow 0}\frac1x$, as shown in Figure \ref{fig:oneoverx}.}
{It is easy to see that the function grows without bound near 0, but it does so in different ways on different sides of 0.  Since its behavior is not consistent, we cannot say that $\ds \lim_{x\to 0}\frac{1}{x}=\infty$. However, we can make a statement about one--sided limits. We can state that $\ds \lim_{x\rightarrow 0^+}\frac1x=\infty$ and $\ds \lim_{x\rightarrow 0^-}\frac1x=-\infty$.  

\mfigure{.65}{Evaluating $\ds\lim_{x\rightarrow 0}\frac1x$.}{fig:oneoverx}{figures/figoneoverx}
}

\vskip \baselineskip
\noindent\textbf{\large Vertical asymptotes}\index{asymptote!vertical}\\

If the limit of $f(x)$ as $x$ approaches $c$ from either the left or right (or both) is $\infty$ or $-\infty$, we say the function has a \sword{vertical asymptote} at $c$.\\

\example{ex_vertasy1}{Finding vertical asymptotes}{
Find the vertical asymptotes of $f(x)=\dfrac{3x}{x^2-4}$.}
{Vertical asymptotes occur where the function grows without bound; this can occur at values of $c$ where the denominator is 0. When $x$ is near $c$, the denominator is small, which in turn can make the function take on large values.  In the case of the given function, the denominator is 0 at $x=\pm 2$.  Substituting in values of $x$ close to $2$ and $-2$ seems to indicate that the function tends toward $\infty$ or $-\infty$ at those points.  We can graphically confirm this by looking at Figure \ref{fig:multipleasymptotes}. Thus the vertical asymptotes are at $x=\pm2$.

\mfigure{.4}{Graphing $f(x) = \dfrac{3x}{x^2-4}$.}{fig:multipleasymptotes}{figures/figmultipleasymptotes}
}\\

When a rational function has a vertical asymptote at $x=c$, we can conclude that the denominator is 0 at $x=c$. However, just because the denominator is 0 at a certain point does not mean there is a vertical asymptote there.  For instance, $f(x)=(x^2-1)/(x-1)$ does not have a vertical asymptote at $x=1$, as shown in Figure \ref{fig:noasy}.  While the denominator does get small near $x=1$, the numerator gets small too, matching the denominator step for step. In fact, factoring the numerator, we get
$$f(x)=\frac{(x-1)(x+1)}{x-1}.$$
Canceling the common term, we get that $f(x)=x+1$ for $x\not=1$.   So there is clearly no asymptote, rather a hole exists in the graph at $x=1$.\\

\mfigure{.8}{Graphically showing that $f(x) = \dfrac{x^2-1}{x-1}$ does not have an asymptote at $x=1$.}{fig:noasy}{figures/fignoasy}

The above example may seem a little contrived.  Another example demonstrating this important concept is $f(x)= (\sin x)/x$. We have considered this function several times in the previous sections. We found that $\ds \lim_{x\to0}\frac{\sin x}{x}=1$; i.e., there is no vertical asymptote. No simple algebraic cancellation makes this fact obvious; we used the Squeeze Theorem in Section \ref{sec:limit_analytically} to prove this.\\

If the denominator is 0 at a certain point but the numerator is not, then there will usually be a vertical asymptote at that point.  On the other hand,  if the numerator and denominator are both zero at that point, then there may or may not be a vertical asymptote at that point.  This case where the numerator and denominator are both zero returns us to an important topic.

\vskip\baselineskip
\noindent\textbf{Indeterminate Forms}\index{limit!indeterminate form}\index{indeterminate form}\\

%When working with limits and infinity, it is important not to go beyond what the rules of algebra and limits allow.  Consider again the limit below:
We have seen how the limits 
$$\lim_{x\rightarrow 0}\frac{\sin x}{x}\quad \text{and}\quad \lim_{x\to1}\frac{x^2-1}{x-1}$$ each return the indeterminate form ``$0/0$'' when we blindly plug in $x=0$ and $x=1$, respectively. However, $0/0$ is not a valid arithmetical expression. It gives no indication that the respective limits are 1 and 2.% (hence the use of the word ``indeterminate.'')
%Blindly plugging in $x=0$ would give us the expression $0/0$.  This is not a valid arithmetical expression because division by 0 is not allowed.  In fact, as we have seen already, the correct value of the limit is 1.

%The expression $0/0$ is called an \emph{indeterminate form}.  For an idea as to where the name comes from, consider the following limit:
%$$\lim_{x\rightarrow 1}\frac{x^2-1}{x-1}.$$
%Blindly plugging in $x=1$ here gives $0/0$.  However, this time, the correct value of the limit is 2, which can be seen by factoring the numerator and then plugging in $x=1$.  So we have seen that the initial expression $0/0$ can correspond to a limit of 1 or a limit of 2.  In fact, w

With a little cleverness, one can come up $0/0$ expressions which have a limit of $\infty$, 0, or any other real number.  That is why this expression is called \emph{indeterminate}.

A key concept to understand is that such limits do not really return $0/0$. Rather, keep in mind that we are taking \textit{limits}. What is really happening is that the numerator is shrinking to 0 while the denominator is also shrinking to 0. The respective rates at which they do this are very important and determine the actual value of the limit.

An indeterminate form indicates that one needs to do more work in order to compute the limit. That work may be algebraic (such as factoring and canceling) or it may require a tool such as the Squeeze Theorem. %algebraically manipulate the expression in some way in order to compute the limit.  It may also indicate that you need a completely different approach, like with $(\sin x)/x$. 
 In a later section we will learn a technique called l'Hospital's Rule that provides another way to handle indeterminate forms.  
 
Some other common indeterminate forms are $\infty-\infty$, $\infty\cdot 0$, $\infty/\infty$, $0^0$, $\infty^0$ and $1^{\infty}$. Again, keep in mind that these are the ``blind'' results of evaluating a limit, and each, in and of itself, has no meaning. The expression $\infty-\infty$ does not really mean ``subtract infinity from infinity.'' Rather, it means ``One quantity is subtracted from the other, but both are growing without bound.'' What is the result? It is possible to get every value between $-\infty$ and $\infty$

Note that $1/0$ and $\infty/0$ are not indeterminate forms, though they are not exactly valid mathematical expressions, either.  In each, the function is growing without bound, indicating that the limit will be $\infty$, $-\infty$, or simply not exist if the left- and right-hand limits do not match.


These are just two quick examples of why we are interested in limits. Many students dislike this topic when they are first introduced to it, but over time an appreciation is often formed based on the scope of its applicability.

\printexercises{exercises/01_06_exercises}

%Maybe some practical example here, like the cost to remove 100% of something and vertical asymptotes and the the long term limit of a population to demonstrate limits at infinity.  Or maybe just put these into the exercises.

%\end{document}

%Do both limits to infinity, limits resulting in �infinity�.
%Define more clearly indeterminate form.
%Thm: Rules about limits involving infinity
%Thm: Something about the limits as x ? 8 and rational functions? 