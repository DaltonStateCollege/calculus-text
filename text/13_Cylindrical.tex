\section{Triple Integration with Cylindrical Coordinates}\label{sec:cylindrical}

Just as polar coordinates gave us a new way of describing curves in the plane, in this section we will see how \emph{cylindrical} coordinates give us new ways of desribing surfaces and regions in space.  Then we will explore \emph{spherical} coordinates in the next section.

In short, cylindrical coordinates can be thought of as a combination of the polar and rectangular coordinate systems. %One can identify a location in the $xy$ plane using polar coordinates (i.e., using $(r,\theta)$), then use a $z$-value to determine the distance
One can identify a point $(x_0,y_0,z_0)$, given in rectangular coordinates, with the point $(r_0,\theta_0,z_0)$, given in cylindrical coordinates, where the $z$-value in both systems is the same, and the point $(x_0,y_0)$ in the $xy$ plane is identified with the polar point $P(r_0,\theta_0)$; see Figure \ref{fig:cylindricalintro}.
\index{cylindrical coordinates}\index{coordinates!cylindrical}
\mfigurethree{width=150pt,3Dmenu,activate=onclick,deactivate=onclick,
3Droll=0,
3Dortho=0.004267800133675337,
3Dc2c=0.6733480095863342 0.6255449652671814 0.39407607913017273,
3Dcoo=-12.612065315246582 -11.544181823730469 33.65996551513672,
3Droo=150.0000065104165,
3Dlights=Headlamp,add3Djscript=asylabels.js}{width=150pt}{.75}{Illustrating the principles behind cylindrical coordinates.}{fig:cylindricalintro}{figures/figcylindricalintro_3D}

The conversions between cylindrical and Cartesian coordinates are given as follows, which follows from our previous study of polar coordinates found in Key Idea \ref{idea:polarconvert}.\\

\keyidea{idea:cylcart_conversion}{Converting between cylindrical and Cartesian coordinates}{
Given a point $(r,\theta,z)$ in cylindrical coordinates, its Cartesian coordinates are
$$x = r \cos\theta, \: y = r \sin\theta, \: z = z.$$
Given a point $(x,y,z)$ in Cartesian coordinates, we use
$$r^2 = x^2 + y^2, \: \tan\theta = \dfrac{y}{x} \text{ (for $x\not=0$)}, \: z = z$$
to convert to cylindrical coordinates.\\
}\\

\example{ex_cylindrical4}{Converting between rectangular and cylindrical coordinates}
{Convert the rectangular point $(-2,2,1)$ to cylindrical coordinates, and convert the cylindrical point $(4,5\pi/6,5)$ to rectangular.
}
{Following the identities given in Key Idea \ref{idea:polarconvert}, we have $r^2=(-2)^2+2^2=8$, so $r = \sqrt8 = 2\sqrt{2}$. Using $\tan\theta = y/x$, we find $\tan\theta=-1$, which has solutions $\theta=-\frac{\pi}{4}+\pi k$.  Since $x<0$, we are in the second quadrant when restricting to the $xy$ plane, so we can use $\theta=\frac{3\pi}{4}$.  Finally, $z = 1$, giving the cylindrical point $(2\sqrt2,3\pi/4,1)$.

In converting the cylindrical point $(4,5\pi/6,5)$ to rectangular, we have
$x = 4\cos\big(5\pi/6\big) = -2\sqrt{3}$, $y = 4\sin\big(5\pi/6\big) = 2$ and $z=5$, giving the rectangular point $(-2\sqrt{3},2,5)$.
}\\

Surfaces and solids described using Cartesian coordinates can also be written using cylindrical coordinates, and vice versa, using our conversion formulas.
Setting each of $r$, $\theta$ and $z$ equal to a constant defines a surface in space, as illustrated in the following example.\\

\example{ex_cylindrical1}{Canonical surfaces in cylindrical coordinates}%
{Describe the surfaces $r=1$, $\theta = \pi/3$ and $z=2$, given in cylindrical coordinates.
}
{The equation $r=1$ describes all points in space that are 1 unit away from the $z$-axis. This surface is a ``tube'' or ``cylinder'' of radius 1, centered on the $z$-axis, as graphed in Figure \ref{fig:spacecylinder1} (which describes the cylinder $x^2+y^2=1$ in space). 

The equation $\theta=\pi/3$ describes the plane formed by extending the line $\theta=\pi/3$, as given by polar coordinates in the $xy$ plane, parallel to the $z$-axis.

The equation $z=2$ describes the plane of all points in space that are 2 units above the $xy$ plane. This plane is the same as the plane described by $z=2$ in rectangular coordinates.
\mfigurethree{width=150pt,3Dmenu,activate=onclick,deactivate=onclick,
3Droll=0,
3Dortho=0.005137230269610882,
3Dc2c=0.8924431204795837 0.30546268820762634 0.3320206105709076,
3Dcoo=-13.9379243850708 -4.1078948974609375 30.89048194885254,
3Droo=149.9999934895832,
3Dlights=Headlamp,add3Djscript=asylabels.js}{width=150pt}{.62}{Graphing the canoncial surfaces in cylindrical coordinates from Example \ref{ex_cylindrical1}.}{fig:cylindrical1}{figures/figcylindrical1_3D}

All three surfaces are graphed in Figure \ref{fig:cylindrical1}. Note how their intersection uniquely defines the point $P=(1,\pi/3,2)$.
}\\

\example{ex_cylsurface}{}{Consider the surface described by $z=r^2$ in cylindrical coordinates.  Write the equation for this surface in Cartesian coordinates and identify the surface.}{We know that $r^2 = x^2 + y^2$, and so we can write $z = x^2 + y^2$.  This is a paraboloid with vertex at the origin opening up. See Figure \ref{fig:fig_cylindrical_spherical_z_equals_r_squared}. \\
}\\

\mfigure[scale=0.67]{.35}{$z=r^2$}{fig:fig_cylindrical_spherical_z_equals_r_squared}{figures/fig_cylindrical_spherical_z_equals_r_squared}

Cylindrical coordinates are useful when describing certain domains in space, allowing us to evaluate triple integrals over these domains more easily than if we used rectangular coordinates.

Theorem \ref{thm:triple_integration2} shows how to evaluate $\iiint_Dh(x,y,z)\ dV$ using rectangular coordinates. In that evaluation, we use $dV = dz\,dy\,dx$ (or one of the other five orders of integration). Recall how, in this order of integration, the bounds on $y$ are ``curve to curve'' and the bounds on $x$ are ``point to point'': these bounds describe a region $R$ in the $xy$ plane. We could describe $R$ using polar coordinates as done in Section \ref{sec:double_int_polar}. In that section, we saw how we used $dA = r\,dr\,d\theta$ instead of $dA = dy\,dx$. 

Considering the above thoughts, we have $dV = dz\big(r\,dr\,d\theta\big) = r\,dz\,dr\,d\theta$. We set bounds on $z$ as ``surface to surface'' as done in the previous section, and then use ``curve to curve'' and ``point to point'' bounds on $r$ and $\theta$, respectively. Finally, using the identities given above, we change the integrand $h(x,y,z)$ to $h(r,\theta,z)$.

This process should sound plausible; the following theorem states it is truly a way of evaluating a triple integral.

\theorem{thm:triple_int_cylindrical}{Triple Integration in Cylindrical Coordinates}
{%
Let $w=h(r,\theta,z)$ be a continuous function on a closed, bounded region $D$ in space, bounded in cylindrical coordinates by $\alpha \leq \theta \leq \beta$, $g_1(\theta)\leq r \leq g_2(\theta)$ and $f_1(r,\theta) \leq z \leq f_2(r,\theta)$. Then \index{integration!with cylindrical coordinates} 
$$\iiint_D h(r,\theta,z)\ dV = \int_\alpha^\beta\int_{g_1(\theta)}^{g_2(\theta)}\int_{f_1(r,\theta)}^{f_2(r,\theta)}h(r,\theta,z) r\,dz\,dr\,d\theta.$$
}

\mfigure[scale=0.7]{.5}{$z=1-r$}{fig:fig_cylindrical_spherical_z_equals_1_minus_r}{figures/fig_cylindrical_spherical_z_equals_1_minus_r}

\example{ex_conevolume}{Evaluating a volume with cylindrical coordinates}{The surface $z = 1 - r$ encloses a conical solid between itself and the $xy$-plane.  Find the volume of this solid as seen in Figure \ref{fig:fig_cylindrical_spherical_z_equals_1_minus_r}.}{The cone $z = 1-r$ intersects the $xy$-plane ($z = 0$) at a radius of $r = 1$.  This puts the bounds on $r$ as $0 \leq r \leq 1$, and clearly the bounds on $\theta$ would be $0 \leq \theta \leq 2\pi$ since the solid is the full cone.  Since the solid is not a cylinder, at least one of the bounds on $z$ must be non-constant.  At any radius $r$, the height of the solid is bounded between the $xy$-plane and the surface $z = 1-r$.  Therefore the bounds on $z$ are $0 \leq z \leq 1-r$, and we will integrate with respect to $dz$ first.  Therefore the volume is
$$V = \int_{0}^{2\pi} \int_{0}^{1} \int_{0}^{1-r} r \: dz \: dr \: d\theta = 2\pi \int_0^1 (1-r)r \: dr$$
which evaluates to a volume of $\dfrac{\pi}{3}$ cubic units.\\
}\\



\example{ex_volume02}{Evaluating a mass with cylindrical coordinates}{A solid $D$ is the region inside the cylinder $x^2 + y^2 = 1$, below the plane $z = 2$ and above the paraboloid $z = 1 - x^2 - y^2$, with all distances in meters.  See Figure \ref{fig:fig_cylindrical_spherical_mass_example}.  If the density of this solid is 
$$\delta(x,y,z) = \sqrt{x^2 + y^2}$$
kilograms per cubic meter, determine the mass of this solid.}{As this solid is a cylinder with a paraboloid cut out of the bottom, it makes sense to use cylindrical coordinates here, where the mass is the integral of the density over the solid.  The $z$-coordinates of this solid solid are bounded below by $1-x^2-y^2$ or $1-r^2$, and bounded above by $z=2$.  Therefore the solid $D$ is described by
$$0 \leq r \leq 1, \: 0 \leq \theta \leq 2\pi, \: 1-x^2-y^2 = 1-r^2 \leq z \leq 2$$
and the density function, when converted to cylindrical coordinates, is $$\delta(r,\theta,z) = \sqrt{r^2} = r.$$  Integrating using the order $dz \: d\theta \: dr$ and the additional factor of $r$ for cylindrical coordinates yields a mass of
$$\int_{0}^{1} \int_{0}^{2\pi} \int_{1-r^2}^{2} r^2 \: dz \: d\theta \: dr = 2\pi \int_0^1 r^2 + r^4 \: dr = 2\pi \left( \dfrac{1}{12} \right) = \dfrac{16 \pi}{15}$$
kilograms.\\
}\\

\mfigure[scale=0.5]{.75}{Solid $D$ bounded above by $z=2$, below by $z=1-x^2-y^2$, laterally by $x^2+y^2=1$}{fig:fig_cylindrical_spherical_mass_example}{figures/fig_cylindrical_spherical_mass_example}

\example{ex_cylindrical2}{Evaluating another mass with cylindrical coordinates}
{Find the mass of the solid represented by the region in space bounded by $z=0$, $z=\sqrt{4-x^2-y^2}+3$ and the cylinder $x^2+y^2=4$ (as shown in Figure \ref{fig:cylindrical2}), with density function $\delta(x,y,z) = x^2+y^2+z+1$, using a triple integral in cylindrical coordinates. Distances are measured in centimeters and density is measured in grams/cm$^3$.
}
{We begin by describing this region of space with cylindrical coordinates. The plane $z=0$ is left unchanged; with the identity $r=\sqrt{x^2+y^2}$, we convert the hemisphere of radius 2 to the equation $z=\sqrt{4-r^2}$; the cylinder $x^2+y^2=4$ is converted to $r^2=4$, or, more simply, $r=2$.  We also convert the density function: $\delta(r,\theta,z) = r^2+z+1$. 
\mfigurethree{width=150pt,3Dmenu,activate=onclick,deactivate=onclick,
3Droll=0,
3Dortho=0.004440656863152981,
3Dc2c=0.6615753173828125 0.6849026679992676 0.30533018708229065,
3Dcoo=-6.77640962600708 2.852377414703369 51.19231414794922,
3Droo=300,
3Dlights=Headlamp,add3Djscript=asylabels.js}{width=150pt}{.35}{Visualizing the solid used in Example \ref{ex_cylindrical2}.}{fig:cylindrical2}{figures/figcylindrical2_3D} 

To describe this solid with the bounds of a triple integral, we bound $z$ with $0\leq z\leq \sqrt{4-r^2}+3$; we bound $r$ with $0 \leq r \leq 2$; we bound $\theta$ with $0 \leq \theta \leq 2\pi$.

Using Definition \ref{def:mass_3d} and Theorem \ref{thm:triple_int_cylindrical}, we have the mass of the solid is
\begin{align*}
M=\iiint_D\delta(x,y,z)\ dV &= \int_0^{2\pi}\int_0^2\int_0^{\sqrt{4-r^2}+3}\big(r^2+z+1\big)r\,dz\,dr\,d\theta \\
&= \int_0^{2\pi}\int_0^2\left(\big(r^3+4r\big)\sqrt{4-r^2}+\frac52r^3+\frac{19}2r\right)\,dr\,d\theta \\
&= \frac{1318\pi}{15} \approx 276.04\text{ grams},
\end{align*}
where we leave the details of the remaining double integral to the reader.
}\\

\example{ex_cylindrical3}{Finding the center of mass using cylindrical coordinates}
{Find the center of mass of the solid with constant density whose base can be described by the polar curve $r=\cos(3\theta)$ and whose top is defined by the plane $z=1-x+0.1y$, where distances are measured in feet, as seen in Figure \ref{fig:cylindrical3}. (The volume of this solid was found in Example \ref{ex_doublepol4}.)
}
{We convert the equation of the plane to use cylindrical coordinates: $z= 1-r\cos\theta+0.1r\sin\theta$. Thus the region is space is bounded by $0 \leq z \leq 1-r\cos\theta + 0.1r\sin\theta$, $0 \leq r \leq \cos(3\theta)$, $0 \leq \theta \leq \pi$ (recall that the rose curve $r=\cos(3\theta)$ is traced out once on $[0,\pi]$.
\mfigurethree{width=150pt,3Dmenu,activate=onclick,deactivate=onclick,
3Droll=-0.545358908878901,
3Dortho=0.004904536530375481,
3Dc2c=0.25393617153167725 0.2084963023662567 0.9444817304611206,
3Dcoo=-6.77640962600708 2.852377414703369 51.19231414794922,
3Droo=299.99999864287065,
3Dlights=Headlamp,add3Djscript=asylabels.js}{width=150pt}{.7}{Visualizing the solid used in Example \ref{ex_cylindrical3}.}{fig:cylindrical3}{figures/figdoublepol4_3D} 

Since density is constant, we set $\delta = 1$ and finding the mass is equivalent to finding the volume of the solid. We set up the triple integral to compute this but do not evaluate it; we leave it to the reader to confirm it evaluates to the same result found in Example \ref{ex_doublepol4}.
$$M = \iiint_D\delta \, dV = \int_0^{\pi}\int_0^{\cos(3\theta)}\int_0^{1-r\cos\theta+0.1r\sin\theta} r\,dz\,dr\,d\theta \approx 0.785.$$

From Definition \ref{def:mass_3d} we set up the triple integrals to compute the moments about the three coordinate planes. The computation of each is left to the reader (using technology is recommended):
\begin{align*}
M_{yz} = \iiint_D x\,dV &= \int_0^{\pi}\int_0^{\cos(3\theta)}\int_0^{1-r\cos\theta+0.1r\sin\theta} (r\cos\theta) r\,dz\,dr\,d\theta\\
&= -0.147.
\end{align*}
\begin{align*}
M_{xz} = \iiint_D y\,dV &= \int_0^{\pi}\int_0^{\cos(3\theta)}\int_0^{1-r\cos\theta+0.1r\sin\theta} (r\sin\theta) r\,dz\,dr\,d\theta\\
&= 0.015.\\
M_{xy} = \iiint_D z\,dV &= \int_0^{\pi}\int_0^{\cos(3\theta)}\int_0^{1-r\cos\theta+0.1r\sin\theta} (z) r\,dz\,dr\,d\theta\\
 &= 0.467.
\end{align*}
The center of mass, in rectangular coordinates,  is located at $(-0.147,0.015,0.467)$, which lies outside the bounds of the solid.
}\\

We have just seen how cylindrical coordinates can be used to make the computation of some triple integrals easier.  Over other regions, however, such as spheres, it is more natural to tackle triple integrals using another coordinate system, spherical coordinates, to be discussed in the next section.




\printexercises{exercises/13_07_exercises}