\section{Limits At Infinity}\label{sec:limits_at_infty}

In this section we relax Definition \ref{def:limit} even further by considering situations when it makes sense to let $c$ be ``infinity'' in the equation $\ds \lim_{x\to c}f(x) = L$.

Consider again $f(x) = 1/x^2$, as shown in Figure \ref{fig:limits_at_infinity_oneoverxsquared}. Note that as $x$ gets very large, $f(x)$ gets very, very small. We could represent this concept with notation such as $$\lim_{x\rightarrow \infty} \frac1{x^2}=0.$$

\mfigure{.8}{Graphing $f(x) = 1/x^2$.}{fig:limits_at_infinity_oneoverxsquared}{figures/figoneoverxsquared}

%In the graph of $f(x)=1/x^2$ shown below, we see that near 0, the function explodes, getting larger and larger, heading off to infinity.

%\myincludegraphics[scale=.4]{text/apex-limits_involving_infinity1.png}
%
%In a case like this, we write
%$$\lim_{x\rightarrow 0} \frac1{x^2}=\infty.$$
%We can make more precise this notion precise as follows:

Graphically, we are concerned with the behavior of the function to the ``far right'' of the graph. We make this notion more explicit in the following definition.

%\setboxwidth{65pt}%
%\noindent\ifthenelse{\isodd{\thepage}}{}{\hskip -65pt}%
%\noindent\begin{minipage}{\specialboxlength}
\definition{def:limit_at_infinity}{Limits at Infinity and Horizontal Asymptote}
{\begin{enumerate}
\item We say $\ds\lim_{x\rightarrow\infty} f(x)=L$ if for every $\epsilon>0$ there exists $M>0$ such that if $x\geq M$, then $|f(x)-L|<\epsilon$.\index{limit!at infinity}\index{asymptote!horizontal} \\

\item We say $\ds\lim_{x\rightarrow-\infty} f(x)=L$ if for every $\epsilon>0$ there exists $M<0$ such that if $x\leq M$, then $|f(x)-L|<\epsilon$. \\

 %In other words, the limit is $L$ if no matter how close you want to get to $L$, for large enough values of $x$ ($x>M$), you can get that close.
\item  If $\ds\lim_{x\rightarrow\infty} f(x)=L$ or $\ds\lim_{x\rightarrow-\infty} f(x)=L$, we say that $y=L$ is a \textbf{horizontal asymptote} of $f$.
\end{enumerate}
}
%\end{minipage}
\restoreboxwidth
%\setlength{\specialboxlength}{\textwidth-2\specialboxinnerseplength}

%We can define $\lim_{x\rightarrow-\infty} f(x)=L$ in an analogous way.  
We can also define limits such as $\ds\lim_{x\rightarrow\infty}f(x)=\infty$ by combining this definition with Definition \ref{def:limit_of_infinity}. \\ %It is a good exercise to try this.

\clearpage
\example{ex_hzasy1}{Approximating horizontal asymptotes}{
Approximate the horizontal asymptote(s) of $\ds f(x)=\frac{x^2}{x^2+4}$.}
{We will approximate the horizontal asymptotes by approximating the limits $$\lim_{x\to-\infty} \frac{x^2}{x^2+4}\quad \text{and}\quad \lim_{x\to\infty} \frac{x^2}{x^2+4}.$$ Figure \ref{fig:hzasy1}(a) shows a sketch of $f$, and part (b) gives values of $f(x)$ for large magnitude values of $x$. It seems reasonable to conclude from both of these sources that $f$ has a horizontal asymptote at $y=1$.

\mtable{.73}{Using a graph and a table to approximate a horizontal asymptote in Example \ref{ex_hzasy1}.}{fig:hzasy1}{%
\begin{tabular}{c}
\myincludegraphics{figures/fighzasy1}\\
(a)\\[10pt]
\begin{tabular}{cc}
		$x$ & $f(x)$ \\ \hline
		\rule{0pt}{11pt} 10 & 0.9615 \\
		100 & 0.9996 \\
		10000 & 0.999996\\
		$-10$ & 0.9615 \\
		$-100$ & 0.9996 \\
		$-10000$ & 0.999996
		\end{tabular} \\[10pt]
		(b)
\end{tabular}
}

Later, we will show how to determine this analytically.}\\

Horizontal asymptotes can take on a variety of forms. Figure \ref{fig:hzasy}(a) shows that $f(x) = x/(x^2+1)$ has a horizontal asymptote of $y=0$, where 0 is approached from both above and below.

%\mfigure{.4}{Graphing $f(x) = \dfrac{x}{x^2+1}$}{fig:hzasy2}{figures/fighzasy2}

Figure \ref{fig:hzasy}(b) shows that $f(x) =x/\sqrt{x^2+1}$ has two horizontal asymptotes; one at $y=1$ and the other at $y=-1$.

%\mfigure{.55}{Graphing $f(x) = \dfrac{x}{\sqrt{x^2+1}}$ to see its two horizontal asymptotes.}{fig:hzasy3}{figures/fighzasy3}

Figure \ref{fig:hzasy}(c) shows that $f(x) = (\sin x)/x$ has even more interesting behavior than at just $x=0$; as $x$ approaches $\pm\infty$, $f(x)$ approaches 0, but oscillates as it does this.\\% 

% because middle of the page figures get labels before the marginal ones do, we need to adjust the figure
% count.
\addtocounter{figure}{1}
\vskip 10pt
\hskip-160pt\noindent\begin{minipage}{\textwidth+100pt}
\begin{tabular}{ccc}
\myincludegraphics{figures/fighzasy2} & \myincludegraphics{figures/fighzasy3}  & \myincludegraphics{figures/fighzasy4} \\
(a) & (b) & (c)
\end{tabular}
\captionsetup{type=figure}%
\caption{Considering different types of horizontal asymptotes.}
\label{fig:hzasy}
\end{minipage}
\\
\vskip10pt
% further adjusting figure count
\addtocounter{figure}{-2}

We can analytically evaluate limits at infinity for rational functions once we understand $\ds\lim_{x\rightarrow\infty} 1/x$.  As $x$ gets larger and larger, the $1/x$ gets smaller and smaller, approaching 0.  We can, in fact, make $1/x$ as small as we want by choosing a large enough value of $x$.  Given $\epsilon$, we can make $1/x<\epsilon$  by choosing $x>1/\epsilon$.  Thus we have $\lim_{x\rightarrow\infty} 1/x=0$.  

%\mfigure{.3}{Graphing $f(x) = \dfrac{\sin x}{x}$ to display the oscillating behavior of its horizontal asymptote.}{fig:hzasy4}{figures/fighzasy4}

It is now not much of a jump to conclude the following:
$$\lim_{x\rightarrow\infty}\frac1{x^n}=0\quad \text{and}\quad \lim_{x\rightarrow-\infty}\frac1{x^n}=0$$

Now suppose we need to compute the following limit:
$$\lim_{x\rightarrow\infty}\frac{x^3+2x+1}{4x^3-2x^2+9}.$$
A good way of approaching this is to divide through the numerator and denominator by $x^3$ (hence dividing by 1), which is the largest power of $x$ to appear in the function.  Doing this, we get
\begin{align*}
\lim_{x\rightarrow\infty}\frac{x^3+2x+1}{4x^3-2x^2+9} &=
\lim_{x\rightarrow\infty}\frac{1/x^3}{1/x^3}\cdot\frac{x^3+2x+1}{4x^3-2x^2+9}\\ &=\lim_{x\rightarrow\infty}\frac{x^3/x^3+2x/x^3+1/x^3}{4x^3/x^3-2x^2/x^3+9/x^3}\\ &= \lim_{x\rightarrow\infty}\frac{1+2/x^2+1/x^3}{4-2/x+9/x^3}.
\end{align*}
Then using the rules for limits (which also hold for limits at infinity), as well as the fact about limits of $1/x^n$, we see that the limit becomes
$$\frac{1+0+0}{4-0+0}=\frac14.$$

This procedure works for any rational function.  In fact, it gives us the following theorem.

% final adjustment to figure count. Fixes future references
% included here to ensure this goes into effect on the next page, after
% previous marginal labels have been created
\addtocounter{figure}{1}
\theorem{thm:lim_rational_fn_at_infty}{Limits of Rational Functions at Infinity}
{Let $f(x)$ be a rational function of the following form:
$$f(x)=\frac{a_nx^n + a_{n-1}x^{n-1}+\dots + a_1x + a_0}{b_mx^m + b_{m-1}x^{m-1} + \dots + b_1x + b_0},$$
where any of the coefficients may be 0 except for $a_n$ and $b_m$.
\begin{enumerate}
\item If $n=m$, then $\ds\lim_{x\rightarrow\infty} f(x) = \lim_{x\rightarrow-\infty} f(x) = \frac{a_n}{b_m}$.
\item If $n<m$, then $\ds\lim_{x\rightarrow\infty} f(x) = \lim_{x\rightarrow-\infty} f(x) = 0$.
\item If $n>m$, then $\ds\lim_{x\rightarrow\infty} f(x)$ and $\lim_{x\rightarrow-\infty} f(x)$ are both infinite ($\infty$ or $-\infty$).
\end{enumerate}
}

We can see why this is true.  If the highest power of $x$ is the same in both the numerator and denominator (i.e. $n=m$), we will be in a situation like the example above, where we will divide by $x^n$ and in the limit all the terms will approach 0 except for $a_nx^n/x^n$ and $b_mx^m/x^n$. Since $n=m$, this will leave us with the limit $a_n/b_m$.  If $n<m$, then after dividing through by $x^m$, all the terms in the numerator will approach 0 in the limit, leaving us with $0/b_m$ or 0.  If $n>m$, and we try dividing through by $x^n$, we end up with all the terms in the denominator tending toward 0, while the $x^n$ term in the numerator does not approach 0.  This is indicative of some sort of infinite limit.

Intuitively, as $x$ gets very large (approaches $\infty$ or $-\infty$), all the terms in the numerator are small in comparison to $a_nx^n$, and likewise all the terms in the denominator are small compared to $b_nx^m$.  If $n=m$, looking only at these two important terms, we have $(a_nx^n)/(b_nx^m)$.  This reduces to $a_n/b_m$.  If $n<m$, the function behaves like $a_n/(b_mx^{m-n})$, which tends toward 0.  If $n>m$, the function behaves like $a_nx^{n-m}/b_m$, which will tend to either $\infty$ or $-\infty$ depending on the values of $n$, $m$, $a_n$, $b_m$ and whether you are looking for $\lim_{x\rightarrow\infty} f(x)$ or $\lim_{x\rightarrow-\infty} f(x)$.

With care, we can quickly evaluate limits at infinity for a large number of functions by considering the largest powers of $x$. For instance, consider again $\ds\lim_{x\to\pm\infty}\frac{x}{\sqrt{x^2+1}},$ graphed in Figure \ref{fig:hzasy}(b). When $x$ is very large, $x^2+1 \approx x^2$. Thus $$\sqrt{x^2+1}\approx \sqrt{x^2} = |x|,\quad \text{and}\quad \frac{x}{\sqrt{x^2+1}} \approx \frac{x}{|x|}.$$ This expression is 1 when $x$ is positive and $-1$ when $x$ is negative. Hence we get asymptotes of $y=1$ and $y=-1$, respectively.\\

\example{ex_hzasy2}{Finding a limit of a rational function}{
Confirm analytically that $y=1$ is the horizontal asymptote of $\ds f(x) = \frac{x^2}{x^2+4}$, as approximated in Example \ref{ex_hzasy1}.}
{Before using Theorem \ref{thm:lim_rational_fn_at_infty}, let's use the technique of evaluating limits at infinity of rational functions that led to that theorem. The largest power of $x$ in $f$ is 2, so divide the numerator and denominator of $f$ by $x^2$, then take limits.
	\begin{align*}
	\lim_{x\to\infty}\frac{x^2}{x^2+4} &= \lim_{x\to\infty}\frac{x^2/x^2}{x^2/x^2+4/x^2}\\
																			&=\lim_{x\to\infty}\frac{1}{1+4/x^2}\\
																			&=\frac{1}{1+0}\\
																			&= 1.
	\end{align*}

We can also use Theorem \ref{thm:lim_rational_fn_at_infty} directly; in this case $n=m$ so the limit is the ratio of the leading coefficients of the numerator and denominator, i.e., 1/1 = 1. 
}\\

\example{ex_hzasy3}{Finding limits of rational functions}{
Use Theorem \ref{thm:lim_rational_fn_at_infty} to evaluate each of the following limits.

\noindent\begin{minipage}[t]{.5\textwidth}
\begin{enumerate}
\item		$\displaystyle\lim_{x\rightarrow-\infty}\frac{x^2+2x-1}{x^3+1}$
\item		$\displaystyle\lim_{x\rightarrow\infty}\frac{x^2+2x-1}{1-x-3x^2}$
\end{enumerate}
\end{minipage}
\begin{minipage}[t]{.5\textwidth}
\begin{enumerate}\addtocounter{enumi}{2}
\item		$\displaystyle\lim_{x\rightarrow\infty}\frac{x^2-1}{3-x}$
\end{enumerate}
\end{minipage}
}
{
	\begin{enumerate}
	\item		The highest power of $x$ is in the denominator.  Therefore, the limit is 0; see Figure \ref{fig:hzasy3}(a).
	
	
	%\mfigure{.8}{Visualizing $\displaystyle\lim_{x\rightarrow-\infty}\frac{x^2+2x-1}{x^3+1}$.}{fig:hzasy5}{figures/fighzasy5}
	
	\item		The highest power of $x$ is $x^2$, which occurs in both the numerator and denominator.  The limit is therefore the ratio of the coefficients of $x^2$, which is $-1/3$. See Figure \ref{fig:hzasy3}(b).
	
	%\mfigure{.55}{Visualizing $\ds\lim_{x\rightarrow\infty}\frac{x^2+2x-1}{1-x-3x^2}$.}{fig:hzasy6}{figures/fighzasy6}
	
	\mtable{.55}{Visualizing the functions in Example \ref{ex_hzasy3}.}{fig:hzasy3}{%
	\noindent\begin{tabular}{c}
	\myincludegraphics{figures/fighzasy5}\\[10pt]
	(a)\\[10pt]
	\myincludegraphics{figures/fighzasy6}\\[10pt]
	(b)\\[10pt]
	\myincludegraphics{figures/fighzasy7}\\[10pt]
	(c)
	\end{tabular}
	}
	
	\item		The highest power of $x$ is in the numerator so the limit will be $\infty$ or $-\infty$.  To see which, consider only the dominant (leading) terms from the numerator and denominator, which are $x^2$ and $-x$.  The expression in the limit will behave like $x^2/(-x) = -x$ for large values of $x$.  Therefore, the limit is $-\infty$. See Figure \ref{fig:hzasy3}(c).
	
	%\mfigure{.3}{Visualizing $\ds\lim_{x\rightarrow\infty}\frac{x^2-1}{3-x}$.}{fig:hzasy7}{figures/fighzasy7}
	\end{enumerate}
\vskip -\baselineskip
}\\

Here are some important facts on limits of transcendental functions involving infinity.

\theorem{thm:lim_transcendental_fn_involving_infty}{Limits of Transcendental Functions Involving Infinity}
{Let $a$ be a real number.
\begin{enumerate}
\item If $a>1$, then $\ds\lim_{x\rightarrow -\infty} a^x = 0$ and $\ds \lim_{x\rightarrow \infty} a^x = \infty$.
\item If $0<a<1$, then $\ds\lim_{x\rightarrow -\infty} a^x = \infty$ and $\ds \lim_{x\rightarrow \infty} a^x = 0$.
\item $\ds \lim_{x \to 0^+} \ln(x) = -\infty$ and $\ds \lim_{x \to \infty} \ln(x) = \infty$
\item $\ds \lim_{x \to -\infty} \tan^{-1}(x) = -\frac{\pi}{2}$ and $\ds \lim_{x \to \infty} \tan^{-1}(x) = \frac{\pi}{2}$
\end{enumerate}
}

A table of values is great for understanding the concept of a limit.  The following example, however, illustrates why a table cannot be used to prove a limit.  Rather, determining a limit must be done analytically.

\example{ex_hzasy4}{A limit for which a table may be misleading}{
Use Theorem \ref{thm:lim_transcendental_fn_involving_infty} to evaluate $\ds\lim_{x\rightarrow\infty}\frac{226}{355}\tan^{-1}(x)$.
}
{The table in Figure~\ref{fig:ex_hzasy4} suggests that the limit is 1.  However, since we can only plug in finite numbers, this is not a proof of this limit.  Using Theorem \ref{thm:lim_transcendental_fn_involving_infty},
\begin{align*}
\lim_{x\rightarrow\infty}\frac{226}{355}\tan^{-1}(x)&=\frac{226}{355}\lim_{x\rightarrow\infty}\tan^{-1}(x)\\
&=\frac{226}{355}\cdot\frac{\pi}{2}\\
&=\frac{113\pi}{355}.
\end{align*}
From this computation, we see that $\lim_{x\rightarrow\infty}\frac{226}{355}\tan^{-1}(x)$ is not 1, but rather $113\pi/355\approx 0.999999915$, also a plausible answer based on the table.
\mtable{.73}{Using this table to compute the limit in Example \ref{ex_hzasy4} suggests the limit is 1.  However, the limit is not 1 as we determine analytically.}{fig:ex_hzasy4}{%
\begin{tabular}{cc}
		$x$ & $\frac{226}{355}\tan^{-1}(x)$ \\ \hline
		\rule{0pt}{11pt} 10 & 0.9365489 \\
		100 & 0.9936339 \\
		1000 & 0.9993633\\
		10000 & 0.9999362\\
		100000 & 0.9999935\\
		1000000 & 0.9999993\\
		\end{tabular} \\[10pt]
}
\vskip -\baselineskip
}\\

The following theorem states that forms like ``$1/\infty$'' or ``$0/\infty$'' are 0.  Basically, it says that if the numerator of a function approaches a number, whereas the denominator approaches $\pm\infty$, then the limit is 0.

\theorem{thm:lim_num_divided_by_infty}{Limit of a Function Whose Denominator Approaches Infinity}
{Let $L$ be a real number.
\begin{enumerate}
\item If $\ds\lim\limits_{x\rightarrow\infty}f(x)=L$ and $\ds\lim\limits_{x\rightarrow\infty}g(x)=\pm\infty$, then $\ds\lim\limits_{x\rightarrow\infty}\frac{f(x)}{g(x)}=0$.
\item If $\ds\lim\limits_{x\rightarrow-\infty}f(x)=L$ and $\ds\lim\limits_{x\rightarrow-\infty}g(x)=\pm\infty$, then $\ds\lim\limits_{x\rightarrow-\infty}\frac{f(x)}{g(x)}=0$.
\end{enumerate}
}

\example{ex_hzasy5}{Finding limits of rational functions}{
Use Theorem \ref{thm:lim_num_divided_by_infty} to evaluate each of the following limits.

\noindent\begin{minipage}[t]{.5\textwidth}
\begin{enumerate}
\item		$\displaystyle\lim_{x\rightarrow\infty}\frac{1}{x+e^x}$
\end{enumerate}
\end{minipage}
\begin{minipage}[t]{.5\textwidth}
\begin{enumerate}\addtocounter{enumi}{1}
\item		$\displaystyle\lim_{x\rightarrow\infty}\frac{\cos x}{x+e^x}$
\end{enumerate}
\end{minipage}
}
{
	\begin{enumerate}
	\item		Clearly the limit of the numerator is $\ds\lim\limits_{x\rightarrow\infty}1=1$.  For the denominator, $\ds\lim\limits_{x\rightarrow\infty}x+e^x=\infty$ as it is the sum of two functions that approach $\infty$.  Using Theorem \ref{thm:lim_num_divided_by_infty}, $$\lim_{x\rightarrow\infty}\frac{1}{x+e^x}=0.$$

\item For the numerator, $\ds\lim\limits_{x\rightarrow\infty}\cos x$ does not exist because $\cos x$ oscillates between $-1$ and $1$ as $x$ approaches infinity.  So Theorem \ref{thm:lim_num_divided_by_infty} cannot be used directly.  However, $\cos x$ is bounded between two numbers, so we can use the Squeeze Theorem.  Note that $-1\leq \cos x \leq 1$ for all real numbers $x$.  Then $\ds\lim_{x\rightarrow\infty}\frac{1}{x+e^x}=0$ by part 1.  For the same reason, $\ds\lim_{x\rightarrow\infty}\frac{-1}{x+e^x}=0$.  Using the Squeeze Theorem,  $$\lim_{x\rightarrow\infty}\frac{\cos x}{x+e^x}=0.$$
	\end{enumerate}
\vskip -\baselineskip
}\\

\clearpage
\vskip 2\baselineskip
\noindent\textbf{\Large Chapter Summary}
\vskip \baselineskip

In this chapter we:
\begin{itemize}
\item defined the limit, 
\item found accessible ways to approximate their values numerically and graphically, 
\item	developed a not--so--easy method of proving the value of a limit ($\epsilon$-$\delta$ proofs),
\item	explored when limits do not exist,
\item	defined continuity and explored properties of continuous functions, and 
\item	considered limits that involved infinity.
\end{itemize}

Why? Mathematics is famous for building on itself and calculus proves to be no exception. In the next chapter we will be interested in ``dividing by 0.'' That is, we will want to divide a quantity by a smaller and smaller number and see what value the quotient approaches. In other words, we will want to find a limit. These limits will enable us to, among other things, determine \textit{exactly} how fast something is moving when we are only given position information.

Later, we will want to add up an infinite list of numbers. We will do so by first adding up a finite list of numbers, then take a limit as the number of things we are adding approaches infinity. Surprisingly, this sum often is finite; that is, we can add up an infinite list of numbers and get, for instance, 42. 

These are just two quick examples of why we are interested in limits. Many students dislike this topic when they are first introduced to it, but over time an appreciation is often formed based on the scope of its applicability.

\printexercises{exercises/01_07_exercises}

%Maybe some practical example here, like the cost to remove 100% of something and vertical asymptotes and the the long term limit of a population to demonstrate limits at infinity.  Or maybe just put these into the exercises.

%\end{document}

%Do both limits to infinity, limits resulting in �infinity�.
%Define more clearly indeterminate form.
%Thm: Rules about limits involving infinity
%Thm: Something about the limits as x ? 8 and rational functions? 