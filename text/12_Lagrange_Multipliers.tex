\section{Method of Lagrange Multipliers}\label{sec:lagrange_multipliers}

This section faces up to a practical problem we encountered at the end of the last section. We often minimize or maximize one function $f(x, y)$ while another function $g(x, y)$ is fixed. There is a constraint on x and y, given by $g(x, y) = k$ for some constant $k$. This restricts the material available, or the funds available, or the energy available. With this constraint, the problem is to do the best possible, maximizing or minimizing $f(x,y)$.

At the absolute minimum of $f(x, y)$, the requirement $g(x, y) = k$ is probably violated.
In that case the minimum point is not allowed. We cannot use $f_x = 0$ and $f_y = 0$ as
those equations don't account for g. We must find equations for the constrained minimum or constrained maximum. They will involve $f_x$ and $f_y$, and also $g_x$ and $g_y$, which give local information about $f$ and $g$. To see the equations, we look at two examples.\\

\example{ex_lagrange1}{}{Minimize $f(x,y) = x^2 + y^2$ subject to the restraint $g(x,y) = 2x + y = k$}{Look at the level curves in Figure \ref{}. They are circles $x^2 + y^2 = c$. When $c$ is small, the circles do not touch the line $2x + y = k$. There are no points that satisfy the constraint, when $c$ is too small. Now increase $c$. Eventually the growing circles $x^2 + y^2 = c$ will just touch the line $x + 2y = k$. The point where they touch is the winner. It gives the smallest value of $c$ that can be achieved on the line. The touching point is $(x_{\text{min}}, y_{\text{min}})$, and the value of $c$ is $f_{\text{min}}$.

What equation describes that point? When the circle touches the line, they are
tangent. They have the same slope. The perpendiculars to the circle and the line go in
the same direction. That is the key fact, which you see in Figure \ref{}.The direction
perpendicular to $f(x,y) = c$ is given by $\nabla f = (f_x,f_y)$. The direction perpendicular to $g(x,y) = k$ is given by $\nabla g = (g_x, g_y)$. The key idea says that those two vectors are
parallel - one gradient vector is a multiple of the other gradient vector, with a multiplier $\lambda$ that is unknown.  That is, $\nabla f = \lambda \nabla g$.  

There are now three unknowns $x$, $y$, and $\lambda$, as well as three equations
\begin{align*}
\frac{\partial f}{\partial x} = \lambda \frac{\partial g}{\partial x} \quad & \Rightarrow 2x = 2\lambda \\
\frac{\partial f}{\partial y} = \lambda \frac{\partial g}{\partial y} \quad & \Rightarrow 2y = \lambda \\
g(x,y) = k \quad & \Rightarrow 2x + y = k.
\end{align*}
In the third equation, substitute $2\lambda$ for $2x$ and $\frac{1}{2}\lambda$ for $y$.  Then $2x+y = \frac{5}{2}\lambda = k$.  Knowing $\lambda = \frac{2}{5}k$, the first two of the above equations give
$$x = \lambda = \frac{2}{5}k, \: y = \frac{1}{2}\lambda = \frac{1}{5}k$$
and so 
$$f_{\text{min}} = \left( \frac{2}{5}k \right)^2 + \left( \frac{1}{5}k \right)^2 = \frac{1}{5}k^2.$$
The winning point is $(x_{\text{min}},y_{\text{min}}) = \left( \frac{2}{5}k, \frac{1}{5}k)$.  It minimizes the distance squared $f = x^2 + y^2 = \frac{1}{5}k^2$ from the origin to the line.}

\theorem{thm:lagrange1}{Lagrange Multipliers with One Constraint}
{At the minimum or maximum of $f(x, y)$ subject to $g(x, y) = k$, the gradient of $f$ is
parallel to the gradient of $g$, with an unknown number $\lambda$ as the multiplier.  That is, 
$$\nabla f = \lambda \nabla g, \text{ and so } \frac{\partial f}{\partial x} = \lambda \frac{\partial g}{\partial x} \text{ and } \frac{\partial f}{\partial y} = \lambda \frac{\partial g}{\partial y} \text{ and } g(x,y) = k.$$
To minimize or maximize $f(x,y)$ subject to $g(x,y) = k$, solve this system of three equations. Evaluate $f(x,y)$ at all solutions $(x,y)$ to this system.  The largest of these is $f_{\text{max}}$, the maximum of $f$ subject to $g = k$, and the smallest is $f_{\text{min}}$, the minimum of $f$ subject to $g = k$.
\index{Lagrange Multipliers}
}

\example{ex_lagrange2}{}{Maximize and minimize $f(x,y) = x^2 + y^2$ on the ellipse $g(x,y) = (x-1)^2 + 4y^2 = 4$.}{As in Figure \ref{}, the circles $x^2 + y^2 = c$ grow until they touch the ellipse.  The touching point is $(x_{\text{min}},y_{\text{min}})$ and that smallest value of $c$ is $f_{\text{min}}$.  As the circles continue to grow, they cut through the ellipse.  Finally there is a point $(x_{\text{max}},y_{\text{max}})$ where the last circle touches.  That largest value of $c$ is $f_{\text{max}}$.  

The minimum and maximum are described by the same rule: the cirle is tangent to the ellipse.  The perpendiculars go in the same direction.  Therefore $\nabla f$ is a multiple of $\nabla g$, and the unknown multiplier is $\lambda$. 
\begin{align*}
\frac{\partial f}{\partial x} = \lambda \frac{\partial g}{\partial x} \quad & \Rightarrow 2x = 2\lambda(x-1) \\
\frac{\partial f}{\partial y} = \lambda \frac{\partial g}{\partial y} \quad & \Rightarrow 2y = 8 \lambda y \\
g(x,y) = k \quad & \Rightarrow (x-1)^2 + 4y^2 = 4.
\end{align*}
The second equation allows two possibilities: $y = 0$ or $\lambda = \frac{1}{4}$.  For the case $y = 0$, the last equation gives $(x-1)^2 = 4$ or $x^2 - 2x -3 = 0$, and thus $x = 3$ or $x = -1$.  Then the first equation gives $\lambda = \frac{3}{2}$ or $\lambda = \frac{1}{2}$.  The values of $f$ are therefore $x^2 + y^2 = 3^2 + 0^2 = 9$ and $x^2 + y^2 = (-1)^2 + 0^2 = 1$.  

In the other case that $\lambda = \frac{1}{4}$, the first equation yields $x = -\frac{1}{3}$.  Then the last equation requires $y^2 = \frac{5}{9}$.  Since $x^2 = \frac{1}{9}$ we find $x^2 + y^2 = \frac{6}{9} = \frac{2}{3}$.  This is the value of $f_{\text{min}}$, and the value of $f_{\text{max}}$ is $9$.  

The equations we started with have four solutions, at which the circle and ellipse are tangent.  The four points are $(3,0)$, $(-1,0)$, $\left( -\frac{1}{3},\frac{\sqrt{5}}{3}\right)$, and $\left( -\frac{1}{3},-\frac{\sqrt{5}}{3} \right)$.  The four values of $f$ are, respectively, $9$, $1$, $\frac{2}{3}$, and $\frac{2}{3}$.
}

Using this method, the three equation are $f_x = \lambda g_x$, $f_y = \lambda g_y$, and $g = k$, with unknowns $x$, $y$, and $\lambda$.  There is no absolute method for solving this system of equations (unless they are all linear equations, then use elimination or Cramer's Rule to solve).  Often, the first two equations yield $x$ and $y$ in terms of $\lambda$, and substituting into $g(x,y) = k$ gives an equation for $\lambda$.

At the minimum, the level curve $f(x,y) = c$ is tangent to the constraint curve $g(x,y) = k$.  If that constraint curve is given parametrically by $x(t)$ and $y(t)$, then minimizing $f\left(x(t),y(t)\right)$ uses the chain rule:
$$\frac{df}{dt} = \frac{\partial f}{\partial x} \frac{dx}{dt} + \frac{\partial f}{\partial y} \frac{dy}{dt}$$
or $\nabla f \cdot \left( \frac{dx}{dt}, \frac{dy}{dt} \right) = 0$.  This is the calculus proof that $\nabla f$ is perpendicular to the curve.  Thus $\nabla f$ is paralell to $\nabla g$.  This means $(f_x,f_y) = \lambda (g_x,g_y)$.\\

\noindent\textbf{\large Maximum and Minimum with Two Constraints}\\

The whole subject of minimization or maximization is called optimization. Its applications to business decisions make up operations research. The special case of linear functions is always important - in this part of mathematics it is called linear programming. A book about those subjects won't fit inside a calculus book, but we can take one more step - to allow a second constraint.

The function to minimize or maximize is now $f(x,y,z)$ and two constraints are given by $g(x,y,z) = k_1$ and $h(x,y,z) = k_2$.  The multipliers are $\lambda_1$ and $\lambda_2$.  We need at least three variables $x$, $y$, and $z$ because two constraints would completely determine $x$ and $y$ without a third.

Figure \ref{} shows the geometry behind these equations.  For convenience $f$ is $x^2 + y^2 + z^2$, so we are minimizing distance squared.  The constraints $g = x + y + z = 9$ and $h = x + 2y + 3z = 20$ are linear - their graphs are planes.  The constraints keep $(x,y,z)$ on both planes - and therefore on the line where they meet.  We are finding the squared distance from $(0,0,0)$ to a line.  

What equations do we solve in this case?  The level surfaces $x^2 + y^2 + z^2 = c$ are spheres.  They grow as $c$ increases.  The first sphere to touch the line is tangent to it, and that touching point gives the minimum solution (the smallest $c$). All three vectors $\nabla f$, $\nabla g$, $\nabla h$ are perpendicular to the line:
\begin{align*}
\text{line tangent to sphere} & \Rightarrow \nabla f \text{ perpendicular to line} \\
\text{line in both planes} & \Rightarrow \nabla g \text{ and } \nabla h \text{ perpendicular to line}
\end{align*}
Thus $\nabla f$, $\nabla g$, and $\nabla h$ are in the same plane - perpendicular to the line.  With three vectors in a plane, $\nabla f$ must be a combination of $\nabla g$ and $\nabla h$:
$$(f_x,f_y,f_z) = \lambda_1 (g_x,g_y,g_z) + \lambda_2 (h_x,h_y,h_z).$$
This is the key equation.  It applies to curved surfaces as well as planes.

\theorem{thm:lagrange2}{Lagrange Multipliers with Two Constraints}
{At the minimum or maximum of $f(x, y, z)$ subject to $g(x, y, z) = k_1$ and $h(x,y,z) = k_2$, the gradient of $f$ is in the same plane as $\nabla g$ and $\nabla h$, or $$\nabla f = \lambda_1 \nabla g + \lambda_2 \nabla h.$$
That is, 
\begin{align*}
\frac{\partial f}{\partial x} & = \lambda_1 \frac{\partial g}{\partial x} + \lambda_2 \frac{\partial h}{\partial x} \\
\frac{\partial f}{\partial y} & = \lambda_1 \frac{\partial g}{\partial y} + \lambda_2 \frac{\partial h}{\partial y} \\
\frac{\partial f}{\partial z} & = \lambda_1 \frac{\partial g}{\partial z} + \lambda_2 \frac{\partial h}{\partial z} \\
\end{align*}
subject to $g(x,y,z) = k_1$ and $h(x,y,z) = k_2$. To minimize or maximize $f(x,y,z)$ subject to $g(x,y,z) = k_1$ and $h(x,y,z) = k_2$, solve this system of five equations in the unknowns $x$, $y$, $z$, $\lambda_1$, and $\lambda_2$. Evaluate $f(x,y,z)$ at all solutions $(x,y,z)$ to this system.  The largest of these is $f_{\text{max}}$, the maximum of $f$ subject to the constraints, and the smallest is $f_{\text{min}}$, the minimum of $f$ subject to the constraints.
\index{Lagrange Multipliers}
}

\example{ex_lagrange3}{}{Minimize $f(x,y,z) = x^2 + y^2 + z^2$ when $g = x+y+z = 9$ and $h = x+2y+3z = 20$.}{In figure \ref{}, the normals to those planes are $\nabla g = (1,1,1)$ and $\nabla h = (1,2,3)$.  The gradient of $f$ is $\nabla f = (2x,2y,2z)$.  The equations become
$$2x = \lambda_1 + \lambda_2, \: 2y = \lambda_1 + 2\lambda_2, \: 2z = \lambda_1 + 3 \lambda_2.$$
Substitute these $x$, $y$, $z$ into the constraint equations $x+y+z=9$ and $x+2y+3z=20$ to get
\begin{align*}
\frac{1}{2}\left( (\lambda_1+\lambda_2) + (\lambda_1+2\lambda_2) + (\lambda_1+3\lambda_2) \right) & = 9 \\
\frac{1}{2}\left( (\lambda_1 + \lambda_2) + 2(\lambda_1+2\lambda_2) + 3(\lambda_1 + 3\lambda_2) +  \right) & = 20
\end{align*}
After multiplying these by $2$, they simplify down to 
$$3\lambda_1 + 6\lambda_2 = 18 \text{ and } 6\lambda_1 + 14\lambda_2 = 40.$$
This is a linear system with solution $\lambda_1 = 2$ and $\lambda_2 = 2$.  The previous equations therefore give $(x,y,z) = (2,3,4)$.  This point gives $f_{\text{min}} = 29$.
}


\printexercises{exercises/12_08_exercises}
