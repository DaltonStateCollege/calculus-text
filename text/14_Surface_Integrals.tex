\section{Surface Integrals}\label{sec:greens_theorem}

The double integral in Green's Theorem is over a region $R$ in the plane.  In this section we consider instead a curved surface $S$ in three-dimensional space, such as part of a sphere or of a cone. When the $z$ is a function of $x$ and $y$, it can be written $z = f(x,y)$, and the graph is a surface, analogous to how $y = f(x)$ was the graph of a particular type of curve.  In more general cases, a curve could be written parametrically, $x = x(t)$ and $y = y(t)$, as functions of a parameter $t$.  Similarly, a surface can be written parametrically as functions of two parameters.\\

When we compute the length of a curve, we had an integral $\int_C \: ds$, a line integral.  Similarly, when we compute the area of a given surface $S$, we will evaluate an integral $\iint_S \: dS$, called a surface integral. This type of integral will can be used to compute surface area even when the surface is not the result of a curve being revolved around an axis, as in Sections \ref{sec:arc_length} and \ref{sec:par_calc}, or when the surface is not a function $z = f(x,y)$. \\

We first start with a review of the case of a surface $S$ given by $z = f(x,y)$ for $(x,y)$ in a region $R$ of the $xy$-plane, as in Section \ref{sec:surface_area}.  When we partitioned the region $R$ into small pieces, such as rectangles, the associated pieces of the surface $z = f(x,y)$ will be almost flat, like small parallelograms. The surface area of such a piece ended up being
$$dS = \sqrt{1 + 1+f_x(x,y)^2+f_y(x,y)^2} \: dx \: dy$$
computed using a cross product of the vectors making up the sides of the parallelogram. The total surface area of $S$ could then be computed as
$$S = \iint_R \: dS = \iint_R \sqrt{1 + 1+f_x(x,y)^2+f_y(x,y)^2} \: dA.$$
We now generalize this method to the situation that the surface $S$ is given parametrically.\\

\noindent\textbf{\large Parametric Surfaces and Surface Area}\\

A \textbf{parametric surface} is one defined by three functions of two parameters, which we will denote by $u$ and $v$.  We will write these functions as $x = x(u,v)$, $y = y(u,v)$, and $z = z(u,v)$, where $(u,v)$ come from a region $R$ in the $uv$-plane.  Points in the $uv$-plane are mapped to three-dimensional points $(x,y,z)$ in three-space.\\

\example{ex_surface_int01}{}{Determine a parametric representation for a cylindrical surface of height $8$ and radius $2$.}{Using cylindrical coordinates, we know that
$$x = 2 \cos \theta \: \text{ and } \: y = 2 \sin \theta$$
and so we can use $u = \theta$ and $v = z$ as our two parameters. This results in parametric equations
$$x(u,v) = 2 \cos u, \: y(u,v) = 2 \sin u, \: z(u,v) = v$$
in the region described by $0 \leq u \leq 2\pi$ and $0 \leq v \leq 8$.  So a rectangle in the $uv$-plane maps to a cylindrical surface in three-dimensional space.
}\\

\example{ex_surface_int02}{}{Determine a parametric representation for the sphere $x^2 + y^2 + z^2 = a^2$, where $a$ is a constant.}{Now using spherical coordinates, we know that
$$x = a \sin\phi \cos\theta, \: y = a \sin\phi \sin\theta, \: z = a \cos\phi$$
and so we can use $u = \phi$ and $v = \theta$ as our two parameters. This results in parametric equations
$$x(u,v) = a \sin u \cos u, \: y(u,v) = a \sin u \sin v, \: z(u,v) = a \cos u$$
for $(u,v)$ in the rectangular region $R$ in the $uv$-plane given by $0 \leq u \leq \pi$ and $0 \leq v \leq 2\pi$. 
}\\

\example{ex_surface_int03}{}{Determine a parametric representation for any surface $z = f(x,y)$ given by a function of two variables, for $(x,y)$ in a region $R$.}{In this case, we can simply use $u = x$ and $v = y$ for the two parameters.  Then
$$x(u,v) = u, \: y(u,v) = v, \: z = f(u,v)$$
gives the parametric equations for the surface. The region $R$ in the $xy$-plane will be the same as the region $R$ in the $uv$-plane in this case.
}\\

Now we find $dS$ for a surface $S$ given parametrically by equations
$$x = x(u,v), \: y = y(u,v), \: z = z(u,v)$$
for $(u,v)$ in a region $R$, following the same idea as before.  Small increments $du$ and $dv$ in the parameters will result in small parallelogram pieces of the surface of area $dS$.  One side of the parallelogram comes from $du$ while the other comes from $dv$.  The two sides are given by vectors $\vec a \: du$ and $\vec b \: dv$ where
$$\vec a = \dfrac{\partial x}{\partial u} \vec i + \dfrac{\partial y}{\partial u} \vec j + \dfrac{\partial z}{\partial u} \vec k \: \text{ and } \: \vec a = \dfrac{\partial x}{\partial v} \vec i + \dfrac{\partial y}{\partial v} \vec j + \dfrac{\partial z}{\partial v} \vec k.$$
To find the area of the parallelogram piece, we take the norm of the cross product, so
$$dS = \left| \vec a \times \vec b \right| \: du \: dv$$
and we integrate this over $R$ to get the total surface area. To simplify the notation, let us write
$$\vec r(u,v) = x(u,v) \vec i + y(u,v) \vec j + z(u,v) \vec k.$$
Then $\vec a = \vec r_u(u,v)$ and $\vec b = \vec r_v(u,v)$ as above.  So $dS$ can now be written as
$$dS = \left| \vec r_u \times \vec r_v \right| \: du \: dv$$ and the total surface area is
$$S = \iint_R \left| \vec r_u \times \vec r_v \right| \: du \: dv.$$\\

\example{ex_surface_int04}{}{Use the above method to compute the surface area of the cone $z = \sqrt{x^2 + y^2}$ up to a height of $4$.}{Using $R$ for the circular region below the cone, the standard parametrization is
$$x = u, \: y = v, \: z = \sqrt{u^2 + v^2}$$
for $(u,v)$ in $R$, as above for a surface given by a function of two variables.  We compute $\vec r_u$, $\vec r_v$ and $\vec r_u \times \vec r_v$ first, using $\vec r = u \vec i + v \vec j + \sqrt{u^2 + v^2} \vec k.$ We have
$$\vec r_u = \vec i + \dfrac{u}{\sqrt{u^2 + v^2}} \vec k \: \text{ and } \: \vec r_v = \vec j + \dfrac{v}{\sqrt{u^2 + v^2}} \vec k$$
and so
$$\vec r_u \times \vec r_v = \left| \begin{array}{ccc} \vec i & \vec j & \vec k \\ 1 & 0 & \frac{u}{\sqrt{u^2 + v^2}} \\ 0 & 1 & \frac{v}{\sqrt{u^2 + v^2}} \end{array} \right| = -\dfrac{v}{\sqrt{u^2 + v^2}} \vec i - \dfrac{u}{\sqrt{u^2 + v^2}} \vec j + \vec k.$$
So the norm $\left| \vec r_u \times \vec r_v \right|$ becomes
$$\left| \vec r_u \times \vec r_v \right| = \sqrt{1 + \dfrac{u^2}{u^2 + v^2} + \dfrac{v^2}{u^2 + v^2}} = \sqrt{2}.$$
Therefore the surface area is 
$$S = \iint_R \sqrt{2} \: du \: dv = 16\sqrt{2} \pi$$
square units, since the region $R$ is a circle of radius 4. Notice that $\left| \vec r_u \times \vec r_v \right|$ is the same as $\sqrt{1 + f_x(x,y)^2 + f_y(x,y)^2}$ as we would have obtained using the method from Section \ref{sec:surface_area}.
}\\

\example{ex_surface_int05}{}{Determine the surface area of the helicoid (a spiral ramp) given parametrically by
$$\vec r(u,v) = u \cos v \vec i + u \sin v \vec j + v \vec k$$ for $0 \leq u \leq 1$ and $0 \leq v \leq 4\pi$.}{Notice that we will be integrating over a rectangle in the $uv$-plane. We can immediately calculate $\vec r_u$ and $\vec r_v$ as
$$\vec r_u = \cos v \vec i + \sin v \vec j \: \text{ and } \vec r_v = -u \sin v \vec i + u \cos v \vec j + \vec k$$
and so
$$\vec r_u \times \vec r_v = \left| \begin{array}{ccc} \vec i & \vec j & \vec k \\ \cos v & \sin v & 0 \\ -u \sin v & u \cos v & 1 \end{array} \right| = \sin v \vec i - \cos v \vec j + u \vec k.$$
Therefore the norm of this cross product is
$$\left| \vec r_u \times \vec r_v \right| = \sqrt{\sin^2 v + \cos^2 v + u^2} = \sqrt{u^2 + 1}.$$
Integrating over the rectangle in the $uv$-plane gives us the surface area
$$S = \int_0^{4\pi} \int_0^1 \sqrt{1+u^2} \: du \: dv = 4\pi \left( \dfrac{1}{2}\left( u \sqrt{1+u^2} + \ln|\sqrt{1+u^2} + u| \right)  \right]_0^1$$
which evaluates to $2\pi \left( \sqrt{2} + \ln(1 + \sqrt{2}) \right)$ square units.
}\\

\noindent\textbf{\large Vector Fields and Flux}\\

Given a parametrized surface $S$ over a region $R$ in the $uv$-plane, we have computed the surface area of $S$ as $\iint_S \: dS = \iint_R 1 \: dS$.  An integral of the form $\iint_S f(x,y,z) \: dS$ is called the \textbf{surface integral} of $f(x,y,z)$ over the surface $S$. To evaluate such a surface integral, one would translate $f(x,y,z)$ to a function of $u$ and $v$ by using the parametric equation $x = x(u,v)$, $y = y(u,v)$, and $z = z(u,v)$ for the given surface.  Then multiply this expression by $dS = \left| \vec r_u \times \vec r_v \right| \: du \: dv$ and integrate over the region $R$ in the $uv$-plane to obtain
$$\iint_S f(x,y,z) \: dS = \iint_R f(x(u,v), y(u,v), z(u,v)) \: \left| \vec r_u \times \vec r_v \right| \: du \: dv.$$
An example of an application of a surface integral is in computing the flow or flux of a vector field, which we discuss next first for a surface given by a function.\\

\definition{defn_flux}{Flux}{Let $z = f(x,y)$ for $(x,y)$ in a region $R$ be a surface $S$, and let $$\vec F(x,y,z) = M(x,y,z) \vec i + N(x,y,z) \vec j + P(x,y,z) \vec k$$ be a vector field. The \textbf{flux} of $\vec F$ across $S$ is  
$$\iint_S \vec F \cdot \vec n \: dS = \iint_R -M \dfrac{\partial f}{\partial x} - N\dfrac{\partial f}{\partial y} + P \: dx \: dy$$
where $\vec n$ is the unit normal vector to the surface. 
}



\printexercises{exercises/14_04_exercises}