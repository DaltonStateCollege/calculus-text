\section{Vector Fields}\label{sec:vector_fields}
We have already introduced double and triple integrals. We went from $\int dx$ to $\int \int  dx dy$ and
$\int \int \int dx dy dz$. All those integrals add up small pieces, and the limit gives area or volume
or mass. What could be more natural than that? I regret to say, after the success of
those multiple integrals, that something is missing. It is even more regrettable that
we didn't notice it. The missing piece is nothing less than the Fundamental Theorem
of Calculus.

The double integral $\int \int dx dy$ equals the area. To compute it, we did not use an
antiderivative of 1. At least not consciously. The method was almost trial and error,
and the hard part was to find the limits of integration. This chapter goes deeper, to
show how the step from a double integral to a single integral is really a new form of
the Fundamental Theorem -- when it is done right.

Two new ideas are needed early, one pleasant and one not. You will like \emph{vector
fields}. You may not think so highly of \emph{line integrals}. Those are ordinary single integrals
like $\int v(x) dx$ , but they go along curves instead of straight lines. The nice step $dx$
becomes the confusing step $ds$. Where $\int dx$ equals the length of the interval, $\int ds$ is
the length of the curve. The point is that regions are enclosed by curves, and we have
to integrate along them. The Fundamental Theorem in its two-dimensional form
(Green's Theorem) connects a \emph{double integral over the region} to \emph{a single integral along
its boundary curve}.

The great applications are in science and engineering, where vector fields are so
natural. But there are changes in the language. Instead of an antiderivative, we speak
about a \emph{potential function}. Instead of the derivative, we take the ``divergence'' and
``curl.'' Instead of area, we compute \emph{flux} and \emph{circulation} and \emph{work}. Examples come
first.

For an ordinary scalar function, the input is a number $x$ and the output is a number
$f(x)$. For a vector field (or vector function), the input is a point $(x, y)$ and the output
is a two-dimensional vector $F(x, y)$. There is a ``field'' of vectors, one at every point.
In three dimensions the input point is $(x, y, z)$ and the output vector $\mathbf{F}$ has three
components.

Let $R$ be a region in the $xy$ plane. A \emph{vector field}  $\mathbf{F}$ assigns to every point
$(x, y)$ in $\mathbb{R}$ a vector $\mathbf{F}(x, y)$ with two components:
$\mathbf{F}(x, y) = M(x, y)\mathbf{i} + N(x, y)\mathbf{j}$.
This plane vector field involves two functions of two variables. They are the components
$M$ and $N$, which vary from point to point. A vector has fixed components; a
vector field has varying components.
A three-dimensional vector field has components $M(x, y, z)$, $N(x, y, z)$, and
$P(x, y, z)$. Then the vectors are $\mathbf{F} = M\mathbf{i} + N\mathbf{j} + P\mathbf{k}$.

\example{ex_lagrange3}{}{The position vector at $(x, y)$ is $\mathbf{R} = x\mathbf{i} + y\mathbf{j}$. Its components are $M = x$
and $N = y$. The vectors grow larger as we leave the origin (Figure 15.la). Their
direction is outward and their length is $\mid R \mid \sqrt{x^2+y^2} = r$. The vector $\mathbf{R}$ is boldface;
the number $r$ is lightface.}

\example{ex_lagrange3}{}{}{ The vector field $\mathbf{R}/r$ consists of unit vectors $\mathbf{u}_r$, pointing outward. We
divide $\mathbf{R} = x\mathbf{i} + y\mathbf{j}$ by its length, at every point except the origin. The components
of $\mathbf{R}/r$ are $M = x/r$ and $N = y/r$. Figure 15.1 shows a third field $\mathbf{R}/r^2$, whose length
is $1/r$.}

Fig. 15.1 The vector fields $\mathbf{R}$, $\mathbf{R}/r$, and $\mathbf{R}/r^2$ are radial. Their lengths are $r$, $1$, and $l/r$.

\example{ex_lagrange3}{}{}{ The spin field or rotation field or turning field goes around the origin
instead of away from it. The field is $\mathbf{S}$. Its components are $M = - y$ and $N = x$:
$$\mathbf{S} = - y\mathbf{i} + x\mathbf{j} \text{ also has length } \mid \mathbf{S} \mid = \sqrt{(-y)^2 + x^2} = r$$
$\mathbf{S}$ is perpendicular to $\mathbf{R}$ -- their dot product is zero: $\mathbf{S}\cdot \mathbf{R} = (- y)(x) + (x)(y) = 0$. The
spin fields $\mathbf{S}/r$ and $\mathbf{S}/r^2$ have lengths $1$ and $1/r$:
The unit vector $\mathbf{S}/r$ is $\mathbf{u}_r$. Notice the blank at $(0,0)$, where this field is not defined.}

Fig. 15.2 The spin fields $\mathbf{S}$, $\mathbf{S}/r$, and $\mathbf{S}/r^2$ go around the origin. Their lengths are $r$, $1$, and $l/r$. 


\example{ex_lagrange3}{}{A Gradient Field.}{ A \emph{gradient field} starts with an ordinary function $f(x, y)$. The components
$M\mathbf{i}$ and $P\mathbf{j}$ are the partial derivatives $\dfrac{\partial f}{\partial x}$ and $\dfrac{\partial f}{\partial y}$. Then the field $\mathbf{F}$ is the gradient
of $f$, denoted $\nabla f$:

$$\mathbf{F} = \nabla f = \dfrac{\partial f}{\partial x}\mathbf{i} + \dfrac{\partial f}{\partial y} \mathbf{j}.$$
This vector field $\nabla f$ is everywhere perpendicular to the level curves $f(x, y) = c$. The
length $\mid \nabla f \mid$ tells how fast $f$ is changing (in the direction it changes fastest). Invent
a function like $f = x^2y$, and you immediately have its gradient field $\mathbf{F} = 2xy\mathbf{i} + x^2\mathbf{j}$.
To repeat, $M$ is $\dfrac{\partial f}{\partial x}$ and $N$ is $\dfrac{\partial f}{\partial y}$.
For every vector field you should ask two questions: \emph{Is it a gradient field?} \emph{If so,
what is $f$?} Here are answers for the radial fields and spin fields:

\begin{mdframed}

The radial fields $\mathbf{R}$, $\mathbf{R}/r$, and $\mathbf{R}/r^2$ are all gradient fields.
	
The spin fields $\mathbf{S}$ and $\mathbf{S}/r$ are not gradients of any $f(x, y)$.

The spin field $S/r^2$ is the gradient of the polar angle $\mathbf{\theta} = \tan^{-1}\left(\dfrac{y}{x}\right)$.
\end{mdframed}

The derivatives of $f = \frac{1}{2}(x^2+ y^2)$ are $x$ and $y$. Thus $\mathbf{R}$ is a gradient field. The gradient
of $f = r$ is the unit vector $\mathbf{R}/r$ pointing outwards. Both fields are perpendicular to
circles around the origin. Those are the level curves of $f = \frac{1}{2}r^2$ and $f = r$.


Question: Is every $\mathbf{R}/r^n$ a gradient field?
Answer: Yes. But among the spin fields, the only gradient is $\mathbf{S}/r^2$.
A major goal of this chapter is to recognize gradient fields by a simple test. The
rejection of $\mathbf{S}$ and $\mathbf{S}/r$ will be interesting. For some reason $-y\mathbf{i} + x\mathbf{j}$ is rejected and
$y\mathbf{i} + x\mathbf{j}}$ is accepted. (Can you tell what it is the gradient of?) The acceptance of $\mathbf{S}/r^2$ as the
gradient of $f = 0$ contains a surprise at the origin (Section 15.3).
Gradient fields are called \emph{conservative}. The function $f$ is the \emph{potential function}.
These words, and the next examples, come from physics and engineering.

EXAMPLE5 The velocity field is $\mathbf{V}$ and the flow field is $\rho \mathbf{V}$.

Suppose: fluid moves steadily down a pipe. Or a river flows smoothly (no waterfall).
Or the air circulates in a fixed pattern. The velocity can be different at different points,
but there is no change with time. The velocity vector $\mathbf{V}$ gives the direction of flow
and speed of flow at every point.
In reality the velocity field is $\mathbf{V}(x, y, z)$, with three components $M$, $N$, and $P$. Those are
the velocities $v_1$, $v_2$, and $v_3$, in the $x$, $y$, and $z$ directions. The speed $\mid \mathbf{V} \mid$ is the length: 
$\mid\mathbf{V}\mid^2 = v_1^2 + v_2^2 +v_3^2$. In a ``plane flow'' the $\mathbf{k}$ component is zero, and the velocity field is
$v_1\mathbf{i}+v_2\mathbf{j}= M\mathbf{i}+ N\mathbf{j}$.

Fig. 15.3 A steady velocity field $\mathbf{V}$ and two force fields $\mathbf{F}$. 

For a compact disc or a turning wheel, $\mathbf{V}$ is a spin field ($V =\omega \mathbf{S}$, $\omega=$ angular
velocity). A tornado might be closer to $\mathbf{V} =\mathbf{S}/r^2$ (except for a dead spot at the center).
An explosion could have $\mathbf{V} =\mathbf{R}/r^2$. A quieter example is flow in and out of a lake
with steady rain as a source term.
The \emph{flow field} $\rho \mathbf{V}$ is the density $\rho$ times the velocity field. While $\mathbf{V}$ gives the rate of
movement, $\rho \mathbf{V}$ gives the rate of movement of mass. A greater density means a greater
rate $\mid \rho \mathbf{V} \mid$ of "mass transport." It is like the number of passengers on a bus times the
speed of the bus.

EXAMPLE 6 Force fields from gravity: $\mathbf{F}$ is downward in the classroom, $\mathbf{F}$ is radial
in space.
When gravity pulls downward, it has only one nonzero component: $\mathbf{F} = -mg\mathbf{k}$. This
assumes that vectors to the center of the Earth are parallel -- almost true in a classroom.
Then F is the gradient of $-mgz$ (note that $\dfrac{\partial f}{\partial z} = -mg$).
\emph{In physics the usual potential is not $-mgz$ but $+mgz$.} The force field is \emph{minus} $\nabla f$
in electrical engineering as well. Electrons flow from high potential to low potential.
The mathematics is the same, but the sign is reversed.

In space, the force is radial inwards: $\mathbf{F} = -mMG\mathbf{R}/r^3$. Its magnitude is proportional
to $1/r^2$ (Newton's inverse square law). The masses are $m$ and $M$, and the
gravitational constant is $G =6.672 x 10^{-11}$ -- with distance in meters, mass in kilograms,
and time in seconds. The dimensions of $G$ are $(\text{force})(\text{distance})^2/(\text{mass})^2$. This
is different from the acceleration $g =9.8m/\text{sec}^2$, which already accounts for the mass
and radius of the Earth.

Like all radial fields, gravity is a gradient field. It comes from a potential $f$:
$$f=\dfrac{mMG}{r} \text{ and }\dfrac{\partial f}{\partial x}=-\dfrac{mMGx}{r^3} \text{ and }\dfrac{\partial f}{\partial y}=-\dfrac{mMGy}{r^3} \text{ and }\dfrac{\partial f}{\partial z}=-\dfrac{mMGz}{r^3}.$$ 

EXAMPLE 7 (a short example) Current in a wire produces a magnetic field $\mathbf{B}$. It is
the spin field $\mathbf{S}/r^2$ around the wire, times the strength of the current.

STREAMLINES AND LINES OF FORCE
Drawing a vector field is not always easy. Even the spin field looks messy when the
vectors are too long (they go in circles and fall across each other). \emph{The circles give a
clearer picture than the vectors}. In any field, the vectors are tangent to ``\emph{field lines}'' which
in the spin case are circles.

DEFINITION $\mathbf{C}$ is a \emph{field line} or \emph{integral curve} if the vectors $\mathbf{F}(x, y)$ are tangent to $C$.
The slope $dy/dx$ of the curve $C$ equals the slope $N/M$ of the vector $\mathbf{F} =M\mathbf{i} + N\mathbf{j}$:

$$\dfrac{dy}{dx}=\dfrac{N(x,y)}{M(x,y)} \left(=-\dfrac{x}{y} \text{ for the spin field}\right).$$

We are still drawing the field of vectors, but now they are infinitesimally short.
They are connected into curves! What is lost is their length, because $\mathbf{S}$, $\mathbf{S}/r$, and
$\mathbf{S}/r^2$ all have the same field lines (circles). For the position field $\mathbf{R}$ and gravity field
$R/r^3$, the field lines are rays from the origin. In this case the ``curves'' are actually
straight.

EXAMPLE 8 Show that the field lines for the velocity field $\mathbf{V} =y\mathbf{i} +x\mathbf{j}$ are hyperbolas.
$$\dfrac{dy}{dx}=\dfrac{N}{M}=\dfrac{x}{y} \Rightarrow ydy=xdx \Rightarrow \frac{1}{2}y^2-\frac{1}{2}x^2=\text{constant}.$$

Fig. 15.4 Velocity fields are tangent to streamlines. Gradient fields also have equipotentials.


At every point these hyperbolas line up with the velocity $\mathbf{V}$. \emph{Each particle of fluid
travels on a field line}. In fluid flow those hyperbolas are called \emph{streamlines}. Drop a
leaf into a river, and it follows a streamline. Figure 15.4 shows the streamlines for a
river going around a bend.


Don't forget the essential question about each vector field. Is it a gradient field?
For $\mathbf{V} = y\mathbf{i} + x\mathbf{j}$ the answer is yes, and the potential is $f = xy$:

$$\nabla(xy)=\dfrac{\partial f}{\partial x}\mathbf{i}+\dfrac{\partial f}{\partial y}\mathbf{j}=y\mathbf{i}+x\mathbf{j}$$

When there is a potential, it has level curves. They connect points of equal potential,
so the curves $f(x, y) = c$ are called \emph{equipotentials}. Here they are the curves $xy = c$ -- also
hyperbolas. Since gradients are perpendicular to level curves, \emph{the streamlines are
perpendicular to the equipotentials}. Figure 15.4 is sliced one way by streamlines and
the other way by equipotentials.

In general, a gradient field $\mathbf{F} =\dfrac{\partial f}{\partial x}\mathbf{i}+\dfrac{\partial f}{\partial y}\mathbf{j}$ is tangent to the field lines (streamlines) and perpendicular to the equipotentials (level curves of $f$).

In the gradient direction $f$ changes fastest. In the level direction $f$ doesn't change at
all. The chain rule along $f (x, y) = c$ proves these directions to be perpendicular:
$$\dfrac{\partial f}{\partial x}\dfrac{dx}{dt}+\dfrac{\partial f}{\partial y}\dfrac{dy}{dt}=0 \text{ or } (\nabla f)\cdot(tangent to level curve)=0.$$

EXAMPLE 9 The streamlines of $\mathbf{S}/r^2$ are circles around $(0,0)$. The equipotentials are
rays $\theta = c$. Add rays to Figure 15.2 for the gradient field $\mathbf{S}/r^2$.

For the gravity field those are reversed. A body is pulled in along the field lines (rays).
The equipotentials are the circles where $f = 1/r$ is constant. The plane is crisscrossed
by ``orthogonal trajectories'' -- curves that meet everywhere at right angles.
If you bring a magnet near a pile of iron filings, a little shake will display the field
lines. In a force field, they are ``lines of force.'' \emph{Here are the other new words.}

Vector field $\mathbf{F}(x, y, z) = M\mathbf{i} + N\mathbf{j} + P\mathbf{k}$ Plane field $\mathbf{F} = M(x, y)\mathbf{i} + N(x, y)\mathbf{j}$
Radial field: multiple of $\mathbf{R} = x\mathbf{i} + y\mathbf{j} + z\mathbf{k}$ Spin field: multiple of  $\mathbf{S}= - y\mathbf{i} + x\mathbf{j}$
Gradient field = conservative field: $M = \dfrac{\partial f}{\partial x}, N = \dfrac{\partial f}{\partial y}, P = \dfrac{\partial f}{\partial z}$
Potential $f(x, y$ (not a vector) Equipotential curves: $f(x, y) = c$.
Streamline = field line = integral curve: a curve that has $\mathbf{F}(x, y)$ as its tangent
vectors. 

%%\printexercises{exercises/13_06_exercises}