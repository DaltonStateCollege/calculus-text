\section{Vector Fields}\label{sec:vector_fields}
Chapter 14introduced double and triple integrals. We went from $\int dx$ to $\int \int  dx dy$ and
$\int \int \int dx dy dz$. All those integrals add up small pieces, and the limit gives area or volume
or mass. What could be more natural than that? I regret to say, after the success of
those multiple integrals, that something is missing. It is even more regrettable that
we didn't notice it. The missing piece is nothing less than the Fundamental Theorem
of Calculus.

The double integral $\int \int dx dy$ equals the area. To compute it, we did not use an
antiderivative of 1. At least not consciously. The method was almost trial and error,
and the hard part was to find the limits of integration. This chapter goes deeper, to
show how the step from a double integral to a single integral is really a new form of
the Fundamental Theorem-when it is done right.

Two new ideas are needed early, one pleasant and one not. You will like vector
fields. You may not think so highly of line integrals. Those are ordinary single integrals
like $\int v(x) dx$ , but they go along curves instead of straight lines. The nice step $dx$
becomes the confusing step $ds$. Where $\int dx$ equals the length of the interval, $\int  ds$ is
the length of the curve. The point is that regions are enclosed by curves, and we have
to integrate along them. The Fundamental Theorem in its two-dimensional form
(Green's Theorem) connects a \textbf{double integral over the region to a single integral along
its boundary curve}.

The great applications are in science and engineering, where vector fields are so
natural. But there are changes in the language. Instead of an antiderivative, we speak
about a \textbf{potential function}. Instead of the derivative, we take the "divergence" and
"curl." Instead of area, we compute \textbf{flux} and \textbf{circulation} and \textbf{work}. Examples come
first.

15.1 Vector Fields

For an ordinary scalar function, the input is a number x and the output is a number
$f(x)$. For a vector field (or vector function), the input is a point $(x, y)$ and the output
is a two-dimensional vector $F(x, y)$. There is a "field" of vectors, one at every point.
In three dimensions the input point is $(x, y, z)$ and the output vector F has three
components.

DEFINITION Let $R$ be a region in the $xy$ plane. A \textbf{vector field}  $F$ assigns to every point
$(x, y)$ in $R$ a vector $F(x, y)$ with two components:
$F(x, y) = M(x, y)i + N(x, y)j. (1)$
This plane vector field involves two functions of two variables. They are the components
M and N, which vary from point to point. A vector has fixed components, a
vector field has varying components.
A three-dimensional vector field has components M(x, y, z) and N(x, y, z) and
P(x, y, 2). Then the vectors are F = Mi + Nj + Pk.
EXAMPLE 1 The position vector at (x, y) is R = xi + yj. Its components are M = x
and N = y. The vectors grow larger as we leave the origin (Figure 15.la). Their
direction is outward and their length is IRI = J;i?;i = r, The vector R is boldface,
the number r is lightface.
EXAMPLE 2 The vector field R/r consists of unit vectors u,, pointing outward. We
divide R = xi + yj by its length, at every point except the origin. The components
of Rlr are M = xlr and N = y/r. Figure 15.1 shows a third field ~/r~, whose length
is 1 /r.
Fig. 15.1 The vector fields R and R/r and R/r2 are radial. Lengths r and 1 and l/r
EXAMPLE 3 The spin field or rotation field or turning field goes around the origin
instead of away from it. The field is S. Its components are M = - y and N = x:
S = - yi + xj also has length IS1 = J(-y)2 + x2 = r. (2)
S is perpendicular to R-their dot product is zero: S R = (- y)(x) + (x)(y) = 0. The
spin fields S/r and S/r2 have lengths 1 and llr:
The unit vector S/r is u,. Notice the blank at (O,O), where this field is not defined.
Fig. 15.2 The spin fields S and S/r and S/r2 go around the origin. Lengths r and 1 and l/r. 
15.1 Vector Fields
EXAMPLE 4 A gradientfield starts with an ordinary function f(x, y). The components
M and PJ are the partial derivatives df/dx and dfldy. Then the field F is the gradient
off:
F = grad f = Vf= dfldx i + dfldy j. (3)
This vector field grad f is everywhere perpendicular to the level curves f(x, y) = c. The
length lgrad f 1 tells how fast f is changing (in the direction it changes fastest). Invent
a function like f = x2y, and you immediately have its gradient field F = 2xyi + x2j.
To repealt, M is df/dx and N is dfldy.
For every vector field you should ask two questions: Is it a gradient field? If so,
what is f? Here are answers for the radial fields and spin fields:
MA The radial fields R and R/r and ~/r~ are a11 gradient fields.
The spin fields S and S/r are not gradients of any f(x, y),
The spin field S/r2 is the gradient of the polar angle 0 = tan- '(ylx).
The derivatives off = f(x2+ y2) are x and y. Thus R is a gradient field. The gradient
off = r is the unit vector R/r pointing outwards. Both fields are perpendicular to
circles around the origin. Those are the level curves off = f r2 and f = r.
Question Is every R/rn a gradient field?
Answer Yes. But among the spin fields, the only gradient is S/r2.
A ma-jor goal of this chapter is to recognize gradient fields by a simple test. The
rejection of S and S/r will be interesting. For some reason -yi + xj is rejected and
yi + xj is accepted. (It is the gradient of .) The acceptance of S/r2 as the
gradient off = 0 contains a surprise at the origin (Section 15.3).
Gradient fields are called conservative. The function f is the potential function.
These words, and the next examples, come from physics and engineering.
EXAMPLE5 The velocity field is V and the flow field is pV.
Suppose: fluid moves steadily down a pipe. Or a river flows smoothly (no waterfall).
Or the air circulates in a fixed pattern. The velocity can be different at different points,
but there is no change with time. The velocity vector V gives the direction offlow
and speed of Jow at every point.
In reality the velocity field is V(x, y, z), with three components M, N, P. Those are
the velocities v,, v2, v, in the x, y, z directions. The speed (VI is the length: IVI2 =
v: + v: -t v:. In a "plane flow" the k component is zero, and the velocity field is
v,i+v2j= Mi+ Nj.
gravity
F = -R//."
Fig. 15.3 A steady velocity field V and two force fields F. 
15 Vector Calculus
For a compact disc or a turning wheel, V is a spin field (V =US, co =angular
velocity). A tornado might be closer to V =S/r2 (except for a dead spot at the center).
An explosion could have V =R/r2. A quieter example is flow in and out of a lake
with steady rain as a source term.
TheJlowJield pV is the density p times the velocity field. While V gives the rate of
movement, pV gives the rate of movement of mass. A greater density means a greater
rate IpVJof "mass transport." It is like the number of passengers on a bus times the
speed of the bus.
EXAMPLE 6 Force fields from gravity: F is downward in the classroom, F is radial
in space.
When gravity pulls downward, it has only one nonzero component: F = -mgk. This
assumes that vectors to the center of the Earth are parallel-almost true in a classroom.
Then F is the gradient of -mgz (note dfldz = -mg).
In physics the usual potential is not -mgz but +mgz. The force field is minus grad f
also in electrical engineering. Electrons flow from high potential to low potential.
The mathematics is the same, but the sign is reversed.
In space, the force is radial inwards: F = -mMGR/r3. Its magnitude is proportional
to l/r2 (Newton's inverse square law). The masses are m and M, and the
gravitational constant is G =6.672 x 10-"--with distance in meters, mass in kilograms,
and time in seconds. The dimensions of G are (force)(di~tance)~/(mass)~. This
is different from the acceleration g =9.8m/sec2, which already accounts for the mass
and radius of the Earth.
Like all radial fields, gravity is a gradient field. It comes from a potential f:
EXAMPLE 7 (a short example) Current in a wire produces a magnetic field B. It is
the spin field S/r2 around the wire, times the strength of the current.
STREAMLINES AND LINES OF FORCE
Drawing a vector field is not always easy. Even the spin field looks messy when the
vectors are too long (they go in circles and fall across each other). The circles give a
clearer picture than the vectors. In any field, the vectors are tangent to "jield lineswwhich
in the spin case are circles.
DEFINITION C is afield line or integral curve if the vectors F(x, y) are tangent to C.
The slope dyldx of the curve C equals the slope N/M of the vector F =Mi + Nj:
We are still drawing the field of vectors, but now they are infinitesimally short.
They are connected into curves! What is lost is their length, because S and S/r and
S/r2 all have the same field lines (circles). For the position field R and gravity field
R/r3, the field lines are rays from the origin. In this case the "curves" are actually
straight.
EXAMPLE 8 Show that the field lines for the velocity field V =yi +xj are hyperbolas.
dy N x -- --- * y dy =x dx *y2 -x2 =constant.
~X-M-~ 
15.1 Vector Fields
reamlines x2 - y2 = C
Fig. 15.4 Velocity fields are tangent to streamlines. Gradient fields also have equipotentials.
At every point these hyperbolas line up with the velocity V. Each particle of fluid
travels on afield line. In fluid flow those hyperbolas are called streamlines. Drop a
leaf into a river, and it follows a streamline. Figure 15.4 shows the streamlines for a
river going around a bend.
Don't forget the essential question about each vector field. Is it a gradient field?
For V = yi + xj the answer is yes, and the potential is f = xy:
the gradient of xy is (8flax)i + (8flay)j = yi + xj. (7)
When there is a potential, it has level curves. They connect points of equal potential,
so the curves f (x, y) = c are called equipotentials. Here they are the curves xy = calso
hyperbolas. Since gradients are perpendicular to level curves, the streamlines are
perpendicular to the equipotentials. Figure 15.4 is sliced one way by streamlines and
the other way by equipotentials.
A gradient field F = afldx i + afldy j is tangent to the field lines (streamlines)
and perpendicular to the equipotentials (level curves off).
In the gradient direction f changes fastest. In the level direction f doesn't change at
all. The chain rule along f (x, y) = c proves these directions to be perpendicular:
af dx af dy -- + - = 0 or (grad f) (tangent to level curve) = 0.
ax dt oy dt
EXAMPLE 9 The streamlines of S/r2 are circles around (0,O). The equipotentials are
rays 0 = c. Add rays to Figure 15.2 for the gradient field S/r2.
For the gravity field those are reversed. A body is pulled in along the field lines (rays).
The equipotentials are the circles where f = llr is constant. The plane is crisscrossed
by "orthogonal trajectories9'-curves that meet everywhere at right angles.
If you bring a magnet near a pile of iron filings, a little shake will display the field
lines. In a force field, they are "lines of force." Here are the other new words.
Vector hid F, y, z) = Mi + Nj + Pk Plane field F = M(x, y)i + N(x, y)j
Radial field: multiple of R = xi + yj + zk Spifl field: multiple of  = - yi + xj
Gradient ktd = conservative field: A4 = wax, N = af, P = 18~
Potmtialf(x, yf (not a vector) Equipotential curves f(x, y) = c
Streamline = field line = integral curve: a curve that has F(x, y) as its tangent
vectors. 
554 15 Vector Calculus
15.1 EXERCISES
Read-through questions
A vector field assigns a a to each point (x, y) or (x, y, z).
In two dimensions F(x,y) = b i + c j. An example is
the position field R = d . Its magnitude is IRI = e
and its direction is f . It is the gradient field for f =
g . The level curves are h , and they are i to
the vectors R.
Reversing this picture, the spin field is S = i . Its magnitude
is IS1 = k and its direction is I . It is not a
gradient field, because no function has af/ax = m and
af/ay = n . S is the velocity field for flow going 0 . The streamlines or P lines or integral s are r . The flow field pV gives the rate at which s is moved
by the flow.
A gravity field from the origin is proportional to F = t
which has IF1 = u . This is Newton's v square law.
It is a gradient field, with potential f = w .The equipotential
curves f(x, y) = c are x . They are Y to the field
lines which are . This illustrates that the A of a
function f(x, y) is B to its level curves.
The velocity field yi + xj is the gradient off = c . Its
streamlines are D .The slope dyldx of a streamline equals
the ratio E of velocity components. The field is F to
the streamlines. Drop a leaf onto the flow, and it goes along
G .
Find a potential f(x, y) for the gradient fields 1-8. Draw the
streamlines perpendicular to the equipotentials f(x, y) = c.
1 F = i + 2j (constant field) 2 F = xi +j
7 F=xyi+ j 8 F=i+ j
9 Draw the shear field F =xj. Check that it is not a gradient
field: If af/ax =0 then af/ay =x is impossible. What are the
streamlines (field lines) in the direction of F?
10 Find all functions that satisfy af/ax = -y and show that
none of them satisfy af/ay = x. Then the spin field S =
-yi + xj is not a gradient field.
Compute af/ax and af/ay in 11-18. Draw the gradient field
F =padf and the equipotentials f(x, y) = c:
15f=x2-y2 16 f = ex cos y
Find equations for the streamlines in 19-24 by solving dyldx =
N/M (including a constant C). Draw the streamlines.
21 F =S (spin field) 22 F =S/r (spin field)
23 F =grad (xly) 24 F =grad (2x + y).
25 The Earth's gravity field is radial, but in a room the field
lines seem to go straight down into the floor. This is because
nearby field lines always look .
26 A line of charges produces the electrostatic force field F =
R/r2=(xi + yj)/(x2+ y2). Find the potential f(x, y). (F is also
the gravity field for a line of =asses.)
In 27-32 write down the vector fields Mi + Nj.
27 F points radially away from the origin with magnitude 5.
28 The velocity is perpendicular to the curves x3 + y3 =c and
the speed is 1.
29 The gravitational force F comes from two unit masses at
(0,O) and (1,O).
30 The streamlines are in the 45" direction and the speed is 4.
31 The streamlines are circles clockwise around the origin
and the speed is 1.
32 The equipotentials are the parabolas y = x2+ c and F is
a gradient field.
33 Show directly that the hyperbolas xy = 2 and x2 -y2 = 3
are perpendicular at the point (2, l), by computing both slopes
dyldx and multiplying to get -1.
34 The derivative off (x, y) = c isf, +f,(dy/dx) =0. Show that
the slope of this level curve is dyldx = -MIN. It is perpendicular
to streamlines because (- M/N)(N/M)= .
35 The x and y derivatives of f(r) are dfldx = and
dflay =-by the chain rule. (Test f =r2.) The equipotentials
are .
36 F =(ax + by)i + (bx + cy)j is a gradient field. Find the
potential f and describe the equipotentials.
37 True or false:
I. The constant field i + 2k is a gradient field.
2. For non-gradient fields, equipotentials meet streamlines
at non-right angles.
3. In three dimensions the equipotentials are surfaces
instead of curves.
4. F = x2i+ y2j+ z2k points outward from (0,0,0)-
a radial field.
38 Create and draw f and F and your own equipotentials
and streamlines. 
15.2 Line Integrals 555
39 How can different vector fields have the same streamlines? 40 Draw arrows at six or eight points to show the direction
Can they have the same equipotentials? Can they have the and magnitude of each field:
same f? (a) R +S (b) Rlr -S/r (c) x2i+x2j (d)yi

%%\printexercises{exercises/13_06_exercises}