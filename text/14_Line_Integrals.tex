\section{Line Integrals}\label{sec:line_integrals}
15.2
along a
Line
curve. It
Integrals
can equal an area, but
.-
that is a special
case and not typical. Instead of area, here are two important line integrals in physics
and engineering:
Work along a curve = F T ds Flow across a curve =
In the first integral, F is a force field. In the second integral, F is a flow field. Work
is done in the direction of movement, so we integrate F T. Flow is measured through
the curve C, so we integrate F n. Here T is the unit tangent vector, and F T is the
force cornponent along the curve. Similarly n is the unit normal vector, at right angles
with T. Then F n is the component of flow perpendicular to the curve.
We will write those integrals in several forms. They may never be as comfortable
as J y(x) dx, but eventually we get them under control. I mention these applications
early, so you can see where we are going. This section concentrates on work, and
flow comes later. (It is also calledflux-the Latin word for flow.) You recognize ds
as the step along the curve, corresponding to dx on the x axis. Where f dx gives the
length of an interval (it equals b - a), 5 ds is the length of the curve.
EXAMPLE 1 Flight from Atlanta to Los Angeles on a straight line and a semicircle.
According to Delta Airlines, the distance straight west is 2000 miles. Atlanta is at
(1000,O) and Los Angeles is at (-1000, O), with the origin halfway between. The
semicircle route C has radius 1000. This is not a great circle route. It is more of a
"flat circle," which goes north past Chicago. No plane could fly it (it probably goes
into space).
The equation for the semicircle is x2 + y2 = 10002. Parametrically this path is x =
1000 cos t, y = 1000 sin t. For a line integral the parameter is better. The plane leaves
Atlanta at t = 0 and reaches L.A. at t = n, more than three hours later. On the straight
2000-mile path, Delta could almost do it. Around the semicircle C, the distance is
lOOOn miles and the speed has to be 1000 miles per hour. Remember that speed is
distance ds divided by time dt:
dsldt = ,/(dx~dt)~ + (dyldt)' = l000,/(- sin t)2 + (cos t)2 = 1000. (1)
The tangent vector to C is proportional to (dxldt, dyldt) = (-1000 sin t, 1000 cos t).
But T is a unit vector, so we divide by 1000-which is the speed.
Suppose the wind blows due east with force F = Mi. The components are M and
zero. Foir M =constant, compute the dot product F *T and the work -2000 M:
FwT= Mi*(-sin t i+cos tj)= M(-sin t)+O(cos t)= - M sin t 
15 Vector Calculus
Work is force times distance moved. It is negative, because the wind acts against the
movement. You may point out that the work could have been found more simplygo
2000 miles and multiply by - M. I would object that this straight route is a
dzrerent path. But you claim that the path doesn't matter-the work of the wind is
-2000M on every path. I concede that this time you are right (but not always).
Most line integrals depend on the path. Those that don't are crucially important.
For a gradient field, we only need to know the starting point P and the finish Q.
158 When F is the gradient of a potential function f (x, y), the work J, F T ds
depends only on the endpoints P and Q. The work is the change in
if F = afpx i + af/ay j then F T ds = f (Q) -f(P).
When F = Mi, its components M and zero are the partial derivatives off = Mx. To
compute the line integral, just evaluate f at the endpoints. Atlanta has x = 1000, Los
Angeles has x = - 1000, and the potential function f = Mx is like an antiderivative:
work = f (Q) - f (P) = M(- 1000) - M(1000) = - 2000M. (3)
LAX LAX - 1000 , - 1000 1000
J F . Tdr = - 2000M depends on path
Fig. 15.5 Force Mi, work -2000M on all paths. Force Myi, no work on straight path.
May I give a rough explanation of the work integral 5 F T ds? It becomes clearer
when the small movement Tds is written as dx i + dy j. The work is the dot product
with F:
The infinitesimal work is df: The total work is 5 df= f(Q) - f (P). This is the Fundamental
Theorem for a line integral. Only one warning: When F is not the gradient of any
f (Example 2), the Theorem does not apply.
EXAMPLE 2 Fly these paths against the non-constant force field F = Myi. Compute
the work.
There is no force on the straight path where y = 0. Along the x axis the wind does
no work. But the semicircle goes up where y = 1000 sin t and the wind is strong:
F*T=(Myi)*(-sinti+costj)= -My sin t= - lOOOM sin2t
This work is enormous (and unrealistic). But the calculations make an important
point-everything is converted to the parameter t. The second point is that F = Myi
is not a gradient field. First reason: The work was zero on the straight path and 
15.2 Line Integrals
nonzero on the semicircle. Second reason: No function has df/ dx = My and df /dy =
0. (The first makes f depend on y and the second forbids it. This F is called a shear
force.) Without a potential we cannot substitute P and Q-and the work depends
on the path.
THE DEFINITION OF LINE INTEGRALS
We go back to the start, to define F T ds. We can think of F T as a function g(x, y)
along the path, and define its integral as a limit of sums:
N IC g(~, y)ds = limit of xi, yi)Asi as (As),,,., -i 0.
i= 1 (5)
The points (xi, y,) lie on the curve C. The last point Q is (x,, y,); the first point P is
(xo, yo). The step Asi is the distance to (xi, yi) from the previous point. As the steps
get small (As -, 0) the straight pieces follow the curve. Exactly as in Section 8.2, the
special case g = 1 gives the arc length. As long as g(x, y) is piecewise continuous
(jumps allowed) and the path is piecewise smooth (corners allowed), the limit exists
and defines the line integral.
When g is the density of a wire, the line integral is the total mass. When g is F T,
the integral is the work. But nobody does the calculation by formula (5). We now
introduce a parameter t-which could be the time, or the arc length s, or the distance
x along the base.
The diflerential ds becomes (ds/dt)dt. Everything changes over to t:
The curve starts when t = a, runs through the points (x(t), y(t)), and ends when t = b.
The square root in the integral is the speed dsldt. In three dimensions the points on
C are (x(t), y(t), z(t)) and (dz/dt)l is in the square root.
EXAMPLE 3 The points on a coil spring are (x, y, z) = (cos t, sin t, t). Find the mass
of two complete turns (from t = 0 to t = 44 if the density is p = 4.
Solution The key is (d~/dt)~ + (d~/dt)~ + (d~ldt)~ = sin2t + cos2t + 1 = 2. Thus
dsldt = fi. To find the mass, integrate the mass per unit length which is g = p = 4:
That is a line integral in three-dimensional space. It shows how to introduce t. But
it misses the main point of this section, because it contains no vector field F. This
section is about work, not just mass.
DIFFERENT FORMS OF THE WORK INTEGRAL
The work integral IF T ds can be written in a better way. The force is F = Mi + Nj.
A small step along the curve is dx i + dy j. Work is force times distance, but it is only
the force component along the path that counts. The dot product F -Tds finds that
component automatically. 
15 Vector Calculus
I 15C The vector to a point on C is R = xi + yj. Then dR = Tds = dx i + dy j:
I Along a space curve the work is jF*~ds=f~*d~=j~dx+~d~+ ~dz.
The product M dx is (force in x direction)(movement in x direction). This is zero if
either factor is zero. When the only force is gravity, pushing a piano takes no work.
It is friction that hurts. Carrying the piano up the stairs brings in Pdz, and the total
work is the piano weight P times the change in z.
To connect the new I F dR with the old IF *T ds, remember the tangent vector
T. It is dRlds. ~herefoie Tds is dR. The best for computations is dR, because the
unit vector T has a division by dsldt = ,/(d~/dt)~ + (d~ldt)~. Later we multiply by this
square root, in converting ds to (dsldtjdt. It makes no sense to compute the square
root, divide by it, and then multiply by it. That is avoided in the improved form
~M~x+N~Y.
EXAMPLE 4 Vector field F = - yi + xj, path from (1,O) to (0, 1): Find the work.
Note 1 This F is the spin field S. It goes around the origin, while R = xi + yj goes
outward. Their dot product is F R = - yx + xy = 0. This does not mean that
F dR = 0. The force is perpendicular to R, but not to the change in R. The work to
move from (I, 0) to (0, I), x axis to y axis, is not zero.
Note 2 We have not described the path C. That must be done. The spin field is not
a gradient field, and the work along a straight line does not equal the work on a
quarter-circle:
straight line x = 1 - t, y = t quarter-circle x = cos t, y = sin t.
Calculation of work Change F dR = M dx + N dy to the parameter t:
Straight line: - y dx + x dy = - t(- dt) + (1 - t)dt = 1 l lo1
S 7T
Quarter-circle: - y dx + x dy = -sin t(- sin t dt) + cos t(cos t dt) = -. 2
General method The path is given by x(t) and y(t). Substitute those into M(x, y)
and N(x, y)-then F is a function of t. Also find dxldt and dyldt. Integrate
M dxldt + N dyldt from the starting time t to the finish.
I work 7[: / 2 no work ' work F.dR = 1
Fig.15.6 Threepath~for~F~dR=~-~dx+.udy=l,n/2,0. 
15.2 Llne Integrals
For practice, take the path down the x axis to the origin (x = 1 - t, y = 0). Then
go up the y axis (x = 0, y = t - 1). The starting time at (1,O) is t = 0. The turning time
at the origin is t = 1. The finishing time at (0, 1) is t = 2. The integral has two parts
because this new path has two parts:
Bent path: J-ydx+xdy=O+O (y=O on one part, then x=O).
Note 3 The answer depended on the path, for this spin field F = S. The answer did
not depend on the choice of parameter. If we follow the same path at a different
speed, the work is the same. We can choose another parameter 2, since (ds/dt)dt and
(ds/dz)dz both equal ds. Traveling twice as fast on the straight path (x = 1 - 22,
y = 22) we finish at t = 4 instead of t = 1. The work is still 1:
CONSERVNION OF TOTAL ENERGY (KINETIC + POTENTIAL)
When a force field does work on a mass m, it normally gives that mass a new velocity.
Newton's Law is F =ma = mdvldt. (It is a vector law. Why write out three components?)
The work F dR is
J (m dvldt) (v dt) = *mv v]: = mv(Q)12 - mlv(P)12.
The work equals the change in the kinetic energy 4mlv12. But for a gradient field the
work is also the change in potential-with a minus sign from physics:
Comparing (8) with (9), the combination m1vl2 + f is the same at P and Q. The total
energy, kinetic plus potential, is conserved.
INDEPENDENCE OF PATH: GRADIENT FIELDS
The work of the spin field S depends on the path. Example 4 took three pathsstraight
line, quarter-circle, bent line. The work was 1, 42, and 0, different on each
path. This happens for more than 99.99% of all vector fields. It does not happen for
the most important fields. Mathematics and physics concentrate on very special
fields-for which the work depends only on the endpoints. We now explain what
happens, when the integral is independent of the path.
Suppose you integrate from P to Q on one path, and back to P on another path.
Combined, that is a closed path from P to P (Figure 15.7). But a backward integral
is the negative of a forward integral, since dR switches sign. If the integrals from P
to Q are equal, the integral around the closed path is zero:
closed path 1 back path 2 path 1 path 2
The circle on the first integral indicates a closed path. Later we will drop the P's.
Not all closed path integrals are zero! For most fields F, different paths yield
different work. For "conservative" fields, all paths yield the same work. Then zero 
15 Vector Calculus
work around a closed path conserves energy. The big question is: How to decide
which fields are conservative, without trying all paths? Here is the crucial information
about conservative fields, in a plane region R with no holes:
15D F = M(x, y)i + N(x, y)j is a conservative field if it has these properties:
A. The work J F dR around every closed path is zero.
B. The work F d R depends only on P and Q, not on the path.
C. F is a gradient field: M = df/ax and N = df/dy for some potential f(x, y).
D. The components satisfy dM/ay = (3Nldx.
A field with one of these properties has them all. D is the quick test.
These statements A-D bring everything together for conservative fields (alias gradient
fields). A closed path goes one way to Q and back the other way to P. The work
cancels, and statements A and B are equivalent. We now connect them to C. Note:
Test D says that the "curl" of F is zero. That can wait for Green's Theorem in the
next section-the full discussion of the curl comes in 15.6.
First, a gradient field F = grad f is conservative. The work is f (Q) - f (P), by the
fundamental theorem for line integrals. It depends only on the endpoints and not the
path. Therefore statement C leads back to B.
Our job is in the other direction, to show that conservative fields Mi + Nj are
gradients. Assume that the work integral depends only on the endpoints. We must
construct a potentialf, so that F is its gradient. In other words, dfldx must be M and
dfldy must be N.
Fix the point P. Define f (Q) as the work to reach Q. Then F equals grad
Check the reasoning. At the starting point P, f is zero. At every other point Q, f is
the work J M dx + N dy to reach that point. Allpathshsfom P to Q give the same f(Q),
because the field is assumed conservative. After two examples we prove that grad f
agrees with F-the construction succeeds.
back path 2 -
Fig. 15.7 Conservative fields:  F dR = 0 and j@ F d R = f (Q) - f (P). Here f (P) = 0.
EXAMPLE 5 Find f (x, y) when F = Mi + Nj = 2xyi + x2j. We want (: f /ax = 2xy
and dfldy = x2.
Solution 1 Choose P = (0,O). Integrate M dx + N dy along to (x, 0) and up to (x, y):
(x. 0) 0, Y)
2xy dx = 0 (since y = 0) x2dY = x2y (which is f ). S (0.0) S (x, 0)
Certainly f = x2y meets the requirements: f, = 2xy and f, = x2. Thus F = gradf Note
that dy = 0 in the first integral (on the x axis). Then dx = 0 in the second integral
(X is fixed). The integrals add to f = x2y. 
15.2 Line Integrals
Solution 2 Integrate 2xydx + x2dy on the straight line (xt, yt) from t = 0 to t = 1:
Iol 2(xt)(yt)(x dt) + (~t)~(y dt)= So13x2yt2dt= x2yt3]: = x2y.
Most authors use Solution 1. I use Solution 2. Most students use Solution 3:
Solution 3 Directly solve df/dx = M = 2xy and then fix up dfldy = N = x2:
af/dx = 2xy gives f = x2y (plus any function of y).
In this example x2y already has the correct derivative dfldy = x2. No additional
function of y is necessary. When we integrate with respect to x, the constant of
integration (usually C ) becomes a function C(y).
You will get practice in finding f. This is only possible for conservative fields! I
tested M = 2xy and N = x2 in advance (using D) to be sure that dM/dy = dN/dx.
EXAMPLE 6 Look for f (x, y) when Mi + Nj is the spin field -yi + xj.
Attempted solution 1 Integrate -y dx + x dy from (0,O)to (x, 0) to (x, y):
I(x, 0)
-ydx=O and I
(x. Y)
x dy = xy (which seems like f).
(0,O) (x. 0)
Attempted solution 2 Integrate -y dx + x dy on the line (xt, yt) from t = 0 to 1 :
So1-(yt)(x dt) + (xt)(y dt) = 0 (a different f, also wrong).
Aitempted solution 3 Directly solve dfldx = -y and try to fix up af/dy = x:
af/dx = -y gives f = -xy (plus any function C(y)).
The y derivative of this f is -x + dC/dy. That does not agree with the required
dfldy = x. Conclusion: The spin field -yi + xj is not conservative. There is no f.
Test D gives dM/dy = -1 and dN/dx = + 1.
To finish this section, we move from examples to a proof. The potential f(Q) is
defined as the work to reach Q. We must show that its partial derivatives are M and
N. This seems reasonable from the formula f (Q) = I M dx + N dy, but we have to
think it through.
Remember statement A, that all paths give the same f(Q). Take a path that goes
from P to the left of Q. It comes in to Q on a line y = constant (so dy = 0). As the
path reaches Q, we are only integrating M dx. The derivative of this integral, at Q, is
df/ax = M. That is the Fundamental Theorem of Calculus.
To show that af/ay= N, take a different path. Go from P to a point below Q. The
path comes up to Q on a vertical line (so dx = 0). Near Q we are only integrating
N dy, so i?f/dy= N.
The requirement that the region must have no holes will be critical for test D.
EXAMPLE 7 Find f (x, y) = x dx + y dy. Test D is passed: aN/ax= 0= dM/dy.
Solution 1 j:",: x dx = +x2is added to j:;:; y dy = fy2.
Solution 2 1; (xt)(x dt) + (yt)(y dt) = 1; (x2+ y2)t dt =f (x2+ y2).
Solution 3 afjax = x gives f = +x2+ C(y).Then af/dy = y needs C(y)= :y2. 
562 15 Vector Calculus
15.2 EXERCISES
Read-through questions
Work is the a of F dR. Here F is the b and R is
the c . The d product finds the component of
in the direction of movement dR = dxi + dyj. The straight
path (x, y) = (t, 2t) goes from f at t = 0 to g at t =
1 withdR=dti+ h .TheworkofF=3i+jisjF=dR=
j i dt= i .
Another form of d R is T ds, where T is the k vector to
the path and ds = ,/T. For the path (t, 2t), the unit vector
Tis m andds= n dt.ForF=3i+j,F*Tdsisstill
0 dt. This F is the gradient off = P . The change in
f= 3x +y from (0,O) to (1,2) is q .
When F = gradf, the dot product F dR is (af/dx)dx + r = df: The work integral from P to Q is j df = s . In this case the work depends on the t but not on the
u . Around a closed path the work is v . The field is
called w . F = (1 + y)i + xj is the gradient off = x .
The work from (0,O) to (1,2) is Y , the change in potential.
For the spin field S = 2 , the work (does)(does not)
depend on the path. The path (x, y) = (3 cos t, 3 sin t) is a
circle with SgdR = A . The work is B around the
complete circle. Formally jg(x, y)ds is the limit of the sum
c.
The four equivalent properties of a conservative field F =
Mi+ Nj are A: D , B: E , C: F , and D: G . Test D is (passed)(not passed) by F = (y + 1)i + xj. The work
IF dR around the circle (cos t, sin t) is H . The work on
the upper semicircle equals the work on I . This field is
the gradient off = J , so the work to (- 1,0) is K .
Compute the line integrals in 1-6.
jcds and jcdy: x = t, y = 2t, 0 6 t < 1.
fcxds and jcxyds: x=cost, y=sint, O<t<n/2.
S, xy ds: bent line from (0,O) to (1, 1) to (1,O).
1, y dx - x dy: any square path, sides of length 3.
fc dx and jc y dx: any closed circle of radius 3.
Jc (dsldt) dt: any path of length 5.
Does if xy dy equal f xy2]:?
Does jfx dx equal fx2]:?
Does (jc d~)~ = (IC d~)~ + (fC dy)l?
Does jc (d~)~ make sense?
11-16 find the work in moving from (1,O) to (0,l). When F
is conservative, construct f: choose your own path when F is
not conservative.
11 F=i+yj 12 F=yi+j
17 For which powers n is S/rn a gradient by test D?
18 For which powers n is R/rn a gradient by test D?
19 A wire hoop around a vertical circle x2 + z2 = a2 has
density p = a + z. Find its mass M = pds.
20 A wire of constant density p lies on the semicircle
x2 + Y2 = a2, y 3 0. Find its mass M and also its moment
Mx = 1 py ds. Where is its center of mass 2 = My/M, j = Mx/
M?
21 If the density around the circle x2 + y2 = a2 is p = x2, what
is the mass and where is the center of mass?
22 Find F dR along the space curve x = t, y = t2, z = t3,
O<t<l.
(a) F = grad (xy + xz) (b) F = yi - xj + zk
23 (a) Find the unit tangent vector T and the speed dsldt
along the path R = 2t i + t2 j.
(b) For F = 3xi + 4j, find F T ds using (a) and F dR
directly.
(c) What is the work from (2, 1) to (4,4)?
24 If M(x, y, z)i + N(x, y, z)j is the gradient of f(x, y, z), show
that none of these functions can depend on z.
25 Find all gradient fields of the form M(y)i + N(x)j.
26 Compute the work W(x, y) = j M dx + N dy on the
straight line path (xt, yt) from t = 0 to t = 1. Test to see if aW/
ax = M and aWpy = N.
(a) M = y3, N = 3xy2 (b) M = x3, N = 3yx2
(c)M=x/y,N=y/x (d)M=ex+Y,N=e"+Y
27 Find a field F whose work around the unit square (y = 0
then x = 1 then y = 1 then x = 0) equals 4.
28 Find a nonconservative F whose work around the unit
circle x2 + y2 = 1 is zero.
In 29-34 compute 1 F dR along the straight line R = ti + tj
and the parabola R = ti + t2j, from (0,O) to (1,l). When F is a
gradient field, use its potential f (x, y).
29 F=i-2j 30 F = x2j
33 F=yi-xj 34 F = (xi + yj)/(x2 + y2 + 1)
35 For which numbers a and b is F = axyi + (x2 + by)j a
gradient field?
36 Compute j - y dx + x dy from (1,O) to (0,l) on the line
x = 1 - t2, y = t2 and the quarter-circle x = cos 2t, y = sin 2t.
Example 4 found 1 and n/2 with different parameters. 

Apply the test Nx = My to 37-42. Find f when test D is passed. 43 Around the unit circle find 4 ds and  dx and 8 xds.
44 True or false, with reason:
(a) When F = yi the line integral lFedR along a curve
xi + yj grad xy from P to Q equals the usual area-under the curve.
39 $F = \frac{xi+yj}{|xi+yj|}$ 40 $F = \frac{grad xy}{|grad xy|}$
(b) That line integral depends only on P and Q, not on the
curve.
41 F=R+S 42 F =(ax + by)i + (cx + dy)j (c) That line integral around the unit circle equals n. 

%%\printexercises{exercises/13_06_exercises}