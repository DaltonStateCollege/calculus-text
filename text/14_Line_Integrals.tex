\section{Line Integrals}\label{sec:line_integrals}
A \emph{line integral} is an integral along a curve. It can equal an area, but
that is a special case and not typical. Instead of area, here are two important line integrals in physics
and engineering:

\keyidea{idea:work_and_flow}{Work and Flow}
{
$$\text{Work along a curve } = \int_C \vec F \cdot \vec T ds$$
 $$\text{Flow across a curve } = \int_C \vec F \cdot \vec n ds$$
}

In the first integral, $\vec F$ is a force field. In the second integral, $\vec F$ is a flow field. Work is done in the direction of movement, so we integrate $\vec F \cdot \vec T$. Flow is measured through the curve $C$, so we integrate $\vec F \cdot \vec n$. Here $\vec T$ is the unit tangent vector, and $\vec F \cdot \vec T$ is the
force component along the curve. Similarly, $\vec n$ is the unit normal vector, at right angles
with $\vec T$. Then $\vec F \cdot \vec n$ is the component of flow perpendicular to the curve.
We will write those integrals in several forms. This section concentrates on work, and
flow comes later. The flow is also called the \emph{flux}. You recognize $ds$
as the step along the curve, corresponding to $dx$ on the $x$ axis. Where $\int_a^b dx$ gives the length of an interval (it equals $b - a$), $\int_C ds$ is the length of the curve $C$.

EXAMPLE 1 Flight from Atlanta to Los Angeles on a straight line and a semicircle.
According to Delta Airlines, the distance straight west is 2000 miles. Atlanta is at
$(1000,0)$ and Los Angeles is at $(-1000, 0)$, with the origin halfway between. The
semicircle route $C$ has radius $1000$. This is not a great circle route. It is more of a
"flat circle," which goes north past Chicago. No plane could fly it (it probably goes
into space).
The equation for the semicircle is $x^2 + y^2 = 1000^2$. Parametrically this path is $x =
1000 \cos t$, $y = 1000 \sin t$. For a line integral the parametric form is usually easier to work with. The plane leaves
Atlanta at $t = 0$ and reaches L.A. at $t = n$, more than three hours later. On the straight
$2000$-mile path, Delta could almost do it. Around the semicircle $C$, the distance is
$lOOO\pi$ miles and the speed has to be $1000$ miles per hour. Remember that speed is
distance $ds$ divided by time $dt$:
$$ds/dt = \sqrt{(dx/dt)^2 + (dy/dt)^2} = 1000\sqrt{(- \sin t)^2 + (\cos t)^2} = 1000.$$

The tangent vector to $C$ is proportional to $(dx/dt, dy/dt) = (-1000 \sin t, 1000 \cos t)$.
But $\vec T$ is a unit vector, so we divide by $1000$ -- which is the speed.
\example{ex_vectorfield_1}{}
{
Suppose the wind blows due east with force $\vec F = M\vec i$. The components are $M$ and
zero. Assuming that $M$ is constant, compute the dot product $\vec F \cdot \vec T$ and the work done by the wind over the course of the flight.}
{
$$\vec F \cdot \vec T= M\vec i \cdot(-\sin t \vec i+\cos t \vec j)= M(-\sin t)+0(\cos t)= -M \sin t$$ 
}

Work is force times distance moved. In the last example it was negative, because the wind acted against the movement of the plane. Because $M$ was constant, the work could have been found more simply by simply taking the distance of the straight-line path between the endpoints
 (2000 miles) and multiplying by $- M$. However, in general this will not be the case.
Most line integrals depend on the path taken. Those that do not depend on the path are crucially important.

For a emph{gradient field}, we only need to know the starting point $P$ and the end point $Q$.

\keyidea{idea:gradient_path_independence}{}
{
When $F$ is the gradient of a potential function $f (x, y)$, the work $\int \vec F \cdot \vec T ds$ depends only on the endpoints $P$ and $Q$. The work is the change in $f$:
$$\text{if } \vec F = \partial f \partial x \vec i + \partial f/\partial y \vec j \text{ then } int_C \vec F \cdot \vec T ds = f (Q) - f(P)$$.
}

When $\vec F = M \vec i$, its components $M$ and zero are the partial derivatives of $f = Mx$. To
compute the line integral, just evaluate $f$ at the endpoints. Atlanta has $x = 1000$, Los
Angeles has $x = - 1000$, and the potential function $f = Mx$ is like an antiderivative:

$$\text{work} = f (Q) - f (P) = M(- 1000) - M(1000) = - 2000M.$$


Here is an intuitive explanation of the work integral $\int_C \vec F \cdot \vec T ds$. Consider the small movement $\vec T ds$, written as $dx \vec i + dy \vec j$. The work is the dot product
with $\vec F$:

$$\vec F \cdot \vec T ds=\left(\dfrac{\partial f}{\partial x}\vec i + \dfrac{\partial f}{\partial y}\vec j \right)\cdot \left(dx \vec i +dy \vec j \right)=\dfrac{\partial f}{\partial x}dx+\dfrac{\partial f}{\partial y}dy=df.$$

The infinitesimal work is $df$. The total work is $\int df= f(Q) - f (P)$. This is the Fundamental
Theorem for a line integral. Only one warning: when $\vec F$ is not the gradient of any
$f$ the Theorem does not apply. Consider the next example.
EXAMPLE 2 Fly these paths against the non-constant force field $\vec F = My\vec i$. Compute
the work.
There is no force on the straight path where $y = 0$. Along the $x$ axis the wind does
no work. But the semicircle goes up where $y = 1000 \sin t$ and the wind is strong:
$\vec F \cdot \vec T=(My \vec i)\cdot(-\sin t \vec i+\cos t \vec j)= -My \sin t= - 1000M \sin 2t$.
This work is enormous (and unrealistic). But the calculations make an important
point: everything is converted to the parameter $t$. The second point is that $\vec F = My \vec i$
is not a gradient field. First reason: The work was zero on the straight path and 
nonzero on the semicircle. Second reason: No function $f$ satisfies both $df/ dx = My$ and $df /dy =
0$. To see this, notice that the first equation requires $f$ to depend on $y$, whereas the second equation requires $f$ to be constant with respect to $y$. Without a potential we cannot substitute $P$ and $Q$ -- and the work depends
on the path.

THE DEFINITION OF LINE INTEGRALS
We go back to the start, to define $\vec F \cdot \vec T ds$. We can think of $\vec F \cdot \vec T$ as a function $g(x, y)$
along the path, and define its integral as a limit of sums:

\begin{definition}
 $$int_C g(x,y)ds = \lim_{||\Delta s||\to\infty} sum_{i=1}^N g(x_i,y_i)\Delta s_i$$
\end{definition}

The points $(x_i, y_i)$ lie on the curve $C$. The last point $Q$ is $(x_N, y_N)$; the first point $P$ is
$(x_0, y_0)$. The step $\Delta s_i$ is the distance to $(x_i, y_i)$ from the previous point. As the steps
get small (as $\Delta s \to 0$) the straight pieces follow the curve. The
special case $g = 1$ gives the arc length. As long as $g(x, y)$ is piecewise continuous
(jumps allowed) and the path is piecewise smooth (corners allowed), the limit exists
and defines the line integral.
When $g$ is the density of a wire, the line integral is the total mass. When $g$ is $\vec F \cdot \vec T$,
the integral is the work. But nobody does the calculation by the definition. We now
introduce a parameter $t$ - which could be the time, or the arc length $s$, or the distance
$x$ along the base.
The differential $ds$ becomes $(ds/dt)dt$. Everything changes over to $t$:

$$\int g(x,y)ds=\int_{t=a}^{t=b}g(x(t),y(t))\sqrt{(dx/dt)^2+(dy/dt)^2}dt$$

The curve starts when $t = a$, runs through the points $(x(t), y(t))$, and ends when $t = b$.
The square root in the integral is the speed $ds/dt$. In three dimensions the points on
$C$ are $(x(t), y(t), z(t))$ and $(dz/dt)^2$ is in the square root.

EXAMPLE 3 The points on a coil spring are $(x, y, z) = (\cos t, \sin t, t)$. Find the mass
of two complete turns (from $t = 0$ to $t = 44$ if the density is $\rho = 4$.
Solution The key is $(dx/dt)^2 + (dy/dt)^2 + (dz/dt)^2 = \sin^2 t + \cos^2 t + 1 = 2$. Thus
$ds/dt = \sqrt{2}$. To find the mass, integrate the mass per unit length which is $g = \rho = 4$:

$$\text{mass}=\int_0^{4\pi}\rho \dfrac{ds}{dt}dt=\int_0^{4\pi}4\sqrt{2}dt=16\sqrt{2}\pi$$

That is a line integral in three-dimensional space. It shows how to introduce $t$. But
it misses the main point of this section, because it contains no vector field $\vec F$. This
section is about work, not just mass.

DIFFERENT FORMS OF THE WORK INTEGRAL

The work integral $\int F \cdot \vec T ds$ can be written in a better way. The force is $\vec F = M\vec i + N\vec j$.
A small step along the curve is $dx \vec i + dy \vec j$. Work is force times distance, but it is only
the force component along the path that counts. The dot product $F\cdot T ds$ finds that
component automatically. 

I 15C The vector to a point on $C$ is $\vec R = x\vec i + y\vec j$. Then $d\vec R = \vec T ds = dx \vec i + dy \vec j$:

$$\text{work}=\int_C\vec F \cdot d\vec R=\int_C M dx+N dy.$$ 

Along a space curve the work is $\int \vec F \cdot \vec T ds=\int \vec F \cdot d \vec R=\int M dx+N dy +P dz$.
The expression $M dx$ is the product of force in the $x$ direction ($M$) and movement in the $x$ direction ($dx$). This product is zero if
either factor is zero. Ignoring friction, pushing a piano along level ground takes no work.
On the other hand, carrying the piano up the stairs brings in a nonzero $P dz$ term, and the total
work is the piano weight $P$ times the change in $z$.
To connect the new $\int \vec F \cdot d\vec R$ with the old $\int \vec F \cdot \vec T ds$, remember the tangent vector
$\vec T$. It is $\d\vec R/ds$. Therefore $\vec T ds$ is $d\vec R$. The best for computations is $d\vec R$, because the
unit vector $\vec T$ has a division by $ds/dt = \sqrt{(dx/dt)^2 + (dy/dt)^2}$. Later we multiply by this
square root, in converting $ds$ to $(ds/dt)dt$. It makes no sense to compute the square
root, divide by it, and then multiply by it. That is avoided in the improved form
$\int Mdx+Ndy$.
EXAMPLE 4 Vector field $\vec F = - y\vec i + x\vec j$, path from $(1,O)$ to $(0, 1)$: Find the work.
Note 1 This $\vec F$ is the spin field $\vec S$. It goes around the origin, while $\vec R = x\vec i + y\vec j$ goes
outward. Their dot product is $\vec F \cdot \vec R = - yx + xy = 0$. This does not mean that
$\vec F \cdot d\vec R = 0$. The force is perpendicular to R, but not to the \emph{change} in R. The work to
move from $(1, 0)$ to $(0, 1$) is not zero.
Note 2 We have not described the path $C$. That must be done. The spin field is not
a gradient field, and the work along a straight line does not equal the work on a
quarter-circle:

$$\text{Straight line path: } x = 1 - t, y = t$$
$$\text{Quarter-circle path: }x = \cos t, y = \sin t$$.

\underline{Calculation of work:} Change $\vec F \cdot d\vec R = M dx + N dy$ to the parameter $t$:

$$\text{Straight line: }\int_C -y dx + x dy =\int_0^1 - t(- dt) + (1 - t)dt = 1

$$\text{Quarter-circle: }\int_C -y dx + x dy =\int_0^{\pi/2} -\sin t(- \sin t dt) + \cos t(\cos t dt) = \dfrac{\pi}{2}$$

\underline{General method:} The path is given by $x(t)$ and $y(t)$. Substitute those into $M(x, y)$
and $N(x, y)$ -- then $\vec F$ is a function of $t$. Also find $dx/dt$ and $dy/dt$. Integrate
$M dx/dt + N dy/dt$ from the starting time $t$ to the finish.

For practice, take the path down the $x$ axis to the origin $(x = 1 - t, y = 0)$. Then
go up the $y$ axis $(x = 0, y = t - 1)$. The starting time at $(1,0)$ is $t = 0$. The turning time
at the origin is $t = 1$. The finishing time at $(0, 1)$ is $t = 2$. The integral has two parts
because this new path has two parts:
$$\text{Bent path: }\int -ydx+xdy=0+0 (y=0 \text{ on one part, then }x=0).$$
Note 3 The answer depended on the path, for this spin field $\vec F = \vec S$. The answer did
not depend on the choice of parameter. If we follow the same path at a different
speed, the work is the same. We can choose another parameter $\tau$, since $(ds/dt)dt$ and
$(ds/d\tau)d\tau$ both equal $ds$. Traveling twice as fast on the straight path $(x = 1 - 2\tau,
y = 2\tau)$ we finish at $\tau = \frac{1}{2}$ instead of $\tau = 1$. The work is still $1$:

CONSERVATION OF TOTAL ENERGY (KINETIC + POTENTIAL)

When a force field does work on a mass $m$, it normally gives that mass a new velocity.
Newton's Law is $\vec F =m\vec a = d\vec v/dt$. (It is a vector law. Why write out three components?)
The work $\int \vec F \cdot d\vec R$ is
\begin{equation}
\int (m d\vec v/dt)\cdot (\vec v dt) = \frac{1}{2}m\vec v\cdot \vec v \right]_P^Q = \frac{1}{2}m\mid \vec v(Q)\mid ^2 - \frac{1}{2}m\mid \vec v(P)\mid ^2
\label{workeq1}
\end{equation}

The work equals the change in the kinetic energy $\frac{1}{2}m\mid \vec v \mid ^2$. But for a gradient field the
work is also the change in potential - with a minus sign from physics:
\begin{equation}
\text{work}=\int \vec F \cdot d\vec R=-\int df=f(P)-f(Q).
\label{workeq2}
\end{equation}

Comparing \ref{workeq1} with \ref{workeq2}, the combination $\frac{1}{2}m\mid\vec v \mid ^2 + f$ is the same at $P$ and $Q$. The total
energy, kinetic plus potential, is conserved.

INDEPENDENCE OF PATH: GRADIENT FIELDS

The work of the spin field $\vec S$ depends on the path. Example 4 took three pathsstraight
line, quarter-circle, bent line. The work was 1, $\pi/2$, and 0: different on each
path. This happens for more than 99.99\% of all vector fields. It does not happen for
the most important fields. Mathematics and physics concentrate on very special
fields, for which the work depends only on the endpoints. We now explain what
happens when the integral is independent of the path.
Suppose you integrate from $P$ to $Q$ on one path, and back to $P$ on another path.
Combined, that is a closed path from $P$ to $P$ (Figure 15.7). But a backward integral
is the negative of a forward integral, since $d\vec R$ switches sign. If the integrals from $P$
to $Q$ are equal, the integral around the closed path is zero:

\begin{align*}
\oint_P^P \vec F \cdot d\vec R &\int_P^Q\vec F \cdot d\vec R + &\int_Q^P\vec F \cdot d\vec R=&\int_P^Q\vec F \cdot d\vec R - &\int_P^Q\vec F \cdot d\vec R=0. \\
\text{closed}&\text{path 1}&\text{back path 2}&\text{path 1}&\text{path 2}
\end{align*}

The circle on the first integral indicates a closed path. In fact it is not necessary for closed paths to list a particular start and endpoint $P$.
Not all closed path integrals are zero! For most fields $\vec F$, different paths yield
different work. For "conservative" fields, all paths yield the same work. Then zero 
work around a closed path conserves energy. The big question is: How to decide
which fields are conservative, without trying all paths? Here is the crucial information
about conservative fields, in a plane region R with no holes:

\keyidea{idea:conservative_field}{Properties of a Conservative Field}{
$\vec F = M(x, y)\vec i + N(x, y)\vec j$ is a conservative field if it has these properties:
\begin{enumerate}
\item The work $\int \vec F d\vec R$ around every closed path is zero.
\item The work $\int_P^Q \vec F d\vec R$ depends only on $P$ and $Q$, not on the path.
\item $\vec F$ is a gradient field: $M = df/dx$ and $N = df/dy$ for some potential function $f(x, y)$.
\item The components satisfy $\partial M/\partial y = \partial N/ \partial x$.
\end{enumerate}
A field with one of these properties has them all. Property 4 gives a quick test.
}
These four properties bring everything together for conservative fields (also known as \emph{gradient
fields}). A closed path goes one way to $Q$ and back the other way to $P$. The work
cancels, and statements 1 and 2 are equivalent. We now connect them to statement 3. 

First, a gradient field $\vec F = \nabla f$ is conservative. The work is $f(Q) - f(P)$, by the
fundamental theorem for line integrals. It depends only on the endpoints and not the
path. Therefore statement $C$ leads back to $B$.
Our job is in the other direction, to show that conservative fields $M\vec i + N\vec j$ are
gradients. Assume that the work integral depends only on the endpoints. We must
construct a potential $f$, so that $\vec F$ is its gradient. In other words, $df/dx$ must be $M$ and
$df/dy$ must be $N$.

Fix the point $P$. Define $f(Q)$ as the work to reach $Q$. Then $\vec F$ equals $\nabla f$.

Check the reasoning. At the starting point $P$, $f$ is zero. At every other point $Q$, $f$ is
the work $\int M dx + N dy$ to reach that point. All paths from $P$ to $Q$ give the same $f(Q)$,
because the field is assumed conservative. After two examples we prove that $\nabla f$
agrees with $\vec F$ -- the construction succeeds.

EXAMPLE 5 Find $f(x,y)$ when $\vec F = M\vec i + N\vec j = 2xy\vec i + x^2\vec j$. We want $\partial f /\partial x = 2xy$
and $\partial f/\partial y = x^2$.
Solution 1 Choose $P = (0,0)$. Integrate $M dx + N dy$ along to $(x, 0)$ and up to $(x, y)$:
$$\int_{(0,0)}^{(x,0)}2xydx=0 \text{(since y = 0)   } \int_{(x,0)}^{(x,y)}x^2dy = x^2y \text{ (which is } f \text{ ).}$$ 

Certainly $f = x^2y$ meets the requirements: $f_x= 2xy$ and $f_y = x^2$. Thus $\vec F = \nabla f$. Note
that $dy = 0$ in the first integral (on the $x$ axis). Then $dx = 0$ in the second integral
($x$ is fixed). The integrals add to $f = x^2y$. 

Solution 2 Integrate $2xydx + x^2dy$ on the straight line $\left(xt, yt\right)$ from $t = 0$ to $t = 1$:

$$\int_0^1 2(xt)(yt)(x dt) + (xt)^2(y dt)= \int_0^1 3x^2yt^2dt = x^2yt^3\right]_0^1: = x^2y.$$

Most authors use Solution 1. I use Solution 2. Most students use Solution 3:

Solution 3 Directly solve $\partial f/\partial x = M = 2xy$ and then fix up $\partial f/\partial y = N = x^2$:
$$\partial f/\partial x = 2xy \text{ gives }f = x^2y \text{ (plus any function of }y).$$

In this example $x^2y$ already has the correct derivative $\partial f/\partial y = x^2$. No additional
function of $y$ is necessary. When we integrate with respect to $x$, the constant of
integration (usually $C$ ) becomes a function $C(y)$.

You will get practice in finding $f$. This is only possible for conservative fields!

EXAMPLE 6 Look for $f(x,y)$ when $M\vec i + N\vec j$ is the spin field $-y\vec i + x\vec j$.
Attempted solution 1 Integrate $-y dx + x dy$ from $(0,0)$ to $(x, 0)$ to $(x, y)$:
$$\int_{(0,0)}^{(x,0)}-ydx=0\text{ and }$$\int_{(x,0)}^{(x,y)}x dy = xy \text{ (which seems like }f).$$

Attempted solution 2 Integrate $-y dx + x dy$ on the line $(xt, yt)$ from $t = 0$ to $1$ :

$$\int_0^1-(yt)(x dt) + (xt)(y dt) = 0 \text{ (a different }f \text{, also wrong).}$$

Attempted solution 3 Directly solve $\partial f/\partial x = -y$ and try to fix up $\partial f/\partial y = x$:
$\partial f/\partial x = -y$ gives $f = -xy$ (plus any function $C(y)$).
The $y$ derivative of this $f$ is $-x + dC/dy$. That does not agree with the required
$\partial f/\partial y = x$. Conclusion: The spin field $-y\vec i + x\vec j$ is not conservative. There is no $f$.

Finally, note that $\partial M/\partial y = -1$ and $\partial N/\partial x =  1$. Since these partial derivatives are not the same, we have another verification that $f$ is not conservative.

To finish this section, we move from examples to a proof. The potential $f(Q)$ is
defined as the work to reach $Q$. We must show that its partial derivatives are $M$ and
$N$. This seems reasonable from the formula $f (Q) = \int M dx + N dy$, but we have to
think it through.

Remember statement A, that all paths give the same $f(Q)$. Take a path that goes
from $P$ to the left of $Q$. It comes in to $Q$ on a horizontal line (so $dy = 0$). As the
path reaches $Q$, we are only integrating $M dx$. The derivative of this integral, at $Q$, is
$\partial f/\partial x = M$. That is the Fundamental Theorem of Calculus.
To show that $\partial f/\partial y= N$, take a different path. Go from $P$ to a point below $Q$. The
path comes up to $Q$ on a vertical line (so $dx = 0$). Near $Q$ we are only integrating
$N dy$, so $\partial f/\partial y= N$.

The requirement that the region must have no holes will be critical for test D.

EXAMPLE 7 Find $f (x, y) =\int_{(0,0)}^{(x,y)} x dx + y dy$. Test D is passed: $\partial N/\partial x= 0= \partial M/\partial y$.
Solution 1 $\int_{(0,0)}^{(x,0)} x dx = \frac{1}{2}x^2$ is added to $\int_{(x,0)}^{(x,y)} y dy = \frac{1}{2}y^2$.
Solution 2 $\int_0^1 (xt)(x dt) + (yt)(y dt) = \int_0^1 (x^2+ y^2)t dt =\frac{1}{2}(x^2+ y^2)$.
Solution 3 $\partial f/\partial x = x$ gives $f = \frac{1}{2}x^2+ C(y)$. Then $\partial f/\partial y = y$ needs $C(y)= \frac{1}{2}y^2$. 

%15.2 EXERCISES
%Read-through questions
%Work is the a of F dR. Here F is the b and R is
%the c . The d product finds the component of
%in the direction of movement dR = dxi + dyj. The straight
%path (x, y) = (t, 2t) goes from f at t = 0 to g at t =
%1 withdR=dti+ h .TheworkofF=3i+jisjF=dR=
%j i dt= i .
%Another form of d R is T ds, where T is the k vector to
%the path and ds = ,/T. For the path (t, 2t), the unit vector
%Tis m andds= n dt.ForF=3i+j,F*Tdsisstill
%0 dt. This F is the gradient off = P . The change in
%f= 3x +y from (0,O) to (1,2) is q .
%When F = gradf, the dot product F dR is (af/dx)dx + r = df: The work integral from P to Q is j df = s . In this case the work depends on the t but not on the
%u . Around a closed path the work is v . The field is
%called w . F = (1 + y)i + xj is the gradient off = x .
%The work from (0,O) to (1,2) is Y , the change in potential.
%For the spin field S = 2 , the work (does)(does not)
%depend on the path. The path (x, y) = (3 cos t, 3 sin t) is a
%circle with SgdR = A . The work is B around the
%complete circle. Formally jg(x, y)ds is the limit of the sum
%c.
%The four equivalent properties of a conservative field F =
%Mi+ Nj are A: D , B: E , C: F , and D: G . Test D is (passed)(not passed) by F = (y + 1)i + xj. The work
%IF dR around the circle (cos t, sin t) is H . The work on
%the upper semicircle equals the work on I . This field is
%the gradient off = J , so the work to (- 1,0) is K .
%Compute the line integrals in 1-6.
%jcds and jcdy: x = t, y = 2t, 0 6 t < 1.
%fcxds and jcxyds: x=cost, y=sint, O<t<n/2.
%S, xy ds: bent line from (0,O) to (1, 1) to (1,O).
%1, y dx - x dy: any square path, sides of length 3.
%fc dx and jc y dx: any closed circle of radius 3.
%Jc (dsldt) dt: any path of length 5.
%Does if xy dy equal f xy2]:?
%Does jfx dx equal fx2]:?
%Does (jc d~)~ = (IC d~)~ + (fC dy)l?
%Does jc (d~)~ make sense?
%11-16 find the work in moving from (1,O) to (0,l). When F
%is conservative, construct f: choose your own path when F is
%not conservative.
%11 F=i+yj 12 F=yi+j
%17 For which powers n is S/rn a gradient by test D?
%18 For which powers n is R/rn a gradient by test D?
%19 A wire hoop around a vertical circle x2 + z2 = a2 has
%density p = a + z. Find its mass M = pds.
%20 A wire of constant density p lies on the semicircle
%x2 + Y2 = a2, y 3 0. Find its mass M and also its moment
%Mx = 1 py ds. Where is its center of mass 2 = My/M, j = Mx/
%M?
%21 If the density around the circle x2 + y2 = a2 is p = x2, what
%is the mass and where is the center of mass?
%22 Find F dR along the space curve x = t, y = t2, z = t3,
%O<t<l.
%(a) F = grad (xy + xz) (b) F = yi - xj + zk
%23 (a) Find the unit tangent vector T and the speed dsldt
%along the path R = 2t i + t2 j.
%(b) For F = 3xi + 4j, find F T ds using (a) and F dR
%directly.
%(c) What is the work from (2, 1) to (4,4)?
%24 If M(x, y, z)i + N(x, y, z)j is the gradient of f(x, y, z), show
%that none of these functions can depend on z.
%25 Find all gradient fields of the form M(y)i + N(x)j.
%26 Compute the work W(x, y) = j M dx + N dy on the
%straight line path (xt, yt) from t = 0 to t = 1. Test to see if aW/
%ax = M and aWpy = N.
%(a) M = y3, N = 3xy2 (b) M = x3, N = 3yx2
%(c)M=x/y,N=y/x (d)M=ex+Y,N=e"+Y
%27 Find a field F whose work around the unit square (y = 0
%then x = 1 then y = 1 then x = 0) equals 4.
%28 Find a nonconservative F whose work around the unit
%circle x2 + y2 = 1 is zero.
%In 29-34 compute 1 F dR along the straight line R = ti + tj
%and the parabola R = ti + t2j, from (0,O) to (1,l). When F is a
%gradient field, use its potential f (x, y).
%29 F=i-2j 30 F = x2j
%33 F=yi-xj 34 F = (xi + yj)/(x2 + y2 + 1)
%35 For which numbers a and b is F = axyi + (x2 + by)j a
%gradient field?
%36 Compute j - y dx + x dy from (1,O) to (0,l) on the line
%x = 1 - t2, y = t2 and the quarter-circle x = cos 2t, y = sin 2t.
%Example 4 found 1 and n/2 with different parameters. 
%
%Apply the test Nx = My to 37-42. Find f when test D is passed. 43 Around the unit circle find 4 ds and  dx and 8 xds.
%44 True or false, with reason:
%(a) When F = yi the line integral lFedR along a curve
%xi + yj grad xy from P to Q equals the usual area-under the curve.
%39 $F = \frac{xi+yj}{|xi+yj|}$ 40 $F = \frac{grad xy}{|grad xy|}$
%(b) That line integral depends only on P and Q, not on the
%curve.
%41 F=R+S 42 F =(ax + by)i + (cx + dy)j (c) That line integral around the unit circle equals n. 

%%\printexercises{exercises/13_06_exercises}