\section{Line Integrals}\label{sec:line_integrals}

Much as the Riemann integral was defined as integral on an interval of the real number line or on a region of the plane,  a \emph{line integral} is an integral along a curve. It can equal an area, but
that is a special case and not typical. Instead of area, two motivating examples of line integrals from physics and engineering are work along a two-dimensional or three dimensional curve and fluid flow across a curve.\\

\keyidea{idea:work_and_flow}{Work and Flow}
{
$$\text{Work along a curve } = \int_C \vec F \cdot \vec T \: ds$$
 $$\text{Flow across a curve } = \int_C \vec F \cdot \vec n \: ds$$
}\\

In the first integral, $\vec F$ is a force field. In the second integral, $\vec F$ is a flow field. Work is done in the direction of movement, so we integrate $\vec F \cdot \vec T$, where $\vec T$ is a unit vector tangent to the curve. Flow is measured through the curve $C$, so we integrate $\vec F \cdot \vec n$, where $\vec n$ is the unit normal vector, a vector perpendicular to $\vec T$. Then $\vec F \cdot \vec n$ is the component of flow perpendicular to the curve.
We will write those integrals in several forms. This section concentrates on work, and
flow comes later. The flow is also called the \emph{flux}. The differential $ds$
represents the step along the curve, corresponding to $dx$ on the $x$-axis for a one-dimensional integral. Whereas $\int_a^b dx$ gives the length of an interval (it equals $b - a$), the value of $\int_C ds$ is the length of the curve $C$.\\

\example{ex_line_integrals0}{}{Consider two possible flight paths from Atlanta to Los Angeles, with Atlanta at $(1000,0)$ and Los Angeles at $(-1000,0)$ on what we will assume is a flat plane.  One flight path is a straight line while the other is a semicircle from Atlanta to $(0,1000)$ to Los Angeles. Suppose the wind blows due east with force $\vec F = M\vec i$, with the component in the direction $\vec j$ being zero. Assuming that $M$ is a constant, compute the dot product $\vec F \cdot \vec T$ and then the work done by the wind over the course of the flight.}
{First, along the straight line path, the work done will be $-2000M$, given that work is distance multiplied by force, and the force is a constant in the direction opposite the path of travel. Next we consider the semicircular path given by $x^2 + y^2 = 1000^2$.  Parametrically we can write this path as
$$x = 1000 \cos(t), \: y = 1000 \sin(t)$$
for $t = 0$ to $t = \pi$ being the values of the parameter $t$ from Atlanta to Los Angeles, respectively. The plane's speed will be
$$\dfrac{ds}{dt} = \sqrt{\left( \dfrac{dx}{dt} \right)^2 + \left( \dfrac{dy}{dt} \right)^2} = 1000 \sqrt{\cos(t)^2 + \sin(t)^2} = 1000.$$ Note that the unit tangent vector $\vec T$ will be 
$$\vec T = \dfrac{1}{1000}\left( \dfrac{d}{dt}(1000\cos(t))\vec i + \dfrac{d}{dt}(1000\sin(t)) \vec j \right)= -\sin(t) \vec i + \cos(t) \vec j$$
and so
$$\vec F \cdot \vec T= M\vec i \cdot (-\sin t \vec i+\cos t \vec j)= M(-\sin t)+0(\cos t)= -M \sin t$$
Now to calculate the work done from $t = 0$ to $t = \pi$, we compute
$$\int_{C} \vec F \cdot \vec T \: ds = \int_{0}^{\pi} \left( -M \sin(t) \right) \left( \dfrac{ds}{dt} \: dt \right) = -1000M \: \sin(t) \: dt = -2000M$$
which is the same as the straight-line path.  It appears that the path of the flight doesn't matter in this example, only the origin and destination.
}\\

Work is force times distance moved. In this last example it was negative, because the wind acted against the movement of the plane. Because $M$ was constant, it appears the work along the semicircular path could have been found more simply by simply taking the distance of the straight-line path between the endpoints and multiplying by $-M$. However, in general this will not be the case. Most line integrals depend on the path taken. Those that do not depend on the path are crucially important. For a emph{gradient field}, we only need to know the starting point $P$ and the end point $Q$.\\

\keyidea{idea:gradient_path_independence}{}
{
When $\vec F$ is the gradient of a potential function $f(x, y)$, the work $\int \vec F \cdot \vec T ds$ depends only on the endpoints $P = (x_0,y_0)$ and $Q = (x_1,y_1)$. The work is the change in $f$: if
$$\vec F = \partial f \partial x \vec i + \partial f/\partial y \vec j$$
then
$$int_C \vec F \cdot \vec T ds = f(Q) - f(P).$$
}\\

In the above example, the vector field $\vec F$ was the gradient of the potential function $f(x,y) = Mx$, and so the work done would be just $f(-1000,0) - f(1000,0) = -2000M$, as we calculated.  We move on next to a more formal definition of line integrals. \\




\noindent\textbf{\large The Definition of Line Integrals}\\

To define $\vec F \cdot \vec T ds$. We can think of $\vec F \cdot \vec T$ as a function $f(x, y)$
along the path $C$, and define its integral as a limit of sums as we are used to. Suppose we have a smooth plane curve $C$ given parametrically by
$$x = x(t), \: y = y(t), \text{ for } a \leq t \leq b$$
and defining a vector-valued function $\vec r (t) = x(t) \vec i + y(t) \vec j$.  We divide $C$ into $n$ sub-arcs by partitioning $[a,b]$ into $n$ subintervals $[t_{i-1},t]$. Taking $x_i = x(t_i)$ and $y_i = y(t_i)$, we have divided $C$ into sub-arcs from $(x_{i-1},y_{i-1})$ to $(x_i,y_i)$ with corresponding arclengths $\Delta s_i$.  Choosing any point $t_i^{\ast}$ in $[t_{i-1},t]$ gives a point $(x_i^{\ast},y_i^{\ast}) = (x(t_i^{\ast}), y(t_i^{\ast})$ in the $i^{\text{th}}$ sub-arc.\\

Now suppose $f(x,y)$ is a function of two variables including the curve $C$ in its domain.  We can then form
$$\sum_{i=1}^{n} f(x_i^{\ast},y_i^{\ast}) \Delta s_i$$
which is a Riemann Sum.  Taking the limit we arrive at the definition of the line integral of $f(x,y)$ along the curve $C$.


\definition{def:lineintegral}{Line Integral of $f(x,y)$ along $C$}{Let $C$ be a smooth curve in the domain of the function $f(x,y)$. Then the line integral of $f(x,y)$ along $C$ is given by
$$\int_C g(x,y) \: ds = \lim_{||\Delta s|| \to 0} \sum_{i=1}^n f(x_i,y_i)\Delta s_i$$}\\


The points $(x_i, y_i)$ lie on the curve $C$. The last point $Q$ is $(x_n, y_n)$ and the first point $P$ is $(x_0, y_0)$. The step $\Delta s_i$ is the distance to $(x_i, y_i)$ from the previous point. As $\Delta s \to 0$, the straight pieces better and better approximate the path of the curve. Note that the special case $f(x,y) = 1$ gives the arc length of $C$. As long as $f(x, y)$ is piecewise continuous (a finite number of jumps allowed) and the path is piecewise smooth (a finite number of corners allowed), the limit exists and defines the line integral.\\

When $f(x,y)$ is the density of a wire $C$, the line integral is the total mass of the wire. When $f(x,y)$ is $\vec F \cdot \vec T$, the integral is the work done along the path $C$. Calculating these values using the above definition is cumbersome. We now parametrize the curve $C$ as above.  The parameter could be thought of as the time for a moving object.  In doing so, the differential $ds$ becomes $\frac{ds}{dt} \: dt$ and everything in the integral can be changed over the new variable $t$, where $t = a$ gives the starting point $P = (x(a),y(a))$ and $t = b$ gives the ending point $Q = (x(b),y(b))$:
$$\int_C f(x,y) \: ds = \int_{t=a}^{t=b} f(x(t),y(t)) \: \sqrt{(\dfrac{dx}{dt})^2+(\dfrac{dy}{dt})^2} \: dt$$\\

We can repeat this process in three dimensions as well. In three dimensions the points on a smooth three-dimensional curve $C$ are given by a parametrization $(x(t), y(t), z(t))$, from $t = a$ to $t = b$, and so given a piecewise continuous function $f(x,y,z)$ we obtain
$$\int_C f(x,y,z) \: ds = \int_{t=a}^{t=b} f(x(t),y(t),z(t)) \: \sqrt{(\dfrac{dx}{dt})^2+(\dfrac{dy}{dt})^2 + \dfrac{dx}{dt})^2} \: dt$$\\


\example{ex_line_integrals1}{}{Suppose the points on a coil spring $C$ are $$(x(t), y(t), z(t)) = (\cos t, \sin t, t)$$ for $t = 0$ to $t = 4\pi$. Find the mass of the spring if the density is given by $\rho = 4$ kilograms per meter.}
{We first calculate 
$$(dx/dt)^2 + (dy/dt)^2 + (dz/dt)^2 = \sin^2 t + \cos^2 t + 1 = 2.$$ Thus
$\dfrac{ds}{dt} = \sqrt{2}$. To find the total mass, integrate the mass per unit length, which is $g = \rho = 4$ from $t = 0$ to $t = 4\pi$.
$$\text{mass}=\int_0^{4\pi} \rho \dfrac{ds}{dt} \: dt = \int_0^{4\pi}4\sqrt{2}\: dt = 16\sqrt{2}\pi$$
kilograms.}\\

\example{ex_line_integrals2}{}{Compute the line integral $\displaystyle\int_C f(x,y) \: ds$ where $f(x,y) = 3x$ and $C$ is the portion of the parabola $y = x^2$ from $(0,0)$ to $(2,4)$.}{First we need to parametrize the curve $C$.  This can be done in any case of a curve $y = g(x)$ by simply letting $t = x$.  In this case we obtain $x = x(t) = t$ and $y = y(t) = 3t$, for $t = 0$ to $t = 2$. Note that $f(x,y)$ becomes $f(t) = 3t$. Then
$$\left(\dfrac{dx}{dt}\right)^2 + \left(\dfrac{dy}{dt}\right)^2 = (1)^2 + (3)^2 = 10$$ and so
$$\int_C f(x,y) \: ds = \int_0^2 3t \: \sqrt{10} \: dt = \dfrac{3\sqrt{10}}{2}\left( 4 - 0 \right) = 6 \sqrt{10}.$$
}\\




\noindent\textbf{\large Different Forms of the Work Integral}\\

The work integral $\int F \cdot \vec T ds$ from before can be written in a different way. Suppose the vector field is $\vec F = M(x,y) \vec i + N(x,y) \vec j$.
A small step along the curve is $ds = dx \vec i + dy \vec j$. Work is force multiplied by distance, but it is only the force component along the direction of the path that counts. The dot product $F\cdot T ds$ finds that component automatically.  The vector to a point on $C$ is $\vec r = x\vec i + y\vec j$. Then $d\vec r = \vec T ds = dx \vec i + dy \vec j$.  So we arrive at
$$\text{work} = \int_C \vec F \cdot d\vec r = \int_C M(x,y) dx + N(x,y) dy.$$
If the curve $C$ is in three-dimensions, the work is similarly
$$\int_C \vec F \cdot \vec T \: ds = \int_C \vec F \cdot d \vec r = \int M(x,y,z) dx+N(x,y,z) dy +P(x,y,z) dz.$$\\

Consider the expression $M \: dx$, which is the product of force in the $x$-direction $M$ and  the movement in the $x$-direction $dx$. This product is zero if either factor is zero. Ignoring friction, pushing a piano along level ground takes no work. On the other hand, carrying the piano up the stairs brings in a nonzero $P \: dz$ term, and the total work is the piano weight $P$ times the change in $z$.\\

To connect the new $\int \vec F \cdot d\vec r$ line integral with the old $\int \vec F \cdot \vec T ds$ expression, consider the tangent vector $\vec T$. It is $\dfrac{d \vec r}{ds}$. Therefore $\vec T ds$ is simply $d\vec r$. The best form for computations is $\vec r$, because the
unit vector $\vec T$ has a division by $ds/dt = \sqrt{(dx/dt)^2 + (dy/dt)^2}$, which can get messy. Later we multiply by this square root, in converting $ds$ to $(ds/dt)dt$. It makes no sense to compute the square root, divide by it, and then multiply by it. That is avoided in the new form
$\int_C M \: dx + N \: dy$.\\

\example{ex_line_integrals3}{}{Consider the vector field $\vec F(x,y) = - y\vec i + x\vec j$ and two different paths from $(1,0)$ to $(0,1)$.  The first path will be a straight line segment path and the second is a quarter-circle path along the unit circle.  Compute the work done by $\vec F$ in moving along each path.}{ First we consider the straight-line path.  This requires parametrizing a line segment from a point $P$ to a point $Q$ which can always be done using
$$\vec r(t) = (1-t)P + tQ$$
for $t = 0$ to $t = 1$.  In this example, we would have
$$\vec r(t) = (1-t)(1,0) + t(0,1) = (1-t,t)$$
or $x(t) = 1-t$ and $y(t) = t$ for $0 \leq t \leq 1$.  Therefore $dx = - dt$ and $dy = dt$.  Computing the work yields
$$\int_C -y \: dx + x \: dy = \int_0^1 -t(-dt) + (1-t)(dt) = \int_0^1  1 \: dt = 1.$$
Next we compute the work done along the quarter-circle path.  Since this follows the unit circle, it can be parametrized by 
$$\vec r(t) = \cos(t) \vec i + \sin(t) \vec j$$
for $t = 0$ to $t = \dfrac{\pi}{2}$.  Then
$$dx = -\sin(t) \: dt \: \text{ and } dy = \cos(t) \: dt$$
giving the work done as
$$\int_C -y \: dx + x \: dy = \int_0^{\pi/2} -sin(t)(-\sin(t) \: dt) + \cos(t)( \cos(t) \: dt) = \int_0^{\pi/2} 1 \: dt = \dfrac{\pi}{2}.$$
Note here that the value of the work done does depend on the path taken, not just the starting and ending points.
}\\

\keyidea{idea:lineintegral_work}{Computing work done by a force field $\vec F$ along a curve $C$}{Suppose a smooth curve $C$ is parametrized by $$\vec r(t) = x(t) \vec i + y(t) \vec j$$ and an object is traveling along the curve $C$ from a point at $t = a$ to $t = b$ subject to a force $\vec F$.  To determine the work done by $\vec F$ on the object, one computes $\int_C M \: dx + N \: dy$ by first substituting $x(t)$ and $y(t)$ into $M(x,y)$ and $N(x,y)$ to create a function of the parameter $t$.  Find $dx$ and $dy$ by taking the derivatives of $x(t)$ and $y(t)$ to turn the integral into an integral of $t$.  Then integrate from $t = a$ to $t = b$ to compute the work done.  The procedure is similar if $C$ is in three-dimensional space and the object subject to the force $\vec F(x,y,z)$.
}\\

\example{ex_line_integrals4}{}{Consider a force field $\vec F(x,y) = x^2 \vec i - xy \vec j$ and a particle moving along the quarter-circle path from $(1,0)$ to $(0,1)$ along the unit circle. Determine the work done by the force field on the particle.}{We will again use the parametrization
$$\vec r(t) = \cos(t) \vec i + \sin(t) \vec j$$
for $t = 0$ to $t = \dfrac{\pi}{2}$, giving $dx = -\sin(t) \: dt$ and $dy = \sin(t) \: dt$.  Then the force field becomes 
$$\vec F(t) = \cos^2(t) \vec i - \cos(t)\sin(t) \vec j$$
as a function of the parameter $t$.  Computing the work done yields
$$\int_C M \: dx + N \: dy = \int_0^{\pi/2} \cos^2(t)\left( -\sin(t) \: dt\right) + \cos(t)\sin(t)\left( \cos(t) \: dt\right) = \int_0^{\pi/2} -2\cos^2(t)\sin(t) \: dt$$
Using a substitution of $u = \cos(t)$ and $du = -\sin(t) \: dt$ gives the work done by $\vec F$ is
$$2\left( \dfrac{\cos^3(t)}{3} \right]_{0}^{\pi/2} = -\dfrac{2}{3}.$$
Note that the work is negative because the particle is moving against the force along its path.
}\\


\noindent\textbf{\large Independence of Path and Conservation of Energy}\\

When a force field does work on a mass $m$, it normally gives that mass a new velocity.
Newton's Law is $\vec F = m \vec a = m \dfrac{d \vec v}{dt}$. The work done $\int_C \vec F \cdot d\vec r$ along a path $C$ from $P$ to $Q$ is
\begin{equation} \int (m \dfrac{d \vec v}{dt}) \cdot (\vec v dt) 
  = \left( \dfrac{1}{2}m \vec v \cdot \vec v \right]_P^Q
  = \dfrac{1}{2}m \mid \vec v(Q) \mid^2 - \dfrac{1}{2}m \mid \vec v(P)\mid^2
\label{workeq1}	
\end{equation}\\

The work equals the change in the kinetic energy $\frac{1}{2}m\mid \vec v \mid^2$. However, for a gradient field $\vec F = \nabla f$ the
work is also the negative change in potential
\begin{equation}
\text{work}=\int \vec F \cdot d\vec r = -\int df = f(P)-f(Q).
\label{workeq2}
\end{equation}\\

Comparing \ref{workeq1} with \ref{workeq2}, the combination $\frac{1}{2}m\mid\vec v \mid ^2 + f$ is the same at $P$ and $Q$. The total energy, kinetic plus potential, is \emph{conserved}.\\


Most of the examples in physics concentrate on special fields in which the work done depends only on the starting and ending points of the path. We now explain what happens when the integral is independent of the path. Suppose you integrate from $P$ to $Q$ on one path, and then back to $P$ on another path. Combined, that is a closed path from containing $P$. See Figure \ref{}. But a backward integral is the negative of a forward integral, since $d\vec r$ switches sign. If the two integrals from $P$ to $Q$ are equal, the integral around the closed path is zero.
\begin{align*}
\oint_C \vec F \cdot d\vec r & = \int_P^Q \vec F \cdot d\vec r + \int_Q^P \vec F \cdot d\vec r \\
 & = \int_P^Q\vec F \cdot d\vec r - \int_P^Q\vec F \cdot d\vec r = 0 \\
\end{align*}\\

Notationally, the circle on the first integral indicates a closed path. It is not necessary for closed paths to indicate a particular start and endpoint $P$. However, not all closed path integrals are zero! For most fields $\vec F$, different paths yield
different work. For what are called \emph{conservative} fields, all paths with the same starting and ending points yield the same work. Then zero work around a closed path indicates conservation of energy. The crucial question is how to determine which fields are conservative without trying an infinite number of paths.\\

\keyidea{idea:conservative_field}{Properties of a Conservative Field}{The vector field
$\vec F = M(x, y)\vec i + N(x, y)\vec j$ is a conservative field if it has these properties:
\begin{enumerate}
\item The work $\oint_C \vec F \cdot d\vec r$ around every closed path $C$ is zero;
\item The work $\int_C \vec F \cdot  d\vec r$ along a curve $C$ depends only on the starting and ending points of the path, not on actual path itself;
\item $\vec F$ is a gradient field.  That is, $M = \dfrac{\partial f}{\partial x}$ and $N = \dfrac{\partial f}{\partial y}$ for some potential function $f(x, y)$.
\end{enumerate}
A field with one of these properties has them all.
}\\

Consider the case of a smooth path $C$ given by $\vec r(t) = x(t) \vec i + y(t) \vec j$ from $t = a$ to $t = b$. If a particle moving along that path is subject to a conservative vector field $\vec F = \nabla f$, then path independence above can be restated as follows, sometimes called the \emph{Fundamental Theorem of Line Integrals}, which is analogous to the second part of the Fundamental Theorem of Calculus for regular integrals.\\

\theorem{thm:lineint_fundthm}{Fundadmental Theorem for Line Integrals}
{Given a smooth curve $C$ from $P$ to $Q$ defined parametrically by $\vec r(t) = x(t) \vec i + y(t) \vec j$ for $a \leq t \leq b$ and a \textbf{conservative} vector field $\vec F(x,y) = \nabla f(x,y)$, then
$$\int_C \vec F \cdot d\vec r = f\left( \vec r(b) \right) - f\left( \vec r(a) \right) = f(Q) - f(P)$$
}

The proof of the above result follows simply from the fact that $\vec F = \nabla f$ and that 
$$\nabla f(x,y) \cdot d\vec r(t) = \dfrac{\partial f}{\partial x} \dfrac{dx}{dt} + \dfrac{\partial f}{\partial y} \dfrac{dy}{dt} = \dfrac{d}{dt} f\left( \vec r(t) \right)$$
by the Chain Rule.  A similar statement for the Fundamental Theorem holds in three dimensions, as well.\\

Next we focus on a simple test for determining whether a vector field in two dimensions is conservative.  If $\vec F(x,y) = M(x,y) \vec i + N(x,y) \vec j$ is conservative and satisfies $\vec F = \nabla f(x,y)$ then
$$\dfrac{\partial f}{\partial x} = M(x,y) \: \text{ and } \dfrac{\partial f}{\partial y} = N(x,y).$$
We know from Chapter 12 that mixed partials are equal under the right circumstances.  That is, if $M(x,y)$ and $N(x,y)$ have continuous  first-order partial derivatives, then
$$\dfrac{\partial^2 f}{\partial y \partial x} = \dfrac{\partial^2 f}{\partial x \partial y}$$
or
$$\dfrac{\partial M}{\partial y} = \dfrac{\partial N}{\partial x}.$$\\

\theorem{thm:lineint_conserv}{Two Dimensional Test for a Conservative Vector Field}
{Let $\vec F(x,y) = M(x,y) \vec i + N(x,y) \vec j$ be a vector field defined on an open simply-connected region $D$.  If $M(x,y)$ and $N(x,y)$ have continuous first-order partial derivatives and $\dfrac{\partial M}{\partial y} = \dfrac{\partial N}{\partial x}$ then $\vec F$ is conservative.
}\\

The property of region $D$ being open and simply-connected means, essentially, that there are no holes in the region on which the vector field is defined. In such situations, the above test will determine whether or not a given vector field is conservative.  In the case that it is, one may be able to employ anti-derivatives to find the potential function $f(x,y)$ so that $\vec F = \nabla f$.  It is necessary to know this potential function if one is to use the Fundamental Theorem to compute line integrals of conservative vector fields.\\


\example{ex_line_integrals5}{}{Show that $\vec F(x,y) = 2xy\vec i + x^2\vec j$ is conservative.  Then find a potential function $f(x,y)$ so that $\nabla f = \vec F$.} {For $M(x,y) = 2xy$ and $N(x,y) = x^2$, note that $\dfrac{\partial M}{\partial y} = 2x$ and $\dfrac{\partial N}{\partial x} = 2x$, also.  Since $M$ and $N$ are defined everywhere and have continuous partial derivatives of every type, it follows that $\vec F$ is conservative.  Now we find the potential.
Start with $M(x,y) = 2xy$.  We need $f(x,y)$ so that $\dfrac{\partial f}{\partial x} = 2xy$. This would occur if
$$f(x,y) = \int 2xy \: dx + g(y) = x^2 y + g(y)$$
for some function $g$ depending only on $y$.   Take the partial derivative with respect to $y$ now.  We should have that $\dfrac{\partial f}{\partial y}$ is $N(x,y) = x^2$, but we also get from above that
$$\dfrac{\partial f}{\partial y} = x^2 + g'(y).$$
Setting these equal, this implies that $g'(y) = 0$ or the $g(y)$ is a constant.  Therefore $f(x,y) = x^2 y + C$ for any constant $C$ is a potential function for $\vec F$.
}\\

\example{ex_line_integrals6}{}{Determine if $\vec F(x,y) = y e^{xy} \vec i + (x e^{xy} + 2y) \vec j$ is conservative. If it is, find a potential function $f(x,y)$ so that $\nabla f = \vec F$.} {For $M(x,y) = y e^{xy}$ and $N(x,y) = x e^{xy} + 2y$, note that both $\dfrac{\partial M}{\partial y}$ and $\dfrac{\partial N}{\partial x}$ equal $e^{xy} + xy e^{xy}$.  Since $M$ and $N$ are defined everywhere and have continuous partial derivatives of every type, it follows that $\vec F$ is conservative.  To find the potential $f(x,y)$, start with $M(x,y) = y e^{xy}$.  We need $f(x,y)$ so that $\dfrac{\partial f}{\partial x} = 2xy$. This would occur if
$$f(x,y) = \int y e^{xy} \: dx + g(y) = y \left( \dfrac{1}{y} e^{xy} \right) + g(y) = e^{xy} + g(y)$$
for some function $g$ depending only on $y$.   Take the partial derivative with respect to $y$ now.  We should have that $\dfrac{\partial f}{\partial y}$ is $N(x,y) = x e^{xy} + 2y$, but we also get from above that
$$\dfrac{\partial f}{\partial y} = e^{xy} + g'(y).$$
Setting these equal, this implies that $g'(y) = 2y$ or that $g(y) = y^2 + C$ for any constant $C$.  Therefore $f(x,y) = e^{xy} + y^2 + C$ is a potential function for $\vec F$.
}\\



Note that this approach of finding $f(x,y)$ should fail if the vector field $\vec F$ is not conservative.  The next example illustrates this for the spin field from a previous example.\\

\example{ex_line_integrals7}{}{Attempt to find a potential function $f(x,y)$ for the vector field $$\vec F(x,y) = -y \vec i + x \vec j.$$}
{Note that this vector field is not conservative as $\dfrac{\partial M}{\partial y} = -1$ while $\dfrac{\partial N}{\partial x} = 1$.  To show how the finding the potential as in the last example results in a problem, we directly solve $\dfrac{\partial f}{\partial x} = M(x,y) = -y$ and try to achieve $\dfrac{\partial f}{\partial y} = x$. First $\dfrac{\partial f}{\partial x} = -y$ gives 
$$f(x,y) = -xy + g(y)$$
for some (possibly constant) function $g(y)$ depending only on $y$. Taking $\dfrac{\partial}{\partial y}$ of this yields $-x + g'(y)$. That does not agree with the requirement that
$\dfrac{\partial f}{\partial y} = x$. There is no way to define $g(y)$ to make this work.  Therefore we conclude that the spin field $-y\vec i + x\vec j$ is not conservative as it has no potential function.}


%15.2 EXERCISES
%Read-through questions
%Work is the a of F dR. Here F is the b and R is
%the c . The d product finds the component of
%in the direction of movement dR = dxi + dyj. The straight
%path (x, y) = (t, 2t) goes from f at t = 0 to g at t =
%1 withdR=dti+ h .TheworkofF=3i+jisjF=dR=
%j i dt= i .
%Another form of d R is T ds, where T is the k vector to
%the path and ds = ,/T. For the path (t, 2t), the unit vector
%Tis m andds= n dt.ForF=3i+j,F*Tdsisstill
%0 dt. This F is the gradient off = P . The change in
%f= 3x +y from (0,O) to (1,2) is q .
%When F = gradf, the dot product F dR is (af/dx)dx + r = df: The work integral from P to Q is j df = s . In this case the work depends on the t but not on the
%u . Around a closed path the work is v . The field is
%called w . F = (1 + y)i + xj is the gradient off = x .
%The work from (0,O) to (1,2) is Y , the change in potential.
%For the spin field S = 2 , the work (does)(does not)
%depend on the path. The path (x, y) = (3 cos t, 3 sin t) is a
%circle with SgdR = A . The work is B around the
%complete circle. Formally jg(x, y)ds is the limit of the sum
%c.
%The four equivalent properties of a conservative field F =
%Mi+ Nj are A: D , B: E , C: F , and D: G . Test D is (passed)(not passed) by F = (y + 1)i + xj. The work
%IF dR around the circle (cos t, sin t) is H . The work on
%the upper semicircle equals the work on I . This field is
%the gradient off = J , so the work to (- 1,0) is K .
%Compute the line integrals in 1-6.
%jcds and jcdy: x = t, y = 2t, 0 6 t < 1.
%fcxds and jcxyds: x=cost, y=sint, O<t<n/2.
%S, xy ds: bent line from (0,O) to (1, 1) to (1,O).
%1, y dx - x dy: any square path, sides of length 3.
%fc dx and jc y dx: any closed circle of radius 3.
%Jc (dsldt) dt: any path of length 5.
%Does if xy dy equal f xy2]:?
%Does jfx dx equal fx2]:?
%Does (jc d~)~ = (IC d~)~ + (fC dy)l?
%Does jc (d~)~ make sense?
%11-16 find the work in moving from (1,O) to (0,l). When F
%is conservative, construct f: choose your own path when F is
%not conservative.
%11 F=i+yj 12 F=yi+j
%17 For which powers n is S/rn a gradient by test D?
%18 For which powers n is R/rn a gradient by test D?
%19 A wire hoop around a vertical circle x2 + z2 = a2 has
%density p = a + z. Find its mass M = pds.
%20 A wire of constant density p lies on the semicircle
%x2 + Y2 = a2, y 3 0. Find its mass M and also its moment
%Mx = 1 py ds. Where is its center of mass 2 = My/M, j = Mx/
%M?
%21 If the density around the circle x2 + y2 = a2 is p = x2, what
%is the mass and where is the center of mass?
%22 Find F dR along the space curve x = t, y = t2, z = t3,
%O<t<l.
%(a) F = grad (xy + xz) (b) F = yi - xj + zk
%23 (a) Find the unit tangent vector T and the speed dsldt
%along the path R = 2t i + t2 j.
%(b) For F = 3xi + 4j, find F T ds using (a) and F dR
%directly.
%(c) What is the work from (2, 1) to (4,4)?
%24 If M(x, y, z)i + N(x, y, z)j is the gradient of f(x, y, z), show
%that none of these functions can depend on z.
%25 Find all gradient fields of the form M(y)i + N(x)j.
%26 Compute the work W(x, y) = j M dx + N dy on the
%straight line path (xt, yt) from t = 0 to t = 1. Test to see if aW/
%ax = M and aWpy = N.
%(a) M = y3, N = 3xy2 (b) M = x3, N = 3yx2
%(c)M=x/y,N=y/x (d)M=ex+Y,N=e"+Y
%27 Find a field F whose work around the unit square (y = 0
%then x = 1 then y = 1 then x = 0) equals 4.
%28 Find a nonconservative F whose work around the unit
%circle x2 + y2 = 1 is zero.
%In 29-34 compute 1 F dR along the straight line R = ti + tj
%and the parabola R = ti + t2j, from (0,O) to (1,l). When F is a
%gradient field, use its potential f (x, y).
%29 F=i-2j 30 F = x2j
%33 F=yi-xj 34 F = (xi + yj)/(x2 + y2 + 1)
%35 For which numbers a and b is F = axyi + (x2 + by)j a
%gradient field?
%36 Compute j - y dx + x dy from (1,O) to (0,l) on the line
%x = 1 - t2, y = t2 and the quarter-circle x = cos 2t, y = sin 2t.
%Example 4 found 1 and n/2 with different parameters. 
%
%Apply the test Nx = My to 37-42. Find f when test D is passed. 43 Around the unit circle find 4 ds and  dx and 8 xds.
%44 True or false, with reason:
%(a) When F = yi the line integral lFedR along a curve
%xi + yj grad xy from P to Q equals the usual area-under the curve.
%39 $F = \frac{xi+yj}{|xi+yj|}$ 40 $F = \frac{grad xy}{|grad xy|}$
%(b) That line integral depends only on P and Q, not on the
%curve.
%41 F=R+S 42 F =(ax + by)i + (cx + dy)j (c) That line integral around the unit circle equals n. 

\printexercises{exercises/14_02_exercises}