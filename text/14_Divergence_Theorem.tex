\section{The Divergence Theorem}\label{sec:div_theorem}

Green's Theorem dealt with a closed curve in two dimensions.  In this section, we look at a closed surface enclosing a solid region in three dimensions.  A surface integral representing the flux through the surface can be written as a triple integral of the enclosed solid.  This is the Divergence Theorem, attributed independently to Gauss and a Russian mathematician Ostrogradsky. First, we define the divergence of a vector field.\\

\definition{def:divergence}{Divergence}{Let $$\vec F(x,y,z) = M(x,y,z) \: \vec i + N(x,y,z) \: \vec j + P(x,y,z) \: \vec k$$ be a vector field.  The divergence $\text{div } \vec F$ of the field $\vec F$ is given by
$$\text{div }\vec F(x,y,z) = \dfrac{\partial M}{\partial x} + \dfrac{\partial N}{\partial y} + \dfrac{\partial P}{\partial z}$$
which is a real-valued function. One can write this as a dot product of $$\nabla = \left\langle \dfrac{\partial}{\partial x}, \dfrac{\partial}{\partial y}, \dfrac{\partial}{\partial z} \right\rangle$$ with the vector field $\vec F$, or as $\text{div } \vec F = \nabla \cdot \vec F$.
}\\

\example{ex_divergence_01}{}{Determine the divergence $\text{div }\vec F$ if the field $\vec F$ is given by $$\vec F(x,y,z) = xz \: \vec i + xyz \: \vec j - y^2 \: \vec k.$$}{We compute
$$\text{div } \vec F = \nabla \cdot \vec F = \dfrac{\partial}{\partial x}(xz) + \dfrac{\partial}{\partial y}(xyz) + \dfrac{\partial}{\partial z}(-y^2) = z + xz.$$
}\\

The reason for the terminology \emph{divergence} relates again to fluid flow. If $\vec F(x,y,z)$ represents the velocity field of a fluid or gas, then $\text{div } \vec F (x_0,y_0,z_0)$ represents the rate of change of the fluid or gas flowing from the point $(x_0,y_0,z_0)$. That is, $\text{div } \vec F$ relates to the tendency of the fluid or gas to diverge from the point in question. If $\text{div } \vec F = 0$ for all points $(x,y,z)$, we say the fluid or gas is \textbf{incompressible}. In other words, when $\text{div } \vec F = 0$, the flow in equals the flow out.\\

For example, the spin fields $$\vec F = -y \: \vec i + x \: \vec j + 0 \: \vec k \: \: \text{ and } \: \: \vec F = 0 \: \vec i - z \: \vec j + y \: \vec k$$
have zero divergence at all points.  The first is a spin around the $z$-axis while the second is a spin around the $x$-axis.  The flow out across a surface will equal the flow into the surface. However, in a radial flow such as $$\vec F = x \: \vec i + y \: vec j + z \: \vec k$$ flow is straight out from the origin.  The divergence is $\text{div } \vec F = 1 + 1 + 1 = 3$ in this case, indicating the tendency of the fluid to diverge from any point.  The Divergence Theorem, stated next, will clarify these situations.\\

\theorem{thm:Div_Thm}{The Divergence Theorem}{Consider a vector field
$$\vec F = M(x,y,z) \: \vec i + N(x,y,z) \: \vec j + P(x,y,z) \: \vec k$$ whose components $M$, $N$< and $P$ have continuous partial derivatives within a region containing a simple solid region $D$ whose boundary surface is $T$. Then the flux of $\vec F$ across $T$ equals the triple integral of the divergence $\text{div } \vec F$ inside $D$.  That is,
$$\iint_T \vec F \cdot \vec n \: dS = \iiint_D \text{div } \vec F \: dV.$$
}\\

\example{ex_divergence_02}{}{Compute the flux of the vector field $$\vec F(x,y,z) = 3xy \: \vec i + x^2 e^{xz^2} \: \vec j + (y + 2z) \: \vec k$$ across the surface of the cylinder $x^2 + y^2 = 1$ between $z = 0$ and $z = 4$.}{Let $D$ be the cylindrical solid and let $T$ be its surface. We employ the Divergence Theorem here to compute $\iint_T \vec F \cdot \vec n \: dS$ via a triple integral over $D$ instead of a surface integral.  The divergence of $\vec F$ is
$$\text{div } \vec F(x,y,z) = 3y + 2$$ and so
$$\iint_T \vec F \cdot \vec n \: dS = \iiint_D 3y + 2 \: dV.$$
The region $D$ is a cylinder, so cylindrical coordinates make sense here.  The triple integral becomes
$$\iint_T \vec F \cdot \vec n \: dS = \int_0^4 \int_0^{2\pi} \int_0^1 (3r \sin\theta + 2)r \: dr \: d\theta \: dz = 8\pi.$$
If we think of $\vec F$ as a fluid or gas flow, the flux being a positive value here indicates that the net flow across the surface is directed outward from the cylinder.
}\\

To see the reasoning behind the Divergence Theorem, consider a small box with center at a point $(x,y,z)$ and edges of length $\Delta x$, $\Delta y$, $\Delta z$.  Out of the top and bottom of the box, the normal vectors are $\vec k$ and $-\vec k$.  The dot product with $\vec F = M \: \vec i + N \: \vec j + P \: \vec k$ is $P$ or $-P$.  So the two fluxes are close to $P(x,y,z + 0.5 \Delta z) \Delta x \: \Delta y$ and $-P(x,y,z-0.5 \Delta z) \Delta x \: \Delta y$.  When the top is combined with the bottom, the difference of those is $\Delta P$:
$$\text{net flux upward} \approx \Delta P \: \Delta x \: \Delta y \approx \dfrac{\partial P}{\partial z} \Delta V.$$
Similarly, the combined flux on the sides and the front and back will be approximately $\dfrac{\partial P}{\partial y} \Delta V$ and $\dfrac{\partial P}{\partial x} \Delta V$, respectively. Adding the six faces, we reach the key point:
$$\text{flux out of the box} \approx \left( \dfrac{\partial M}{\partial x} + \dfrac{\partial N}{\partial y} + \dfrac{\partial P}{\partial z} \right) \: \Delta V$$
which of course is $\left( \text{div } \vec F \right) \: \Delta V$.\\

Note however that the ratio $\dfrac{\Delta P}{\Delta z}$ is not exactly $\dfrac{\partial P}{\partial z}$ - the difference is of order $\Delta z$.  So the difference in the net flux upward is of higher order than $\Delta V \: \Delta z$.  Added over many boxes, this error disappears as $\Delta z$ approaches zero.\\

The sum of $\left( \text{div } \vec F \right) \: \Delta V$ over all the boxes approaches $\iiint_D \text{div } \vec F \: dV$.  On the other side of the equation is a sum of fluxes.  There is $\vec F \cdot \vec n \: \Delta S$ out of the top of one box, plus $\vec F \cdot \vec n \: \Delta S$ out the bottom of the box above.  The first has $\vec n = \vec k$ and the second has $\vec n = -\vec k$.  They cancel each other out - the flow goes from box to box.  This happens every two boxes that meet.  The only fluxes to survive are at the outer surface $T$.  The final step, as $\Delta x, \Delta y, \Delta z$ approach zero, gives that those outside terms approach $\iint \vec F \cdot \vec n \: dS$. This would finish the argument for the Divergence Theorem.  A formal proof would be similar to the one done for Green's Theorem.  The reasoning above, however, is probably more useful than the detailed proof.\\

Another example of flux involves \emph{heat flow} and the rate of heat flow across a surface.  Suppose that $T(x,y,z)$ is the temperature at the point $(x,y,z)$ in a solid or substance. Then the \textbf{heat flow} is defined as the vector field
$$\vec F = -K \nabla T$$
where $K$ is a constant called the \textbf{conductivity} of the substance. The rate of heat flow across the surface of a simple solid $D$ in the substance is given by the flux integral
$$\iint \vec F \cdot \vec n \: dS = -K \iint \nabla T \cdot \vec n \: dS = -K \iiint_D \text{div } \nabla T \: dS$$
by the Divergence Theorem.\\

\example{ex_divergence_03}{Heat Flow}{Suppose that $T(x,y,z) = x^2 + y^2 + z^2$ is the temperature of a solid sphere centered at the origin of radius $2$.  Determine the heat flow across the surface of the sphere if the conductivity is the constant $K$.}{The heat flow is the vector field
$$\vec F = -K \nabla T = -2K(x \: \vec i + y \: \vec j + z \: \vec k)$$
and so the heat flow across the surface of the sphere is
$$\iint \vec F \cdot \vec n \: dS = -2K \iint (x \: \vec i + y \: \vec j + z \: \vec k) \cdot \vec n \: dS$$
which we can evaluate using the Divergence Theorem.  This integral becomes
$$\iint \vec F \cdot \vec n \: dS = -2K \iiint_D (1+1+1) \: dV$$
where $D$ is the sphere.  Noting that the volume of the sphere is $\dfrac{4}{3} \pi (2)^3$ cubic units, we get a heat flow of $-2K(3)\left( \dfrac{4}{3} \pi (2)^3 \right) = -64K \pi$. The flux is negative, and so heat is flowing inward, which makes sense since the surface of the sphere is warmer than the inside.
}\\

In the next final section, we look at the case of a surface which is not closed.  In such a case, the boundary of the surface is a curve in three-dimensional space, and the flux across the surface can be evaluated using a line integral around the closed curve.  This is yet another generalization of Green's Theorem called Stokes' Theorem.


\printexercises{exercises/14_05_exercises}