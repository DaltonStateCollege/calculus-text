\section{Stokes' Theorem and Curl}\label{sec:stokes_theorem}

In the last section, the Divergence Theorem gave us a generalization of Green's Theorem to a closed surface. In this section we consider a surface which is not closed, and hence has a boundary.  This boundary will be a curve in three-dimensional space. This puts us back in the setting of Green's Theorem - a region bounded by a curve. Stokes' Theorem will be this generalization of Green's Theorem to three-dimensional space. Instead of a region in the plane, the region will be a surface in three-dimensional space.\\

Consider a three-dimensional vector field $$\vec F(x,y,z) = M(x,y,z) \vec i + N(x,y,z) \vec j + P(x,y,z) \vec k$$ In Green's Theorem, the vector field had only two dimensions and the line integral around the curve turned into a double integral of the scalar function $\dfrac{\partial N}{\partial x} - \dfrac{\partial M}{\partial y}$.  In three-dimensions, this function changes to what is known as the \emph{curl} of the vector field. \\

\definition{def:curl}{Curl}{Let $$\vec F(x,y,z) = M(x,y,z) \: \vec i + N(x,y,z) \: \vec j + P(x,y,z) \: \vec k$$ be a vector field.  The curl $\text{curl } \vec F$ of the field $\vec F$ is given by
$$\text{curl }\vec F(x,y,z) = \left( \dfrac{\partial P}{\partial y} - \dfrac{\partial N}{\partial z} \right) \vec i + \left( \dfrac{\partial M}{\partial z} - \dfrac{\partial P}{\partial x} \right) \vec j + \left( \dfrac{\partial N}{\partial x} - \dfrac{\partial M}{\partial y} \right) \vec k$$
which is a real-valued function. An easier way to remember this is as the cross product of $$\nabla = \left\langle \dfrac{\partial}{\partial x}, \dfrac{\partial}{\partial y}, \dfrac{\partial}{\partial z} \right\rangle$$ with the vector field $\vec F$, or as 
$$\text{curl } \vec F = \nabla \times \vec F = \left| \begin{array}{ccc} \vec i & \vec j & \vec k \\ & & \\ \dfrac{\partial}{\partial x} & \dfrac{\partial}{\partial y} & \dfrac{\partial}{\partial z} \\ &  & \\ M & N & P \end{array} \right|$$
}\\

Note that if $P(x,y,z) = 0$ above and we have a two-dimensional vector field with $M(x,y,z) = M(x,y)$ and $N(x,y,z) = N(x,y)$ in the plane not depending on $z$, then $\dfrac{\partial M}{\partial z} = \dfrac{\partial N}{\partial z} = 0$ and
$$\text{curl } \vec F = \left| \begin{array}{ccc} \vec i & \vec j & \vec k \\ & & \\ \dfrac{\partial}{\partial x} & \dfrac{\partial}{\partial y} & \dfrac{\partial}{\partial z} \\ &  & \\ M(x,y) & N(x,y) & 0 \end{array} \right| = \left(\dfrac{\partial N}{\partial x} - \dfrac{\partial M}{\partial y} \right) \vec k$$
as in Green's Theorem.  That is, in this case we have the restatement of Green's Theorem as
$$\oint_C M(x,y) \: dx + N(x,y) \: dy = \iint_R \text{curl } \vec F \cdot \vec n \: dA$$
where $C$ is the closed boundary curve of the region $R$ in the plane.\\

\example{ex_curl_01}{}{Determine the curl of the spin field $$\vec F(x,y,z) = (z-y) \vec i + (x-z) \vec j + (y-x) \vec k$$ which has an axis of rotation given by the vector $\vec a = (1,1,1)$.}{The curl of $\vec F$ is given by the cross product
\begin{eqnarray*}
\text{curl } F & = & \nabla \times \vec F = \left| \begin{array}{ccc} \vec i & \vec j & \vec k \\ & & \\ \dfrac{\partial}{\partial x} & \dfrac{\partial}{\partial y} & \dfrac{\partial}{\partial z} \\ &  & \\ (z-y) & (x-z) & (y-x) \end{array} \right| \\
& = & (1 + 1) \vec i - (-1-1) \vec j + (1+1) \vec k = (2,2,2) = 2 \vec a
\end{eqnarray*}
This result generalizes. The curl of a spin field about an axis of rotation given by the vector $\vec a$ is in the direction of the axis of rotation. This gives us our first clue as to what the curl of a vector field represents - it measures the spin.
}\\

We have spent a significant amount of time talking about conservative vector fields in this chapter.  What is the curl of such a vector field, one which is the gradient of a scalar function?  Consider the potential function $f(x,y,z)$ with continuous derivatives with gradient vector field $$\vec F = \nabla f(x,y,z) = \dfrac{\partial f}{\partial x} \vec i + \dfrac{\partial f}{\partial y} \vec j + \dfrac{\partial f}{\partial z} \vec k$$
Then the curl of $\vec F$ is
\begin{eqnarray*}
\text{curl } \vec F & = & \left| \begin{array}{ccc} \vec i & \vec j & \vec k \\ & & \\ \dfrac{\partial}{\partial x} & \dfrac{\partial}{\partial y} & \dfrac{\partial}{\partial z} \\ &  & \\ \dfrac{\partial f}{\partial x} & \dfrac{\partial f}{\partial y} & \dfrac{\partial f}{\partial z} \end{array} \right| \\
& = & \left( \dfrac{\partial}{\partial y} \dfrac{\partial f}{\partial z} - \dfrac{\partial}{\partial z} \dfrac{\partial f}{\partial y} \right) \vec i + \left( \dfrac{\partial}{\partial z} \dfrac{\partial f}{\partial x} - \dfrac{\partial}{\partial x} \dfrac{\partial f}{\partial z} \right) \right) \vec j + \left( \dfrac{\partial}{\partial x} \dfrac{\partial f}{\partial y} - \dfrac{\partial}{\partial y} \dfrac{\partial f}{\partial x} \right) \right) \vec k
\end{eqnarray*}
Since $f_{yz} = f_{zy}$, $f_{xz} = f_{zx}$, and $f_{xy} = f_{yx}$, we arrive at $\text{curl } \vec F = \vec 0$.  Therefore the curl of any conservative vector field is zero.\\

Similarly, if $\vec F$ is the curl of another vector field, say $\vec F = \text{curl } \vec G$, then $\text{div } \vec F = \text{div } \text{curl } \vec G = 0$.  That is, the divergence of a curl field is zero.  We summarize these last two results in the following theorem.

\theorem{thm:Div_Thm}{}{Consider a vector field
$$\vec F = M(x,y,z) \: \vec i + N(x,y,z) \: \vec j + P(x,y,z) \: \vec k$$ whose components $M$, $N$< and $P$ have continuous partial derivatives within a region containing a simple solid region $D$ whose boundary surface is $T$. Then the flux of $\vec F$ across $T$ equals the triple integral of the divergence $\text{div } \vec F$ inside $D$.  That is,
$$\iint_T \vec F \cdot \vec n \: dS = \iiint_D \text{div } \vec F \: dV.$$
}\\


\printexercises{exercises/14_06_exercises}